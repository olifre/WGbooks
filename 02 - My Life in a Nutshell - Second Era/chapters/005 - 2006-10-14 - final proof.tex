\chapter{Final Proof}
\label{cha:final-proof}
\subsection*{Originally published: \DTMdate{2006-10-14}}
\begin{quote}
Hi! 

This time, loads of things had happened. Things that could prove my theory, and development of the same. OliFre is helping me to build it into the current scheme of theories so it might be accepted by the scientific world though it sounds really improbable. 
And, in addition to that, he is building himself a new homepage; If he succeeds in the next months, he may also move my page there. Still, you will be able to access it by using the normal links. Be patient!

Finding the final proof\ldots
\end{quote}

He had found it. 
The Final Proof, on that special evening. 
Y. had not been there, but that was not the major speciality.

O.'s friend wasn't there, and he wouldn't return. Why had this happened? Of course, it could only happen because this was the moment when he had least believed in their separation. For him, this was the Final Proof; In addition to that thinking, she had also been so distant towards him he could not believe it. G. had shared part of this inexplicable distance, but he suddenly knew the reasons for all of it. He could not only see through people's minds, but also through the mind of the world, and this vision seemed to become clearer every minute.

This night, even, he had dreamt of her. Dreamt of the old, long-gone times when the two of them were in the same bus; But her behaviour had changed, and she was nearly as distant as she was now in reality. He had approached her and talked to her in this dream, not the other way round; And finally, she'd blocked the place next to her so he could only sit down behind O. He wasn't sure whether silence followed, as his memory of this dream ceased to give him more information at this point; A misty cloud was in his mind, fogging O.'s existence --- and his own.
In reality, she had become a bit more friendly, probably having dealt with her loss now; but she wasn't really \emph{caring} for him the way she had done before.

He decided to ignore all these revived feelings, leaving her alone so as to free the way for the decision of fate; Then, if things seemed inevitable, he could consciously follow that path, whether it meant unity --- or separation.

Something else had really troubled him some days before, something he'd never experienced, not even in films or books; Something he had always feared to happen without knowing what it really was. 
One morning, L.-B. had approached him right after the lesson and said \enquote{We must talk.} The sentence everybody said when there was trouble lying ahead --- and there was. Trouble for his already tortured soul. 
He had already noticed that L.-B. suddenly seemed to ignore him, and the phone calls had stopped; Since that moment when he'd most offensively ignored her in the library. He'd felt her mind \emph{break}, but P., sitting next to him, seemed not to realize, being absorbed by something else.

Without saying a word of goodbye, L.-B. suddenly stood up and was gone; He had pretended to be absorbed by something else, too. Several days later, he'd talked to P. about L.-B.'s sudden ignorance and expressed his happiness about it, but he felt that something was going on. P. didn't, and remarked that L.-B. was probably thinking \emph{them} to be a couple now; She smiled saying so, and he knew she was joking, but he decided to accept this explanation for there was no other for the moment.

That Friday, one week before Friday the thirteenth, he was given the explanation. L.-B. was with him, telling him she had the feeling he was in some way ignorant towards her. 
Her eyes were glistening, and he knew there was a terrible sadness in there. No, not a terrible sadness --- these eyes seemed hurt and rejected. He knew this would happen some day, but he'd never promised ANYTHING to her.

He was silent. 
She was talking.

Nervously moving her head from here to there, she told him that this feeling was giving her a hard time --- then, when some students both of them knew passed by, she was smiling, and only a sensitive person could see what was going on --- but there was nobody around here who could sense such a profiling system, with the exception of him. Such a system was inside everybody, and she had just activated it, very likely without realizing it. He knew what was going on, and she continued talking when they were \enquote{alone} once more.

He sensed a single tear escaping her right eye and hitting the ground, but she ignored that, trying to hide that feelings, escaping his look. 
He decided he was to say something --- quickly. \enquote{I would not have noticed}, was the best he could think of. That meant that from her point of view, the ignorance could be tracked down to his unconsciousness; And that was the very place where it came from, for he didn't love her.

So, concerning the cause of it, he hadn't even lied. The first time that sentence ran over his lips, she didn't even notice. This was something he couldn't understand, for he'd always thought that love made you the most attentive person concerning the beloved. He'd noticed that ignorance with her more than once, and in a way, he was mirroring it back. It had made him realize that this love of her's was some kind of childish fondness, and no love at all.

His and her English professor passed by, touching her shoulder and wishing her a happy day and nice holidays --- Her profiling system was activated once more, she smiled and thanked her. Even this professor didn't notice a thing, and he had the terrible vision in his mind of her realizing that L.-B. was --- crying, though she did so silently. What would L.-B. have said, and how would the professor have reacted?

In another line of time, this may have happened, but he was happy he was right here, and his awareness of the high probability of the normally sensitive professor to notice that fact had probably made him continue his life in this timeline.

As the professor was gone, he'd repeated his sentence, and this time, she'd noticed it. But she had not understood. He wondered whether love was not only blinding, but also deafening --- But normally, this was a result of the raised attentiveness towards the beloved, and thus should --- though it would affect all senses --- not affect the intercourse with the beloved.

He repeated his sentence, a magical third time. She seemed to understand, and while his face remained unmoved, she became more silent, keeping herself together even stronger and then trying a weak smile. Maybe, her hand was even taking away the tears, but he wasn't aware of that right now. Why had P. already gone ahead instead of waiting? Because she wanted to reserve a place in the library for him. To do him a favour.

And there it was, that \enquote{favour}. She asked him \enquote{So, can we now be --- friends again?} That childish sentence. 
No, this certainly wasn't real love. It was childish fondness, but it hurt her nevertheless.

He accepted her offer, and she asked him whether she was allowed to phone him once more in the holidays. Before, she'd admitted her fear of phoning him --- A fear he could not really understand \emph{without} taking her strong feelings for him into account, even if that were only feelings of fondness.

They went down to the library together, she sat down some eight metres away from him, as he took his seat next to P. 
She would approach him once more that day, to wish him some nice weeks of holiday and repeat her (threat?) to phone him. 
He couldn't tell P. right now about that. Not while L.-B. was sitting only some metres away, and not without having thought about all that for some time. 
He'd tell her later, maybe, probably by sending her this extract of his story --- his life.

When he thought about all that had happened up to now, the only person that could come to know more about him than P. was O., and she didn't show any real interest in gaining such knowledge. But there were still most things left to himself and those that read his stories.

He wondered what was to happen if he printed this story one day and sent it to O.; He was decided to do so in some years, though he was pretty sure she wouldn't even read it. She was not the \enquote{big reader}, and especially wouldn't finish such long stories.

His best friend had come a bit closer to him now, from his own point of view --- But he couldn't ignore the changes that had taken hold of him, and others were also mentioning it --- if he wasn't there. 
He hadn't given her such a big amount of presents this year; Last year, she had received some books, but he was sure she hadn't even read them. O. was not the \enquote{big reader}, though he'd thought so when he'd two years ago tried to read one book along with her, and finished another one she was reading (without finishing it) for ages. He was still wondering why she had allowed him to tell her the end of that story.

Maybe, she wanted to get rid of that memories, of these connections between them, for this was the time when she was together with her now-gone boyfriend. 
We should now try to lean back, enjoying some rest from work by working, for we can'd do anything else. More is to come, but be patient! 

\begin{verse}
Love is a wave \\
blinding your senses \\
towards the world --- \\
only to make them more attentive \\
towards the beloved. \\
--- W.G.
\end{verse}

\begin{verse}
Let me paint a contrast --- \\
go far, far away \\
and see it's all the same: \\
Each difference shrinks away \\
to nothing \\
in the distance. \\
--- W.G.
\end{verse}

%%% Local Variables:
%%% mode: latex
%%% LaTeX-command: "latex -shell-escape -synctex=1 --file-line-error-style"
%%% ispell-local-dictionary: "british-ize-w_accents"
%%% TeX-master: "../My_Life_in_a_Nutshell_-_Second_Era.tex"
%%% End:
