Hi! 

I was just caught in a process of endless work, not only for university, but also concerning the eternal turn of the wheels of my mind. Recapitulating what had been, and what was to come, and endless tragedy seemed to manifest itself, nevertheless giving an outlook that might be far more beautiful. Nothing was to be seen yet --- and who but chance could tell the truth? 

[u]Final fade of forlorn remembrance?[/u]
\textbf{[u]Unipolar Depression?[/u]}
[quote]
Fancy pictures 
having flown through a mind --- 
leaving traces, scars 
signs that forever remind of lost wars; 
Why do we aim to win? 
Scars teach us more than a sin, 
and the power achieved by their healing 
is the strongest weapon of peace. 
[/quote]
Today, the most extreme feeling ever had approached him: He was in an exam, and not completely sure whether O. --- the person he'd thought to be O. --- would also be there. Some weeks ago he would have known for sure, but now, the first pieces of information were fading. 
It was after some time that he found her, and the thought that shot through his mind when seeing her made his heart stumble --- he had not only forgotten her, this was more than that. 
He was in the position he had wanted to be in before: Hating or at least scorning her character, regarding the once beloved with despise. 
And that was the position he was in right now: As he had seen her that day, some feeling was growing inside his mind. And a word was there, a subconscious idea he could neither deny nor pronounce. 
When he looked at her, it seemed there was too much skin and her former vividness seemed to have transformed into selfish enjoyment: And the boyfriend fit totally with that image, only that he didn't hide this character of his --- at least, when with boys. 
It was no tragic loss anymore, as he had gained so much more than he had lost. Even if the time he had been contemplating may have been the outcome of a unipolar depression, he had realized that most things we call "mental diseases" are in fact ways of the beautiful and often misunderstood machine we call a brain to deal with the problems we cannot accept so easily. However, the question still remains whether happiness may be the result of knowledge or ignorance --- or, whether we'll finally realize that it is the same. 
L.-B. was still there, stealing his time in more and more oppressive ways. But this was not the only thing grabbing his mind --- He had finally decided to take the chance to learn the thing that had long ago made him realize that there was a big difference between G. and O.: Dancing. According to the daughter of the teacher, he seemed to be talented --- Just to remind you: G. had been a natural dancer, while O. had just done as she had been teached. One could see that when looking at these two, and it had made him think that in O. there was more sense, while G. was sensibility. 
With O., he had been proven wrong, but with G., he seemed to have been right --- She was still smiling at him for a greeting, and though she lacked a certain degree of attention (as most people did), there was still a more subtle attraction emanating from her than from O. His having been crazy about her had certainly not been based on fantasy only. But this had gone, and all there was to remain was an aura of friendship. Most interestingly, she had used the word aura that day while talking to him. 
Yes, some signs were still out there, and he couldn't help but wait for something to happen. 
However, one thought was eternally hunting through his mind --- one single idea, one virtual thing he couldn't follow but could not forget --- What if he changed his view on reality, imaginating the real O. being with him, invisible to the others, but as vivid and perfect to his senses as he had always thought her to be? 
No, he couldn't do it --- Even if he would succeed in betraying his senses, he would not be able to face that decision to give up his perspective upon reality. Nobody could live with an image for a long time, even if that was to be perfect. 
But he had found another way, a way that would take a long time, but may be worth all that energy he'd have to invest. He could construct a virtual image of her, a picture seemingly alive, but nevertheless being perfectly virtual. The process of making that image would help him deal with all that had gone by, but he'd need time he didn't have. 
He'd have to crack this nutshell open himself, but only for him --- he'd have to find the core of it. 
And so he sat there, looking around, working, thinking, typing. Typing the words that he felt. 
It was a mindless race his thoughts were taking part in. Up and down, to and fro, swimming without control or visible aim; yet, he was able to realize that there was an aim he couldn't see. 
He had to wait, looking forward to an unknown destiny, facing hard times or something completely different --- the story of life would continue. 
The rush of endless chaos wouldn't stop in a minute --- the final countdown would (hopefully?) take a little longer. 
We shall see what comes next, if something is to come at all --- new pages are waiting, and time is still rare. 
To be continued...
This month, by Wilkie Goldentongue...

--------------------------------------------------------------
(If you want to comment on this, please start a new thread; this one is reserved for this story, my life, the story that'll never end, while the end is still near, coming nearer every minute; it's just around the corner...)

A rainbow 
is a sign of red love, 
of orange expectance, 
of yellow vividness, 
of green hope, 
of blue wisdom 
and of violet depth. 

May it also be 
as red as blood, 
as orange as betrayal, 
as yellow as a burning sun, 
as green as a poisonous frog, 
as blue as cold and suffocating water 
and as violet as a reign of terror. 

Yes, a rainbow may be all of it. 
A rainbow is the bridge of life, 
and as its symbol, it must contain 
all the facets of life 
there are. 
\emph{W.G.}

A stone broke out of the wall --- 
with a loud 'bang' it hit the ground, 
and finally, there was a hole 
in front of him. 
A hole to pass through, 
a path to continue. 
But the stone was gone, 
never to return; 
Should one wait and contemplate 
or continue to tread the path of wisdom? 
\emph{W.G.}
