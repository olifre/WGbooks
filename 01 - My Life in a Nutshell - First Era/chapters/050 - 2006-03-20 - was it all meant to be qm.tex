\chapter{Was it all meant to be?}
\label{cha:was-it-all-meant-to-be}
\subsection*{Originally published: \DTMDate{2006-03-20}}
\begin{quote}
Hello! That's me again...

Today has been a day full of new, interesting things. There had been a lot of time to explore, but time for writing was limited, again. However, I'll try to focus on the most important things and hope you share my opinion on that. 

He was wondering: Was it all meant to be?
\end{quote}

Monday; once more, a day when he \emph{should} see her. However, he'd felt that something would be different that day, and in fact, fate wouldn't prove him to be wrong. 
The lecture he should have had together with her was cancelled.

She stood there, talking to people he didn't know, while he was reading this fact; in a sudden surge of energy, he managed to walk up to her, touch her on her shoulder and tell her about these news. 
She was happy, and she began jumping; then, she said goodbye to the people she'd talked to and went with him; he told her that they'd probably be together in the next lecture, as there would be some project, but he didn't know for sure, and she didn't seem to care, really.

And the sudden happiness of him being together with her wouldn't last much longer than this moment, as they were heading to other people who hadn't read these news yet. Well, she informed them, and he could just smile and nod, assuring all these guys. When he was heading to the next board, looking for more information, she was suddenly gone, and he was not about to search her; she'd taken the initiative before, calling him to join her on her way from the auditorium to --- well, somewhere else. And now, she had been absorbed by other people, having dumped him; he was getting used to that, as she'd done so pretty often, at least, since she was together with her current boyfriend.

Some time was passing; he was together with P. and her friend again, and pretty happy in that company. Later that day, somebody would ask him whether P. was interested in a relationship with him or not; he answered truthfully that he didn't know. When asked for his own perspective, he answered the same, which was true, too.

He didn't know a thing, it occured to him. However, he'd still be together with so many people; He knew them pretty well, but he didn't know what they were really thinking about him. Thus, it had also come as a shock when he'd heard from somebody that his best friend was getting tired of what he kept doing; of course, he'd known that this would happen; but he'd never believed that he'd tell anybody, or at least, that he'd tell him first. The only thing that reassured him was the fact that he seemed to feel guilty whenever any inquiry on that topic was done by our protagonist.

I notice that I've just started to talk about something else while I was trying to tell you about O.; probably, this is a sign of success; probably, it's not. He didn't really care right now. 
But the day was to continue; it was not yet finished. He was to talk to several professors, and all of them would be happy to do so; he was friends with everybody, of every age and generation, of course. Sometimes, he'd feel like the person who knew everything at this university; but he knew that he knew nothing about the things only a person itself could know about.

He felt cold, and began to shiver, though it was pretty warm; suddenly, he felt he was lonely, though he was deeply integrated in society.

L.-B. had been with him for some longer moment that day, as she'd asked him for help, and he'd never denied such a proposal from nobody. Probably, he couldn't do so, but he tried to escape, as she'd wished to go somewhere where P. couldn't watch the two of them --- somewhere, where she'd have been alone with him. He'd managed to prevent this from happening, though he hadn't really explained her \emph{why} he'd wished \emph{not} to got there; as persons who were in love would never allow themselves to bother the people they loved, she had to accept, as soon as she'd notice that he was bothered. Gladly, she'd noticed this time, as she didn't seem to be able to make out such --- sometimes subtle, sometimes apparent --- moods.

He knew how to treat her, but she didn't understand the messages he'd send. The moment when she was criticizing P. was one of the singular times when he'd really tell somebody about his real opinion: That P. \emph{didn't} go on his nerves, and that she wasn't \emph{that} bad a person at all. She decided to change the topic then, but he'd already learnt what to make of that low noises she'd make now and then; he could notice how she dealt with the things she heard, and she was the perfect person to learn about human interaction, as everything seemed apparent with her. The problem was, that she was too simple; of course, he'd never fear that she'd read these texts, and even if she would do so, she'd never find out that she was L.-B. Though she liked reading, he knew that she was different; that black part of her aura was missing, and though she could simply deal with other people, she'd never go further than some special point; all of her relationships seemed shallow. When you'd talk to her for two minutes, you coudl realize that, and when you knew about her, you could be sure that she'd never had a boyfriend, and, in fact, you'd wonder if she'd ever be together with anybody you could make friends with.

He could hear the radio in the background while he was writing; he wondered whether it would really do him something good if he listened to a German station; would it help him to learn any German? Well, the radio was pretty low, so he could still concentrate on this here; but he was prepared to record anything special that would come in via the receiver, and he'd use a method of continuous buffering so as to be ablo to record something \emph{after} it had been sent. Technical devices were really interesting these days; but life was more complex. This was the method to keep harmony: When life would become more comfortable by progress, social interaction would be the more complicated, as the topics one could know about had increased wildly, which meant that two experts could join each other, and the only knowledge they'd have in common was the weather. Thus, this was one of the most used tactics to make friends with somebody: Talking about something everybody knew about. But if we'd all be specialized on something, such things would become rare; progress was indeed destroying our social interaction in many ways.

But we shall continue with the story: He was in the library, with P. and her friend. Y. and her friend would soon walk up to him, asking whether he could join them for some minutes later on to help them with some work, the same L.-B. had already done and would now want him to check. He agreed, of course, and he was much happier to do so than he'd been with L.-B., as she'd phoned him yesterday without any real reason, just asking whether he could spare a minute \emph{only} for her, to have a look at this work. She forced him to correct it with a pen, but he could escape her offer to bring something for him in return. This offer was so strong it was close to being rude, and he could really make out the sound in her voice when she'd greeted another girl walking up to them; it was one of her single 'friends'. She would not greet her happily, though she was smiling; L.-B. seemed embarrassed about her presence, but this girl wouldn't go away, and he was happy about that, as it made things easier to keep L.-B. away. She had always turned towards him, while he'd looked the other way, and he had managed to escape the touch of her legs in an astounding effort. No, men were \emph{not} always searching for that single thing everybody was talking about. Even those 'primitive creatures' would be proud and select carefully. At least, he did, and L.-B. wasn't selected.

However, he wondered whether tings would develop in a similar way to these films, where the protagonist didn't like the girl which had always loved him; and finally, they'd end up as couple. But he felt it wouldn't be this way with L.-B., as those people had never realized the love of the girls they'd finally end up with.

He had, and he was happy that he could prevent her from putting it into practice. When she'd phoned him, he'd finally managed to get rid of her quite quickly; and once, he'd even had a philosophic discussion with her, and he'd known she wouldn't be able to take part in it, really; she'd tried, and finally, she had to accept his position as she hadn't known what she could do otherwise.

He was together with P. again when O. and B.-B. would enter the library. They'd greet him politely, while this greeting could also have applied to anybody else there, but he was the only person who'd really shown any reaction. They were searching for some book, and he \emph{could} have helped them, while he only told them where it \emph{should} have been. B.-B. found something quite old-fashioned; while she was searching, O. was half-standing, half-sitting on a table, in a way you could see her profile very well. She was still beautiful, and in his mind's eye, things could have gone differently\ldots

But it was too late now. And, in fact, it could all have been an illusion. He managed to talk to P. and look at O. at the same time without any of them noticing it; he could do many things like these, as a short flicker of an eye could be interpreted as a nervous reaction or a sign of a lack of sleep; nobody would believe that anybody could just take a look at somebody else. And even looking somewhere else --- preferably, at some point in the air behind that person --- was to be managed easily, as nobody would like to show you how rude you were not to look into his / her eyes directly; and some people never did so. He could, but he could also do without if that did him some good.

When B.-B. and O. were leaving, he felt he could have shown them whatever information they were searching for on the computer; but they \emph{could} also have asked him, though they'd probably not realized that the internet could also contain such information. On the other hand, P. was sitting next to him, and he felt it was too late --- again. This morning, he'd also taken something that had fallen to the floor, giving it to somebody else who would have had to bow herself; he didn't know whether things like these were signs, possibilities, chances --- or nothing.

But he knew that the world was based on a wave, on many waves, which would join in interference and harmony; the strings, which would probably be the base of the universe, of all matter, could probably transfer that frequence to the macrocosm, to our life --- who'd know as long as the theory wasn't completed? Who could tell if all those things that happened --- which were indeed arranged in the form of a wave --- were based on that frequence that would also make matter exist? Who could explain the result of the existence of all that matter we couldn't characterize? Would there be some interaction with the thing we called 'consciousness'? Was consciousness itself a result of that wave, and was the interference or the harmony of it the basis of everything we feeled and conceived?

Who could tell? He remembered the time when he had been pretty small, playing games on the computer, and changing the games when he longed for something else; some system had developed itself, and in the sort of a wave he'd repeatedly liked to play this now and that then, followed by this again, concluding with the other thing. Then, he'd first noticed that foundation of life on a wave, and he'd learned that he should do some logging on his likes and dislikes over time; this was the first moment he'd wished to do scientific research. But before he could do so, he had ceased to play such games after the system of the wave had stopped to flow, beginning to be chaotic before it had finally broke completely.
The day was still not finished.

He'd see B.-B. again, when she'd bring that book back; O. wasn't with her, and B.-B. just smiled for a greeting. Finally, this was the time he'd noticed that the two of them had really found such an old book and taken it, as he hadn't watched B.-B. so closely before. His focus had been on O., once more.

This day, he'd also been asked which kind of girl he'd like; he hadn't really answered, explaining that people who had an image clear in mind would always end up with the opposite --- which was true, in fact.
But we're not to discuss on that facts of life; you can find out about these on your own.

Today, he'd learned that the boy who had been after P. was now together with his former girlfriend again, in exactly the way P. had predicted --- and hoped for --- it. She'd shaken our protagonist today when she'd learned about the lecture being cancelled --- that was P. alive.

The next things we should focus on have happened in the late afternoon and in the evening. Well, he had been granted the possibility to talk to P.'s friend again before she'd have to go to catch the bus; later, he'd seen her again, but he wasn't sure whether she'd missed the bus which he had to catch; just because she'd liked to talk to him. Well, he wasn't equal to that, but he couldn't do a thing about it now.

In addition to that, he'd also been able to wave at B.-B. while she was driving home in her car; next, he could wave at G., while O. was sitting in the back; at first, he'd wondered where she was, till he'd realized that she'd raised her hand, too; in the front, there was this girl he'd often met in the bus the last weeks.
Well, we can't go into detail right now; time's running out.

But I don't want to keep this from you: He'd also been together with Y. and her friend to help the two of them, and he'd felt guilty as he'd let the two of them wait. Then, he'd also organized many things that day; he had been visited in the evening, and some old \enquote{friend} --- he was not so sure about that word --- asked him for something that was a normal favour everybody would have done. He agreed, of course --- he wouldn't argue about things like these even with people he didn't know.

Then, just some minutes after G. and O. had gone, he'd seen O.'s best friend, and the two greeted each other by waving two times; he'd also met that girl he'd once talked to in a lecture again, in the room where that painting was located that reminded him of O.; it was still impressive, and he'd just exchanged one look with that girl, leaving the two of them smiling without any reason to do so. 
This day had been so long, and now, we're bound to finish this text, though it's story is still not told completely. So many details have been forgotten, so many things summed up in short sentences\ldots
We'll probably have a look at them at some later time, if you decide to join me again\ldots
Any opinions?

\begin{verse}
Today, \\
when he was thinking about the interference of the waves of life, \\
he'd seen \\
that everything \\
was meant to be. \\
When he opened the book in front of him, \\
randomly, \\
G.'s colour was to be seen; \\
quite exactly her colour. \\
And the topic of that pages was \\
interference. \\
Shall we search for a meaning \\
in what is meant to be? \\
Or isn't that meant to be? \\
--- W.G.
\end{verse}

\begin{verse}
Cruelty \\
is the basis of joy. \\
Pain \\
is the expression of hope. \\
For without pain --- \\
or the vision of it --- \\
we'd never hope. \\
And without the experience \\
of cruelty \\
we'd never be capable \\
of feeling happy. \\
--- W.G.
\end{verse}

%%% Local Variables:
%%% mode: latex
%%% LaTeX-command: "latex -shell-escape -synctex=1 --file-line-error-style"
%%% ispell-local-dictionary: "british-ize-w_accents"
%%% TeX-master: "../My_Life_in_a_Nutshell_-_First_Era.tex"
%%% End:
