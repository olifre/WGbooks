\chapter{More Signs --- More Fear}
\label{cha:more-signs-more-fear}
\subsection*{Originally published: \DTMDate{2006-01-21}}
\begin{quote}
Hello; I hope you grow fond of this story, and start to understand my feelings. 

This time, I'll tell you the other signs that were just jumping at me today; they're making me afraid and give me a hard time, as I don't know what to do or think about it. 
\end{quote}

It was Saturday --- A day he wouldn't see her, but a day full of new signs approaching his mind.

It all started at about 4 pm: He approached a radio his mother hadn't turned of; it somehow attracted him with a magnetic power. He stared at the volume level: It matched HER number. Later, her mother would turn the radio on again, and it wouldn't match this number again; and it wouldn't be matched so exactly, showing it so clearly, as if it was intended to do so.

A new song was heard, and it told him to wait for summer and start over then, in a kind of metaphorical way; when the song was finished, he knew he could turn of the radio safely as no more information would be given to him. He turned around and saw the antenna of another radio, its tip just in front of the left eye of a portrait that showed him. His mother had hung it there. The tip of the antenna seemed to touch the right eye of the image, or at least cast a shadow upon it; and he couldn't help but remember a scientific extract about the method the human brain was working.

The right halfth dealed with images, as there are pictures --- and emotions.
The left halfth would be the logical part, analyzing things systematically and mathematically by using numbers.

The eyes and nearly the complete halfth of a man's body would be \enquote{wired} the other way round; thus, the left part of the body would be controlled by the right section of the brain and vice versa. 
The right eye of the portrait (heraldically speaking) was thus blocked by the antenna; thus, the left part of the brain, the logical part, was blocked.

This sign could mean two things: Either, he should try to think logically, trying to remove that blockage, or it could be exactly the opposite. 
This duality or ambiguity of meaning was something quite natural to life: It was the only thing left to choose between, as everything else was borne --- or predetermined.
This was the main reason that caused the discussion about the illusion of decision, and the final cause of that discussion not to be able to give a right answer. 
Most of these answers to the main questions of life were to be found somewhere in between; only change was to be found everywhere, that was sure.

The Chinese had known that for centuries, even several thousands of years ago; today, we just happen to know about the Yin-Yang-Symbol, but people who could tell you about the philosophy that it symbolizes are rare. 
The Feng-Shui and things like these are just marketing ideas, and are not to be taken seriously; No philosophy is ever to be sold. 
But now, we'll have a look at the next sign that was approaching him: He was just finishing some work for university, finally counting the words he'd used on his computer.

The number he received matched the number on the plate of the car the parents of O. owned. 
However, after he'd printed the homework, he found he'd forgotten something, added that and thus increased the amount of words by ten, finally getting the checksum of G.'s name. 
The meaning of this sign puzzled him, as he couldn't think of one. 
Or did the repitition of feelings also include the emotions he'd once harnessed for G.?
He didn't know, and he didn't want to find out, either.

Having a look at the temperature, summing the temperature outside and inside, like he always did, finally calculating the checksum, he found it rose from two up to four. 
Two was her number, three the number of her boyfriend, and four his number. 
This was the next puzzling sign, but he decided not to try finding out about the meaning about that one, too. 
Right now, he was just walking into the kitchen again, not intending to check the temperature; he did so by chance. The checksum was two, as it always was when he checked the checksum without intending to do so; it had become a ritual. If he wanted to check it, though, it always seemed to be something else; but this was another law of human existence, as only the uncontrolled intention can tell us the truth about ourselves --- and the world.

The analgesic power of music was entrancing and fascinating him at the same time, when he listened to a ballad; this seemed to have an impact on his mind that was comparable to drugs or alcohol. Instantly, his thoughts began to race, when his concentration on the music stopped and the steady flow of rhymes didn't do the trick of a diversion anymore; he wasn't yet back to reality yet, but the racing thoughts were inevitably and unpredictably focused on her. He tried to concentrate on the song once again; the feeling of momentary bliss was gone, and he was back into the cold reality again, caught by the walls of the house and the much stronger adamant fence of his conscience he could never escape without her help, as they could only be opened from the outside. 
He took of the headphones, and listened to the powerful, steady music of the world that never went to sleep; the buzzing and summing, humming and crackling sounds of thoughts, of pure energy flowing from mankind into eternity would always be there, even if the last soul ceased to exist; remembrance would persevere.

Persevere; that was the word he had been told when he'd tried to find out about the future.

None should try to do it; the aim should be the perseverance of intention, but he didn't want to realize that right now, as he liked to stay passive. 
He wondered whether he was simply concentrating his feelings on her, thus not really being in love; he also asked himself if that would be possible with another person. Could he do it? Would it be the same?
He knew it was different. He'd once been crazy about G., and that was something like just concentrating feelings on a person; that was the way all creatures capable of feeling love would start to discover their emotions. With O., this was different, though he couldn't describe of which type this distinction was.
\begin{verse}
When I was young,\\
I never needed anyone,\\
and making love was just for fun,\\
those days are gone. \\[1\baselineskip]

Livin' alone,\\
I think of all the friends I've known;\\
when I dial the telephone,\\
Nobody's home. \\[1\baselineskip]

All by myself,\\
Don't wanna be,\\
All by myself,\\
Anymore. \\[1\baselineskip]

Hard to be sure,\\
Sometimes, I feel so insecure;\\
And love's so distant and obscure,\\
Remains the cure.\\
--- (I think there's no need to tell you the name of the song; you'll all know it)
\end{verse}
That was the ballad he'd heard. 
It somehow showed the distinction puberty would make: Suddenly, there would be some wish for something you didn't know before; you'd always known it was there, but you didn't wish to go there, which remembers us of China and the way Philip Pullman put it.

Maybe you'll now want to ask him about the outcome of the attempt to contact her: He'd told her about that on Monday, and she hadn't reacted ever since. He'd included the question if they were still friends in quite a direct and possibly offending formulation, but it was necessary, as she'd offended him with her quietness, too. In addition to that, he'd also sent her some important informations about a lecture she'd partly missed; thus, it would be quite important for her to receive that information, but she seemed not to wish to do so, and he didn't want to push her to do it.

For now, he'd done enough; this weekend, nothing more could be done, with the possible exception of continuing to write these texts. 
That moment, he listened to the ringtone of \enquote{The Godfather}; it was a sad, melancholic, but strong melody, though it was not polyphonic.

Switching over to writing again, he thought about the things he was yet about to tell you; the things he \emph{ought} to tell you. For example, the moment when a friend of Y. had asked him for his number so as to contact him; or the moment when he tried to sit together with P. when O. wasn't there just to escape L.-B.

P. had noticed that L.-B. was after him, and she'd also told him that she nerved her in quite a similar way as she nerved him. His friendship with P. and her friends was growing, thus giving him the possibility to attain a certain degree of stability again, so he would probably be able to cope with his subdued feelings. P. knew about the texts he was writing, though she wouldn't have guessed; he'd told her, and one of her friends even started to read them as he'd told her, too. But he trusted both not to tell anybody else.

However, he feared the power of the signs; and nobody but him new what he was really writing about, and only O. would be able to find out --- if she ever wished to do so.
At the moment, she always seemed to be in a hurry, caught by her own thoughts, probably not thinking about him and not realizing the slightest trace of his feelings for her.

Nobody knew what would happen, and nobody wished to find out, as the person who touches the secrets of the future is near to desastrously throwing the ability to decide about his own away. 
Let's wait and see, though this is one of the worst things to do. 
Hope you're still taking part and understand the way I feel\ldots

\begin{verse}
The persons everybody thinks to be happy\\
are the saddest of them all. \\
--- W.G.
\end{verse}

\begin{verse}
Have a look at that childish play of emotions:\\
They stare at each other, knowing what each other feels,\\
and throwing the power of acting\\
away,\\
forever,\\
never to be found again,\\
feeling aghast and lonely forever,\\
the two of them:\\
Not united\\
but separated. \\
--- W.G.
\end{verse}

%%% Local Variables:
%%% mode: latex
%%% LaTeX-command: "latex -shell-escape -synctex=1 --file-line-error-style"
%%% ispell-local-dictionary: "british-ize-w_accents"
%%% TeX-master: "../My_Life_in_a_Nutshell_-_First_Era.tex"
%%% End:
