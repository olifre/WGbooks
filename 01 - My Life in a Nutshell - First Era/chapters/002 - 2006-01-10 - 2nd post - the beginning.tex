\Chapter{The Beginning}{or: the day he heard \textbf{HER} name}
\label{cha:the-beginning}
\subsection*{Originally published: \DTMdate{2006-01-10}}
\begin{quote}
Hello again!
It's time to continue the story, as my present is changing faster than I can type this story, and I haven't even told you the beginning\ldots
\end{quote}

It was a normal day. A nice day. Also for him, as he had changed to another school some time ago, and his profiling system soon gave him the chance to be everybodies and nobodies friend, the only way he'd found to be accepted by other children, because that's what he was: A child. At least, to anybody he knew. You must know, he normally was no person to remember something word by word or the exact place where something happened; but that day, that moment when he heard her name for the first time --- even though it was just her surname --- was something he would never forget. He remembered the exact place, the position of the sun that shone on his back, while he faced the other direction and simply heard two of the pupils that were in his class talking about two girls who newly moved somewhere near (the two boys didn't tell where \textbf{THEY} came from or where \textbf{THEY} went, and a long time was to pass till he'd find out about that). At that moment, he just felt kind of special when hearing that names; they somehow sounded important, and he couldn't help but remember them. And the situation when he heard them was saved in every detail on the harddisk every human being owns, but few of them really use: Our brains. At that time, he decided to suck in all information concerning these two girls; he wasn't yet able to decide which of them was so important to ring a bell in his conscience, and he wouldn't learn before one of his most fatal errors had become a fact nobody would ever be able to change again.

About one year, nothing happened; his method that included profiling and sending a special kind of profile to his \enquote{friends} worked, and soon he was accepted at his new school, often asked for help with homework or \enquote{other problems}. He couldn't help but give this help, and be happy about it. This was his only joy at that time, and it overwhelmed his emotional experiences until then. \enquote{Friends} everywhere around him, calling his name, asking questions; but just about school, a sad fact he ignored on purpose. But if one happened to ask him somebody about himself, he always remembered it and this person was marked as somebody special for him. Soon, he had something you'd call his \enquote{best friend}; well, he was, even in an objective view. Who wouldn't like somebody pretending to be perfect, somebody who really seemed to know everything? Well, everybody did, but only few people want to be friends of such men. So finally, he had a best friend, and on one occasion he was talking to him, discussing a topic he can't remember today; for him, it's about the third of his life ago. But like before, he can exactly remember where the sun was located; \textbf{THEY} couldn't be seen clearly, as he had to look into this gigantic source of light if he wanted to see \textbf{THEM}. Blinded as he was, not knowing what he did expect to see neither noticing he wouldn't ever be able to see completely clear again, he saw two girls; he saved their voices for eternity in the part of his brain that would never forget a thing, though he didn't know their names nor recognize their faces. In that position, they looked the same to him, and they had heard the discussion, supporting him by calling his nickname, urging him to enforce his will, the thing he didn't own anymore. He decided with the logical part of his conscience to regard these two as two silly children, who liked to make fun of him, and didn't ever remember this moment even when he knew them, their names and the places where they lived; the time of remembrance was yet to come.

Making the connections between names and faces had always been a problem for him, and thus, it would take him an even longer time to recognize \textbf{THEM} as it would have taken another boy; but when he knew them, he would have been able to \emph{\textbf{FEEL}} if they were near and he could tell you who was who from such a far distance you couldn't notice a single feature. But this knowledge hadn't come yet; he didn't know nothing about his future, and he wouldn't know till some years would have passed. 

\begin{verse}
Think only of the past as its remembrance gives you pleasure.\\
-- Jane Austen (1775-1817), Pride and Prejudice\\
(if you're a masochist --- comment by W.G.)  
\end{verse}

\begin{verse}
A friend asking you to do something for him is no friend --- a friend asking you to do something for yourself is a good friend.\\
-- W.G.
\end{verse}

%%% Local Variables:
%%% mode: latex
%%% LaTeX-command: "latex -shell-escape -synctex=1 --file-line-error-style"
%%% ispell-local-dictionary: "british-ize-w_accents"
%%% TeX-master: "../My_Life_in_a_Nutshell_-_First_Era.tex"
%%% End:
