\chapter{Abysmal Heaven \& Towering Hell}
\label{cha:abysmal-heaven-and-towering-hell}
\subsection*{Originally published: \DTMDate{2006-03-21}}
\begin{quote}
Hi, once more!

Another day had just passed by, and many things have happened which were of concern to him; he'd already contemplated a lot, and was now caught by the complex silence of his own mind. Let's try to have a look at it together\ldots

He was standing in front of it as we all were: Abysmal Heaven \& Towering Hell.
\end{quote}

Tuesday --- a special day, this time. 
He would contemplate, remember the time when one of his professors told him that he should write books after he'd answered something in one or two short sentences, just using a special technique of putting what he wanted to say; others had given shorter answers, and they were blamed because they hadn't thought of any more. This single sentence she'd said him, the look upon her face, the expression of her eyes --- she'd meant it.

Probably, this was one of the reasons why he started writing, as it gave him the confidence to be able to do so; on the other hand, it was simply giving him the possibility to cope with his life, as writing is contemplation, and for him, thinking about everything in detail was his life.

But before we get lost in contemplation again, we shall have a look at those details I've promised to add --- those things I've forgotten yesterday. Some pretty short things: He'd --- once more --- talked to one of P.'s friends, and she was the person to address him --- not the other way round. Then, B. hadn't answered, but was friendly nevertheless --- he didn't dare asking her again. But there was something else, something we should have a closer look on; something he'd learned about G., and some conclusion he'd made about O.'s character as a consequence of that.

But we'll start with something he'd just heard about G. from a person that seemed to show interest in nearly all girls; and he'd even heard about him going on O.'s and Y.'s nerves. G., however, seemed to have told him her opinion about some object; and when he took that opinion he'd now heard from that boy for real, as G. hadn't been as ironic as O. had always been, but quite the opposite instead, he'd have to come to a new conclusion. The machinery of solving a new maze was put into action, again; some minutes, or sometimes some hours later, he would think differently. The jigsaw puzzle was soon solved; several facts he'd once learned would lead to a new conclusion, if they were connected in the right way. G. was different from the person he'd taken her for. He'd always believed that there was some deep and powerful, sub-emotional talent to be found which would allow her to learn about life, and about the basic foundation of the world; now, this idea was shaken to the ground, when he realized that there was a different motivation behind the social interaction. Of course, that was totally natural, and she would act like all others would do; he couldn't help but smile, when he remembered the contents of the book \enquote{Stranger in a Strange Land} by Robert A. Heinlein. The original version would have to be cut before publishing, as it contained too many offending parts for its time; he'd read the uncut original, which had been reconstructed later. It proclaimed the main intention of all human beings, and made things simple, which was normally a pretty long and complex progress. Thus, this book contained some main ideas about life itself:

The basic fear humans would show when something new, something unknown would arrive; they'd rather destroy it than try to explore. 
The basic instincts which were the gain of property and the possibility --- or the wish --- to see others nude. 
The basic idea of religion --- it was to be a sort of entertainment.

Well, some things were rather extreme; but in fact, our world is still based on such rules, and our \enquote{sophisticated} ideologies are just systems to hide these basic ideas, as we don't like to have a look at them. So, what had he learned about G.? She'd probably show these basic instincts, and he was disillusionized, as he'd hoped that there were other people who'd successfully tried to conceal these. However, he felt that this made her more truthful --- but on the other hand, he knew that it made her look stupid in these times.

So --- what had he hoped for? He'd hoped for something, for somebody perfect, somebody without these human flaws, or at least somebody who'd try to learn about them, in the was he was doing all the time. With G., he felt that this hope was probably gone; she'd not be a person he could start a serious relationship with. He'd now just realized that the basic flaw of love was the difference between the physical preference for somebody and the reality of a stable relationship --- probably, it was impossible to join these two ideas forever. Probably, they would sum up to zero again.
And which impact did this have on his view upon O.?

He'd thought about her the same way he'd thought about G., and their characters were in some way comparable, though they had developed quite differently, especially in the last years. Their former friendship was now just a system of give and take, and they had got used to it; probably, this was to be the fate of all stronger friendships, to make us copy the opposite features of the person we like. Right now, he was wondering whether it was the same with O.; his mind began working, but there was too small an amount of information to judge on her right now; however, he was also unable to answer this question negatively. The problem was that she was a kind of easy-going character, somebody vivid who'd take life as it is and make most of it. This made it nearly impossible for him to find out if there was something else in her, some part which would like to find out about human nature and about the nature of herself. However, he felt pretty strongly that it could be this way, while he knew that this could just be an interpretation based on the still-not-completely-extinguished feelings.

This day, G. hadn't been there, and O. had left earlier, though both of them should have stayed to the end; he felt that she wasn't taking everything serious right now, which was another change in her character. But he himself was changing even more: P. and her friend had given him the ability to deal with people without the steady fear of a word escaping his mouth which would render the social intercourse a social struggle. Now, he was the one his friends would send somewhere to ask somebody; he knew the fear they had, and he realized that he was losing it. Which meant that he was becoming more equal to O.\ldots

He'd taken pictures of her, her best friend, B.-B. and some others that day; this time, she hadn't ignored him, but stood there happily, waiting for him to press the button. She had even waited when he had to go to the process for two times because of a technical problem. And then, he'd found a reason to walk up to her: He would show her the picture on the display. But no real conversation was to start.

He'd done so with O.'s best friend and some others, too; O.'s best friend seemed not to be together with O. all the time now, and he wondered whether this was some sign of change.

He realized that paradoxa where the basis of stability; he'd learned that by looking at all those details around him. Then, he'd also tried to find out about the so-called \enquote{mid-life crisis}: He'd noticed that it just meant people realizing that life would go on without them, and that they seemed to exist without importance. It took a whole bunch of time to be able to accept that the people around oneself would act independently, no matter if you were there; and it took some year to realize that you had to be part of it for the sake of the system, for most people, that is.

He was still in the process of realizing these facts, but he knew about the aim, and he knew that this process of finding the world and oneselves position in it would probably repeat itself nevertheless; but he wouldn't think about that now.

Something else had happened: He'd now gained more self-confidence and was building up a real character. Thus, he'd told the \enquote{Evil One with the Glasses} his \emph{real} opinion about something. He knew that she would probably be offended; she was, in fact, but she told him. Their friendship wasn't changed, it seemed, however. Finally, she knew more about him now then O. had known, as he hadn't told her so many things, because he had always evaded the things that could have offended her. However, his opinion seemed to be more like that of her current boyfriend\ldots
He wouldn't think about the probability of them having similar characters.

Some thoughts were too cruel to think, and some too happy; but finally, he'd always liked the extreme, and he'd long ago imagined all this, feeling once more that life was a wave. The only thing he'd have to do was to wait for the next maximum, minimum or zero concerning his relationship(s); the problem was that he wouldn't know when this point would close in.

And he'd also realized something else: His close friendship with O. had only worked because he had ignored her; real love was always based on some conflict. 
L.-B.'s love was, and his ignorance in the past had been.

Sometimes, he felt that timing was important, indeed; but there was also the importance of a \enquote{breakthrough-event} which would link to people who had been apart or in struggle together. 
This hope would be kept, and he felt that his self-determination was indeed limited extremely by that outlook on life. But this was the way it was, and we'll see what's yet to come --- the next time. 
For now, I'm tired to the death, and even less time will be available the next days. 
I hope you'll stay with me\ldots
And tell me your opinions\ldots

\begin{verse}
His best friend \\
had been close to finding out about all this; \\
but he had noticed, \\
and been a good actor, \\
as he'd always been; \\
his profiling system was hiding his life, \\
his emotions, \\
and his character; \\
he was trying to break through that shell \\
in a tremendous effort; \\
would he succeed without destruction? \\
--- W.G.
\end{verse}

\begin{verse}
If everything was relative, \\
then nothing was absolute; \\
but zero had to be. \\
Thus, relativity was illusion; \\
at least, when it came to things apart from matter, \\
to the basics of our thoughts, \\
the harmony of the universe, \\
and the interference \\
the wave of life \\
would create. \\
--- W.G.
\end{verse}

%%% Local Variables:
%%% mode: latex
%%% LaTeX-command: "latex -shell-escape -synctex=1 --file-line-error-style"
%%% ispell-local-dictionary: "british-ize-w_accents"
%%% TeX-master: "../My_Life_in_a_Nutshell_-_First_Era.tex"
%%% End:
