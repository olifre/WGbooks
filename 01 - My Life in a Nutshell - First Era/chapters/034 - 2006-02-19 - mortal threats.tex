\chapter{Mortal Threats}
\label{cha:mortal-threats}
\subsection*{Originally published: \DTMDate{2006-02-19}}
\begin{quote}
Back again --- still alive!

An interesting weekend\ldots
Some contacts to the past were giving me a hard time, as you'll see when having a look at the title. On the other hand, I remembered some more details about Friday; I don't want to keep these from you. 
So, if you want to find out about all this, continue reading!

Shocking Past: Mortal Threats.
\end{quote}

Before we' have a look at those recent events, we'll cast some light on the fact that I've told you on Friday, that this day was a kind of \emph{summary} of what had happened. Indeed, it was, as I've forgotten to tell you something; we shall have a look at this now. \\
\textbf{Returning to Friday --- Flashback}\\
He had just exited the library, and P. was with him; both were climbing the stairs to reach the next lecture in time. Not realizing that O. would have to pass him by, as her lecture had just finished close to the place he was heading to. 
Thus, as he didn't hope to be near her, he would be, for the fraction of a second. When he reached the top of the stairs, P. was walking somewhere next to or behind him; at that second, he didn't mind, as he'd got used to her presence, enjoying it. B.-B. was just passing by, beginning to go down; presenting him a warm smile. He was happy, and he thought that this would be the only time for that day when he'd communicate with one of O.'s friends in some way.

It wasn't, as he'd proved wrong some seconds later, when G. was passing by, smiling at him in an even happier way. The next person walking up to him was Y., and though she seemed captured by her own thoughts, she returned his smile, too. And then\ldots O. was there, and she seemed completely absorbed by \emph{anything} that wasn't him, as her eyes were jumping around as if escaping his presence. Nevertheless, he smiled, and in a subtle way, he saw her smiling back; On the other hand, this smile was only a trace of a smile, something that happens if you want to suppress your feelings --- or if you don't really care. 
Once more, he wasn't sure which of the two options he'd prefer, and which one was real.

But that had all happened on Friday; so, why was this some sort of summary?
He remembered how he'd felt that B.-B. was important in some way, when he'd arrived at university; The next person he noticed was G., and he felt something special about her. Then, he noticed that Y. seemed to be G.'s best friend; At the same time, he realized that the two of them spend too long a time together, which was a sign of the friendship burning, which meant it would be strong --- but just for a short time.

And time didn't prove him wrong, this time, as Y. was now the best friend of another girl. 
And finally, O. came into his vision, in the same puzzled way she'd done before\ldots

All the time, P. was with him. Somehow, she'd been with him all the time, since the beginning; He realized the idea of most films: A man loved a woman, he really wanted her, and in the end, he noticed that the girl that had been with him --- and helped him --- all the time was the person he really loved. 
But he was pretty sure that this was different; Life was always like a movie, and at the same time, it wasn't.\\
\textbf{Back to the Reality --- End of Flashback}\\
However, it's Sunday now, and we're going to have a look at the rest of the weekend. Saturday evening, he was browsing the web again, communicating with P.'s friend, \textbf{the} one he knew from university. Of course, their talk was not really about something; nevertheless, both seemed to be enjoying it, quite unlike the way L.-B. was talking to him; This time, he was happy when one of them found a topic to discuss. However, he got to know that she'd found a new friend, a boy; but it seemed that it was not to be her boyfriend, though he felt some jealousy, at least a glimpse of it. Which told him, that he liked her. On the other hand, he'd probably have felt the same with P., and finally, he wasn't even sure that P.'s friend was single. 
Which left him at the beginning.

But the real important thing, the confrontation with his past, was yet to come, this Saturday evening, and in the night. One of his relatives, the same boy who'd lead the \emph{gang} in the past, the same people that had humiliated him every day, from the beginning, until he'd had the possibility to escape to university\ldots

The things he'd locked away, and the same things he wanted to cope with now. Probably, this was one of the best things to happen; Though he saw this boy every day, no words were exchanged, and their eyes never met since years of time --- He hoped he could stand his stare, but he was quite sure he would, as he was a leader and \emph{nothing} without a group around, protecting him. 
But most times, he \emph{had} this group around him --- On the other hand, our protagonist was amongst so many \emph{normal} people at university, such a bunch of persons who were his friends, that he would never really be alone at this place, though he felt like it, sometimes. 
Especially, when he was in the bus, and alone, indeed. 
And that boy would be in the bus, too.

His problematic way of having dealt with the past by not really dealing with it at all had made the mortal threats sound real, somehow, and they'd impressed him in a way they shouldn't have done. 
But we'll start with the beginning, as everybody should.

This old enemy of his was having a party this weekend, but he didn't know about it --- however, his mother did. Thus, when the phone call of some person he didn't know reached him, he assumed it to be somebody from that place who would like to mock him --- and told him to prepare for his death on Monday, when he'd get to know that person, at the university he was attending. The voice sounded somehow like a voice he'd heard somewhere before, but the name seemed to be a fake. Probably, it was that boy, but he wasn't sure about it, as he sounded different; but he hadn't heard him talking for years. When his mother told him that this boy was having a party that night, he was pretty sure it had been him, or one of his friends; He'd told him to go in the garden and prepare his grave. 
Though he knew that he didn't have to fear anything, he shuddered, and his stomach felt clenched like a fist for about half an hour, or even longer.

He'd simply stopped talking, listening and finally dropping the phone, so that the connection was cut off; And the boy still seemed to be talking\ldots

The next time, when it was night, his mother answered the phone, assuming that it was that so-special boy, wishing him a nice time. Then, when she gave him the phone, the boy pretended to be somebody who had dialed the wrong number, and he could now quite clearly make out this voice that had once commanded others to hurt him in a whisper. 
That totalitarian person, the one who'd started the game of absolute power in his youth. Would he ever be able to stop playing before somebody got seriously injured?
The only moment he was now fearing was the time when he'd be in the bus; Probably, he'd be alone with that boy and some of his friends. 
But he'd said on the phone, that he'd kill him at the university, and that would be quite impossible, as nearly everybody knew him there. And finally, he'd have to find him first, as though he wouldn't hide anywhere, a university was quite a big building\ldots

Thus assured, he was able to smile again, though some doubt was still running around inside his mind. 
He'd try to ignore that.

Something that seemed quite peculiar was the fact, that his mother couldn't remember what that person had said to her on the beginning; He assumed that to be a result of her really low blood-pressure, especially in the middle of the night. On the other hand, she could be keeping something away from him she didn't want him to know\ldots\\
\textbf{CUTOFF!}\\
Once again. 
It was finished, and you'll learn more about that soon, if I prove to stay alive. If not, the this will be the last post, probably; But I don't think so. 
However, there is still some fear left, as when the phone rang this morning and some friend was calling, the sound of that gadget made my heart stop beating for a second\ldots
But we'll hope for some happy ending of this childish joke. 
Hope to be writing again, pretty soon\ldots
And please don't leave me alone!
Give me your opinions!

\begin{verse}
A murderer was walking around, 
but he could have guessed; 
even if the threat was just a joke, 
happy times must be recompensed 
with shocking experiences. 
Would he agree with that guy 
smashing him on the floor, 
attacking him, 
if it meant a future together with O.? 
Yes, he'd still agree. 
--- W.G.
\end{verse}

\begin{verse}
O.'s number was all around, 
the basis of the world, 
the basis of himself, 
the basis of his soul; 
the number two. 
Would it rescue him? 
He'd agree with every hardship if it would. 
Would it kill him? 
If that meant dying for a dream, 
he'd agree, too. 
--- W.G.
\end{verse}

%%% Local Variables:
%%% mode: latex
%%% LaTeX-command: "latex -shell-escape -synctex=1 --file-line-error-style"
%%% ispell-local-dictionary: "british-ize-w_accents"
%%% TeX-master: "../My_Life_in_a_Nutshell_-_First_Era.tex"
%%% End:
