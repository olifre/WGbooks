\chapter{Flirtatious?}
\label{cha:flirtatious-qm}
\subsection*{Originally published: \DTMDate{2006-03-01}}
\begin{quote}
March has just begun; let's see what's gonna happen!

As you may have noticed, I've been silent for some time; there was less contact to everybody, and my own voice fell silent for some days, captured by isolation. 
Thus, it's return would be the more powerful; He wondered, why several people started to adore him, or at least to respect and accept his character the way it was, even though the profiling system would pretend them something they could accept more easily. 
Please join me now; we'll go into detail, though there's not much time left.

Was he becoming: Flirtatious?
\end{quote}

Monday, Tuesday --- days of silence. He was able to contact P. and her friend via digital means of communication, and he felt he'd found some real friends --- nevertheless, real contact was different. He hadn't understood this for a long time, but now, when he started joining others to talk to them, he realized that virtual reality was not to be a substitute for life itself.

He remembered the thought that had been hunting through his mind for the last few weeks: That he'd still be able to contact O. when they were not attending the same university anymore; but this contact would be virtual, and he knew that this could never be the real foundation of anything. Noticing that those other friend of P.'s he'd found on the internet had fallen silent, he'd realized the irreality of virtuality.

And, at the same time, the reality of the words that were formed by fingers that hit keys. However, those words were gone at quite the same speed words in a conversation could be forgotten, and most times, even quicker, as the contents were of even less importance.

Nevertheless, P.'s friend had \enquote{talked} to him, and both attempts to contact P. and her friend had succeeded: The answers he'd received were long and satisfying, though they told him how different O. and her best friend must be. But differences can attract each other, while they can also be the foundation of eternal trouble\ldots

Now, we'll have a look at today, a day full of peculiar actions. We'll start with the morning, when he entered the bus, reserving a seat for some friend who'd probably not come; then, at the same place O. was living, a woman entered the bus, and though a black man stood next to him, staring out of the window though he could have asked if he could sit down \emph{(he'd have allowed him to do so, as adults were rarely seen in those buses, and so as not to prove himself being prejudiced)}, that woman asked him --- not politely, but offending --- to take down the bag he'd laid on that seat. At the same time, she was trying to sit down; thus, he had to obey, though he felt a glimpse of rage that was now residing inside him. As a result, he stared out of the window, trying to ignore that woman who was sitting so close that her leg touched his. But a quarter of an hour later, he realized that she would start talking soon, and she did, suddenly telling him all those blows fate had given her: Searching for jobs, not being able to be truthful to others as the money was the only thing of interest in these days, an accident with the bicycle, having worn no helmet, her boss being late for work, all the places she'd worked at, and, finally, that she'd be working here till Saturday. He was left to his profiling system and his intuition, which rarely betrayed him; thus, she was not offended, but continued talking, though he was feeling that this was somehow \emph{wrong}: A woman, that seemed not to like people reserving places for others, suddenly talking to him --- a boy --- as if she'd known him for years, using words and stylistic devices of a younger girl --- or a poor person, not really a beggar, but something not that rich. And those young girls liked to imitate these people, as they wanted to please others with those words, showing they knew what they were talking about --- the \enquote{have-been-there-and-seen-it-all}-phenomenon --- which would finally seem to make them older.

Which was exactly what they wanted to be, though they would want to be young again when they were old. This was the wheel of time: Nobody would feel to be at the right place. 
And \textbf{that} wouldn't change, once and for all. Change\ldots he'd changed reality that day, in a combination of conscious and subconscious methods. One lecture had been cancelled in a real peculiar way, just because he'd managed to concentrate his thoughts the right way: He'd wished to talk to one of his friends in that lecture, but he was not so sure if he'd like it to be cancelled; then, he thought that he'd rather not come if he continued to hope than the lecture to be cancelled.

Finally, he was left bewildered, when he was in a state of mind \emph{in between} all those decisions; and then, he left this part of his brain in confusion. What he didn't realize at that time was the fact, that he'd thus exactly reached the state of mind Keats had once described, the condition of questioning without demanding for an answer. This was, according to some people, the state of mind one needed to influence reality, or to gain insight into the future. Achieving this condition was something that seemed impossible, but he realized the power he could gain when mastering to control it. However, he felt that one part of it was the fact that it could not be controlled; On the other hand, it could be trained.

That lecture hadn't been cancelled in advance, and thus, the fact that he didn't see that friend of his and the professor led him to the funny --- but at that moment improbable --- conclusion that he'd known in advance that this lecture had been cancelled. And, when the professor didn't come, he was able to fetch the bus together with some other friends of his, dropping the things he should have guarded for somebody --- the girl that was in a stable relationship with another student for a pretty long time now --- in a room where she should find them. Meeting P. at the bus stop, he talked some moments to her, while she had already planned what to do with him, leaving him stuck between accepting or denying it. But he felt he wouldn't react in any way. Those other two girls --- \emph{the evil one with the glasses} and a girl he hadn't seen for a long time --- joined him on his way home, talking to each other, while he tried to join in now and then.

But that was the end of the day, and we've left out the central part. The woman in the morning, that had finally been going on his nerves, had gone and he was late for the first lecture; but that's not of importance to us now.

He felt that it was Monday, though it wasn't; However, after some days of break he felt that the first day \emph{must} be a Monday, realizing that he'd meet O. then and telling him that this was impossible. 
When he was standing in front of another auditorium, P. standing in front of him, both absorbed in a conversation, next to B. and her best friend, he saw Y. and some others standing to the right hand side. Y. had asked him before, whether he could help her with some exercises tomorrow, and he'd gladly agreed, though he felt she didn't like asking him and seemed to be thinking that she was getting on his nerves. Nevertheless, he didn't tell her she wasn't, as he could clearly remember the way L.-B. kept asking him if she was going on his nerves, thus leaving him stuck with the answer she wasn't, feeling it would become worse if he told her the truth. And Y. would notice in some future time, that he enjoyed her presence.

Then, O. arrived. 
She was walking up to Y.; he saw G. standing some way behind her, together with her boyfriend, while O.'s and his eyes met for another moment, longer than all those short seconds in the last weeks; she smiled, and he did so, too. Though he was in a conversation with P., he could slowly lift his hand a bit, waving, though she was finally looking another way at this time after she'd waved at him on herself. She was now talking to Y., and he felt he hadn't greeted her warmly enough, as she had simply reacted as if that had just been a job: Greeting him with a smile, as if it was something on her To-Do-List. On the other hand, he'd realized that for that fraction of a second, she'd been completely concentrated on him, and even dared looking into his eyes once more; he wondered whether something had changed. P. kept helping him to protect himself from L.-B. during the next lectures, while that girl had phoned him yesterday once more, asking things she should've known and her mother --- who was heard in the background --- assuming immediately that she was phoning \emph{him}; which meant, that she must've shown her with her facial expression.

He didn't want that, and was happy that P. was with him. The two of them stuck together, leaving the auditorium together as the last to students, and they would wait for each other having tucked everything in his / her bag. It surely was a strong friendship, and probably a basis for more, though he wouldn't know whether he really wanted to take part in such a development at this moment. But the friendship would be secured.

He wondered whether O. was thinking in a similar way; some time later this day, when he was in the library together with P.'s friend, P. herself sitting some metres away, another boy, sitting even farther away and in the other direction asked him something. As he couldn't understand what he was saying, he leaned forward, facing P.'s friend and finally cupping his hand around his ear, so as to amplify the quiet words that escaped that boy's mouth. He understood him, finally, and P.'s friend was suddenly gesticulating in a happy or maybe even wild manner, pointing at some (person?) standing behind him. Or, so to say, to the right of him as he'd had to turn to the left to listen to that boy and was now offering that person his back. Turning around quickly and wondering who might want to ask him something right here without walking around him so as to see into his face, he suddenly faced O. Astonished, both pairs of eyes met once more, and both smiled warmly at each other. She was standing more close than another boy would have stood, symbolizing a subconscious, warm feeling --- probably. She was searching for somebody, and he told her where he'd last seen those two students; however, she seemed to try to be funny once more, that time not using her normal way of doing a joke. It seemed to be more shallow\ldots

Even if that could just be an impression or interpretation of his, he felt that something had changed; whether it was the influence of her boyfriend taking control and changing her life or having withdrawn. He wouldn't know.

Some seconds later, she was gone, and he focussed on P.'s friend once more, concealing the expression on his face and in his eyes by switching back as quickly as he could. 
Nobody seemed to have noticed.

But he was still wondering whether O. valued this friendship --- if it \emph{was} a friendship \textbf{to her}, that is --- highly. 
The next lecture was to begin, and he saw a picture hanging at the wall, something a student had done; it showed a screaming person with thin cheeks, pressing his (her?) hands to these, coloured in some blue and white wavish colour. A picture of a person that seemed shocked or mentally ill; he was remembered of the book O. had read, and he had (then) read, too; and of the way she was concealed that Wednesday and Thursday that was now quite a week ago. He compared her with that picture, and wondered whether she'd like it --- or pass without noticing it was there. 
He'd like to find out, but he couldn't think of any way to do so.

Sometimes, he thought that O. would see him writing those texts, and then, he felt she'd rarely enter the library, the place where the computers were located; and, in fact, he wondered whether he'd tell her about the real contents, or --- at least --- about him being the person writing those stories, those patches of reality.

Probably, he'd do the latter; please look forward to the next text, probably, everything will be solved --- or more complicated once more --- at that time. 
Be patient, and don't worry: I'll go on!
Still there? Any opinions?

\begin{verse}
Breaking the silence, \\
with the power of thought, \\
is the sign of restless \\
contemplation; \\
that's the basis \\
of intelligence, \\
and the foundation \\
of death. \\
--- W.G.
\end{verse}

\begin{verse}
Reality \\
is thinking. \\
And thoughts \\
are reality. \\
The only thing, \\
to keep those brothers \\
apart, \\
is the fact \\
of us not understanding \\
ourselves. \\
--- W.G.
\end{verse}

%%% Local Variables:
%%% mode: latex
%%% LaTeX-command: "latex -shell-escape -synctex=1 --file-line-error-style"
%%% ispell-local-dictionary: "british-ize-w_accents"
%%% TeX-master: "../My_Life_in_a_Nutshell_-_First_Era.tex"
%%% End:
