Wilkie Goldentongue --- another text, another day, another mood!

It's Tuesday; another day I wouldn't really see her. And another day when I'll have a lot of time to spend writing. 
That's an interesting day, however; it's the day of the eight. We'll see what happened today; and I hope you'll be able to understand, as I'm quite incapable of coping with all the sensations that are recorded by my senses all the time. 
Just give my texts a chance; it may be worthwhile to read them, and there's nothing you could lose by doing that. 

\textbf{Let's give it another try:}
\textbf{[u]Surprising Sensations[/u]}
This was the day he was open for everything; a yellow day, a day completely inflenced by the eight. 
A number representing social interaction, communication and friendship; the number of Y. and her colour, but we'll go into detail about that later. 
This day seemed only to be influenced by these attributes, and it was one of the moments of freedom; today, he would be able to forget about O. for some time. 
And he did, though --- or perhaps, \emph{because} --- he saw her for two times. 
But both times, it took him some time to realize it; one moment, he was even imagining that she was watching him, but he quickly denied that feeling of his. 
The other occasion seemed quite peculiar, as he was passing her with one of his friends, and that friend simply stopped near to her; nevertheless, it took him --- me --- some time to realize that O. was standing right before his very eyes. 
But she didn't look at him, though she must've noticed his presence. 
Finally, he urged his friend to go on, as both seemed as if they had been caught by a sudden nervosity and a certain bewilderment, that none of them could deny nor backtrace to some reason. 
He was the first who gained back the control over his senses, and finally, he caught the bus in a hurry. 
O. was yet to stay; she had to study longer, as they were studying completely different subjects, though both were into sciences. 
But we don't want to go into detail about that, though these impressions can also be regarded as \textbf{"Surprising Sensations"}; Right now, we're going to have a look at the effect this \emph{"day of social interaction"} was having on our protagonist. 
As we already noticed, he was now capable of ignoring her, even if she was present. 
Thus, he had gained part of his heart back, and was now able to be charming and interesting towards his friends again, which was something he'd lost somehow. 
He wasn't sure if his love for her would ever cease to be, but he'd noticed something: It was the same month --- just a year later, quite exactly --- when he'd stopped loving G. 
He had just been through another depression. 
Indeed. 
Maybe, this was it; but he wasn't so sure. There was no other person that had stepped into his life with the possible exception of P., a person that would always be a friend to him; maybe, his love wasn't really subsiding, but the burning and most destroying fire had been extinguished to make place for a wise flame. A flame that would probably burn forever, and now, she was the one to decide whether it would calm down to rest eternally in the form of glowing embers or whether this fire would ever be able to warm both of them, shading light on the world around this selected couple. 
He knew that with a big fire, there were always smoke and shadows; but he knew, that the light would be strong enough for him not to notice. 
But we don't know what is going to happen, and nobody can tell. 
Not at this moment. 
Maybe never. 
But the fire is still there, slowly, but eternally burning within his soul, as he's still seeing her picture in his mind, clearly pointed out as somebody special. 
Somebody, who never brought a ruler, which gave him the most happy feeling to be able to lend his to her at exactly the same time she needed it by simply dropping it on her paper. 
Or by giving her a paper, even before she asked another person for one; he could read those wishes in her eyes, but as he rarely saw these sources of her vivid spirit the last months, he started to adapt and guess. 
It worked, at least most times, and both seemed to be happy with it; but just for a person that watched from the outside, as he was longing for her embrace. 
There were two hours a week when he would sit next to her, something he'd arranged and she'd accepted or probably even wished; He didn't know how long this would last, but of course, he wished that condition to be static. 
But as we announced before, it was Tuesday; and he was receiving new, surprising sensations, as he'd gained control over some of his feelings again. 
He didn't fall in love with somebody else, he was quite sure he couldn't; but a friendship started, something that helped him cope with this loss and all the other things that were happening around him, all the terrible masses of sensations he received. 
Suddenly, he was waving with his hands to greet somebody; he was smiling at the world, not just profiling, but really gripping part of the joy such a smile induces. A friendship with P. was developing, while he noticed, that the colour of this girl wasn't really pink, but something darker, more interesting; just right for an exciting friendship. 
On the other hand, there was L.-B., a girl that seemed to be in love with him --- or at least crazy about him, as she was trying to take every chance there was to talk to him, even in search of most uninteresting topics. 
What does L.-B. mean? Light-Blue. Not the dark blue that shows intelligence; she wasn't stupid, but a bit of a shallow, and she seemed to be a bit of a bore, too. 
And she was getting on his nerves, something P. noticed at once, also confirming the idea that L.-B. was in love with him in quite a subtle way. 
Maybe she was, but he decided to ignore that. 
However, he wasn't going to tell her that she was getting on his nerves; she would have to find out on her own --- which she never would --- but if he'd told her about it, the situation would have been even worse. 
Thus, his surprising were something quite natural, but he'd ignored anything like that before; now, he was able to \emph{feel} friendship, something he just pretended before, when all his emotions were concentrated on O. 
[quote]
I've been cheated by you since I don't know when;
So I made up my mind, it must come to an end;
Look at me now, will I ever learn? 
I don't know how but I suddenly lose control,
There's a fire within my soul. 
Just one look and I can hear a bell ring,
One more look and I forget everything. 
 
Mamma mia, here I go again,
My my, how can I resist you? 
Mamma mia, does it show again?
My my, just how much I've missed you. 
Yes, I've been brokenhearted, 
Blue since the day we parted, 
Why, why did I ever let you go? 
Mamma mia, now I really know,
My my, I could never let you go. 
[/quote]
So, you may ask, why is this text here? I guess everybody out there knows where it comes from; thus, I won't have to explain. 
It's just a part of a song, and it shows how I'm feeling at the moment, though I've never had a girlfriend; but being in love and having a girlfriend isn't that much a difference as it seems. There are several parts of a human being that can be in love: His brain, his body --- and his heart (or his soul, if you like). 
Real love combines all these several aspects; but if you feel connected or drawn to somebody by the innermost feelings of your soul, it's not really of importance if the other parts are in love, too, as they will fall in love, when the time has come. 
However, there is one thing everybody should have achieved but rare people do: Being in love with somebody, even if there's no chance of a relationship. 
I'm not telling you that only some people are in love, but most of them don't feel this fire, this burning in their souls, the butterflies in their bellies or the twitching in their guts; Most of them don't spend their time really thinking about their partner, and they are not capable to write such texts which are inspired by real emotion. 
There is no such thing as talent; everybody is intelligent, it just depends on the methods he or she is applying to learn. 
What I'm telling you is: \textbf{Absolutely everybody} would be capable of writing something like this, if the motivation is real. 
For me, it is. 
I guess, it's the same for most important writers or rather story-tellers: All of them draw power from their emotions and pass it on, doubling and tripling the energy that would consume themselves if they didn't. 
Maybe, this is the reason for our world drifting apart, for our educational systems failing and all the misunderstandings leading to wars. 
Decide on your own, as you now were given insight into some of the surprising sensations that keep chasing me around the globe. 
Hoping to receive your opinions................
To be continued...
This week, by Wilkie Goldentongue...

--------------------------------------------------------------
(If you want to comment on this, please start a new thread; this one is reserved for this story, my life, the story that'll never end, while the end is still near, coming nearer every minute; it's just around the corner...)


Do you recall the time
when the snail slurped its slime?
when life was worth a dime?
and when the sun didn't shine?

Or do you remember the day
when belief led astray?
when you were heading the wrong way?
and when there was nothing more to say?

I hope you say 'Nay',
as that is the day,
when my love for thee,
would cease to be. 
\emph{W.G.}

Sense is sensation --- and exasperation. 
\emph{W.G.}

%%% Local Variables:
%%% mode: latex
%%% LaTeX-command: "latex -shell-escape -synctex=1 --file-line-error-style"
%%% ispell-local-dictionary: "british-ize-w_accents"
%%% TeX-master: "../My_Life_in_a_Nutshell_-_First_Era.tex"
%%% End:
