Hello, again, nice to know you're still with me!

I'm glad that you still read further; and that I'm still capable of continuing that story of mine. But before the introduction will become as long as the last, we should concentrate on the text:

\textbf{Here's another extract, just for you, and especially for O.:}
\textbf{[u]Bless you![/u]}
It was Thursday, the day he'd waited for so long. 
And that was to be his fault; as we already learnt, hopes are never fulfilled if they are of real importance to us, and this basic law of existence never failed. 
It all began in the morning; hoping she would come, but not doubting this idea strongly enough, he freed a seat for her, right next to him. 
He'd always done that, in times that were gone now, as she took a car to reach school. 
But today, something that was the case about a month ago was repeated again; at that time, she took the bus, because there was no other possibility. 
As you may have guessed, she didn't do so today. 
Our protagonist had reserved a seat right next to him, and carefully chosen it in a way that there won't be too many people who would ask questions to her or to him nearby. He realized that this could offend her, if she really tried to stop the friendship they had; but if that was the case, she'd have to stand for more than half an hour in a completely filled bus. 
Otherwise, she'd have to ask somebody else, but then, he'd at least have an answer. 
But fate seemed to think differently: He didn't know how she made her way to school,but as there were several other possibilities, he didn't have to search for a logical explanation for too long. 
When the bus arrived at the place where she would enter, he sat on the other side so he couldn't make out the car or her house. 
But he could see the bus stop,and she wasn't there. 
Another girl standing next to him said to him in quite a desperate way: "This place's reserved, isn't it?" He took a final look around, before the last spark of hope seemed extinguished; then, he had a look at this person and told her: "Indeed, this place had been reserved; but nobody seems to be coming." 
Then, he took his bag and placed it on his legs, thus making room for her to sit down. Finally, after she sat, another girl took her seat on her lap, and he stared out of the window for the next 45 minutes that were to come before he arrived at university. 
Finally, he began to sleep, as he was quite drowsy, and he didn't even notice when the place next to him went empty; but finally, he arrived at university. 
The bus was late, as all buses are; and he had to hurry up, though he knew he'd arrive too late at the lecture he was joining today. 
When he passed the courtyard, one person was approaching him; it was O.
She just passed him, greeting and smiling, while he did the same, if we decide to ignore his still rising heartbeat that seemed to have reached its climax just before he met her and was now rising even higher. He went more slowly, though he knew he was late; there would be three explanations for her coming from that direction, and he decided to focus on the most probable one: She had just visited the toilet, as her lecture started later. 
He knew that, as he knew all the lectures she was joining and the times they started; sometimes, he thought he knew even more about her than about himself. 
On the other occasions he'd hoped to meet her, he couldn't, as the lectures that would take place in the afternoon had been canceled, but his hopes had kept him from realizing that yesterday. 
Nevertheless, that would mean that she would finish at the same time he would; but on the other hand, she had a car she could drive, as in the afternoon, the problem that should have made her use the bus in the morning didn't apply anymore. 
But she didn't seem to use the car; maybe, she was together with her boyfriend and he'd catched her from university; but he was not to find out about that today. 
He'd just see the car in front of her house on his way home. 
And tomorrow, the chance to see her was nearly equal to zero. 
He remembered happier times, contemplating his faults; the moment when she'd come up to him just to tell him about her marks, escaping an auditorium to meet him, as he was passing by; the occasion when she'd planned which bus they would take to be able to talk on their way home; or the moment when she advised G. that she wanted to sit next to him; or the other time when her boyfriend was sitting to her right, while he sat to her left, and put her arms around his shoulders two times in gestures of friendship. 
These times were gone, but he kept everything in mind, though it would probably crack him up. 
But there was something else, something special about that day; I've left it out in the beginning, as you should have wondered about the title. 
He noticed today that he was becoming more and more integrated in all those social groups you find at university, school or every place people meet at. 
One result was, that he seemed to be the only person who always said "Bless you!", no matter who was the victim of that illness that made you sneeze, and also disregarding the fact that one might sneeze for three or four times. 
That was just the change one may have noticed who knew him; but there was also something else. Starting this week, and maybe also the end of the last one, he began making friends with everybody, even people nobody of his old friends seemed to know. 
He adapted. 
Maybe, just to cope with the sadness her behaviour was imposing upon him. 
And he did quite well: Soon, he knew loads of people he'd just seen several times before, new names were jumping at him from all sides, and the moments when he didn't have to think of her gained importance; but when he was alone, he was more lonely than ever, and his suppressed emotions seemed to come back in the eruption of some hidden volcano. 
We've already learned that everything in life is balanced; the Chinese happen to call it the "Yin-Yang", others the "harmony of the universe", but most people simply call it "religion"; or "God". 
You may choose for yourself. 
But it's cruel to know that the strong emotions we experience are finally all balanced and thus just the normal vibrations of life, that allow us to feel anything at all. 
These eternal vibrations are the thing our scientists are struggling to find out about, without even knowing it exists. 
One may imagine a simplified model of a sinus, a curve going up and down; but finally, if you substract the areas below zero from the area above you'll see that the result will be zero. That's the way our life is made: The rising entropy is the result for the impossibility to predict anything clearly, and the only possible way to do that is to try becoming a part of it; but even then, only percentages of such probability may be given. 
That's the basis of life; and thus, also the basis of love. 
Nobody really can predict both; but if anybody tries, he'll probably join the garbage of rising entropy and go back to where he came from:
[quote]
"In the sweat of thy face shalt thou eat bread, till thou return unto the ground; for out of it wast thou taken: for dust thou art, and unto dust shalt thou return...."
\emph{The Holy Bible}
[/quote]
Dust is the symbol of this entropy, and the symbol for life, as life is not predictable; if you try, you'll have to go back; you'll have to die, slowly cracking up. 
Always keep that in mind: We can try to find the GUT, but if we do, this could be the end of our existence, as knowledge always is a weapon. 
I hope you understood what I wanted to tell you; I know it's not easy, but life and love aren't, either. 
But that was just an explanation of the logical analysis I often apply to things it shouldn't ever be applied to, and of the danger that can result from such abuse. 
One may say abuse is another basic law of human existence; I won't contradict him, but some laws should not be obeyed. 
I closed his eyes; it was summer again, the summer of O. The sun was shining, the two of them were sitting in the bus, near to each other: He looked into the blending light the sun was spreading, and soon, he was blended in a way quite similar to the moments when he had a look at her. He remembered the time when she'd told him about her favourite flower, and he found the light emanating from her shining eyes in every sunflower he saw since that moment. 
The vision was stopped abruptly, and he found himself back to present, eyes wide open: He was shivering, as it was winter, early in January and cold even inside, though there was no real snow outside; she liked snow, and he wished for it, too. 
Maybe, this was the reason why it wasn't snowing. 
He decided to ignore this idea and closed his eyes again; the vision was gone, but her face still was there, as if it was burnt into the backside of his eyelids, and it wouldn't go away like a tattoo that would stay forever. 
However, a tattoo could be hidden to the own eyes; the picture on his lids couldn't. 
I warmed my hands, but they stayed cold, remembering the things that could have been but now were lost for eternity. 
Opening my eyes, the vision of an image still stayed as if imprinted into the retina. I decided to ignore it and had a look at some of the things I had to do for university, but concentrating was becoming more and more difficult in the last days. 
The withdrawal of feelings I had received from her was now a strain on my soul and my conscience and soon became a fact I couldn't deny. 
But I couldn't tell her, either. 
A fixed relationship is something that's an important decision, as it kills all mutability; and she seemed to feel the same way, and her decision was made. 
If her boyfriend doesn't change his, I guess, nothing would change. 
I know it's hopeless, but the last tiny patches of hope that are left keep me alive, and I don't want anybody to take these away from me. 
We'll see if the basic rules I've worked out --- that real hopes are not to be fulfilled --- will be proved. 
I hope they won't. 
I'm looking forward to meet you again, here. 
Hoping to receive your opinions................
To be continued...
This week, by Wilkie Goldentongue...

--------------------------------------------------------------
(If you want to comment on this, please start a new thread; this one is reserved for this story, my life, the story that'll never end, while the end is still near, coming nearer every minute; it's just around the corner...)

One may guess;
but only who is preordained to die,
may know. 
\emph{W.G.}

Who will end the curse that's holding me in its grip so tight?
I can't escape;
I need the source of light,
that's sealed up itself;
ever so tight. 
\emph{W.G.}

%%% Local Variables:
%%% mode: latex
%%% LaTeX-command: "latex -shell-escape -synctex=1 --file-line-error-style"
%%% ispell-local-dictionary: "british-ize-w_accents"
%%% TeX-master: "../My_Life_in_a_Nutshell_-_First_Era.tex"
%%% End:
