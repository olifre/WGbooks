Hi, I'm back to the living, once more!

This time, we'll have some shorter text, again; tomorrow, there'll probably be some silence, but then, I may be capable of writing some more. However, this also means that there will be less to write down, then. 

\textbf{Suddenly important again:}
\textbf{[u]The Silent Sun[/u]}
Before we deal with the present, we shall have a look at what some important words had revealed him on Saturday evening and Sunday. 
Now, he was just looking at the bunch of notes he'd typed; digital reminders of what he wanted to tell you. All these cryptic signs that nobody could understand; letters without visible connection, sometimes encrypted, sometimes simply arranged in a special order. Our protagonist was always in favour of concealing himself, and he had to do this as perfect as he'd always tried to do things. His special encryption would not even reveal its meaning when shown to the people who knew a lot about him; or at least, the people who \emph{thought} they knew about him. 
We've learned that he had offended B.; and just some days ago, on Saturday evening, a message had reached him, telling him that she had felt offended, in fact, and explaining him that she had blocked that thingy. He'd answered her using the same means of communication, but when he'd ask her on Monday if she'd received an answer, she hadn't. He just told her he'd send it again, now using another way of sending it, and choosing the words once more so as not to offend her again. 
This new message was sent on Monday, in the afternoon, and no answer has been received up to now. 
He'd also sent something to R., whom he'd thought to ignore him; he'd told her in quite beautiful words that wondered why she was always leaving when he arrived; and she'd answered on Sunday, telling him that she wasn't doing so on purpose, but that it was just some coincidence. She stayed some time to tell him all that, though she'd already told him that she wished to go; and one moment later, she was gone, without a word of goodbye. Nevertheless, everything that had to be said had been said up to then. 
On the other hand, he wasn't sure that she was really telling the truth, but he believed her, as he knew some things about her character; she wouldn't feel that a word of goodbye was so important as he thought it to be. sometimes, she seemed as vivid as O. or G.; then, she shared her time with B. and also had some of that deeper knowledge that he assumed B. to have. 
He was happy when B. had told him on Monday that she hadn't received any message, as this told him that she had not ignored him; for some moments, he'd assumed she had, and he knew that people could easily betray their same kind. 
P. was now becoming more and more interested in the things he did, in a way of friendship; the two of them stuck together quite closely, though he felt that 'closely' would mean something different to her. However, he'd learned a lot of things about social interaction from her, and he was happy to be able to accept that knowledge. Knowledge he hadn't been given in his youth, when people didn't make friends with him........
The exam he had taken today was easier than he'd thought, as he'd believed that all those new sensations that this knowledge included would distract him, including the fact that P. was in the same room. Before, he'd noticed something different: On Monday, Y. had asked him to do something, and he'd thought about the possibility of her taking an exam in the same room the very next day; it was just some minute later that he'd realized that O. would take an exam just then, and he felt he'd betrayed her memory as he'd not thought of her directly. On the other hand, this memory wasn't conscious anymore, and she was now a 'normal' girl to him, as far as that was possible with her. 
He felt he should have said something like: "I don't want to keep you from doing something important," to R. when she was talking to him and when he knew that she was about to leave. This just occured to him some minutes after she'd left, and it came back to his memory right now. 
He took a look at the last lines he'd written down. He felt he'd changed the topic without thinking about doing so, as his mind had just given him some other memory; he was subconsciously forgetting about O, not knowing whether this was good --- or bad. 
On Monday, something else had happened with P.; he'd been with her when she had been sorting things, and she just threw away some things he'd written; she'd read them once, probably, and now they would be gone forever. Even the possibility that she'd kept a copy at home was not really probable, as she would not have destroyed this one in front of his eyes, then. There were some people calling her beautiful names, and he knew in which way she was different: She was into society, and at the same time, she wasn't. That is to say, that she was completely integrated, but she was not to notice things unless they were told to her directly. And even then, she didn't realize some emotions; on the other hand, she could give away masses of positive feelings, most times not caring about possible reactions. 
How did he figure that out? It was a talent of his, and he was believing that he was not only crazy, but spooky, sometimes, for knowing about people --- about animals of your own kind --- is to know part of yourself; and he who knows himself is either a maniac or a genius which is close to suicide. He hoped he was neither, and if he was to be one of these, he wanted to be the maniac; but probably, he could not decide on that, as it was the same. 
Now, you may ask why this text here is entitled "The Silent Sun". There was a song, by "Genesis", and it reflected something we're yet to deal with; somebody, and some action this person took just today. Y. 
You may remember that he'd promised her on Monday to do something for her; well, he had. 
On Tuesday, the two of them would meet. They had. However, she'd (of course) forgotten about it and arrived late, but not too late; they went to some more silent place, and he showed her what he'd done. 
A magnificient reaction was the result; she wondered whether he'd really done that for \emph{her} and gave him a \emph{(friendly)} hug. He felt that it had been worth the effort, and he realized how different people react on what you've done for them; some wouldn't mind, some would say "Thank you!", and some would give you a hug, even if you just did something that wasn't really extraordinary. He was happy about Y.'s reaction; it showed him that the sun was still out there, and that the flame of friendship was never to be extinguished; of course, he was P.'s friend, but this was something different; Y. was something special in different aspects. Her warm-hearted character and her sense for literature gave her some special atmosphere, while she'd also be easy to be betrayed, as she didn't believe in the bad that was there, in all of us. 
This made that day something special; but were not yet finished with it. Some minutes later, several students would pass behind him, O. among these; she'd be the only person who'd bump into his bag, saying sorry and going on without really turning around. She'd changed; the care she'd once shown towards him was gone. 
The fire in her eyes didn't show up if she dared to meet his look. Something was gone, and he felt he'd coped with it; however, he wasn't sure he'd done the right thing. On the other hand, he felt she hadn't, though he wasn't sure about that. 
Still, there was something inside him; the clock that had today --- once more --- shown the numbers which linked her to his life was something he wished to throw out of the window for some seconds; of course, he'd never have done such a thing, but he felt the sense of it clearly. 
B. was silent that day; probably, she hadn't received that message he'd sent again; but he was not to tell her. Then, he'd seen his best friend walking just some steps behind R.; but somebody else was soon in front of him, and he gave in, not pushing through the door to keep close to her. He wondered whether he'd ever have done so if he had been behind O. Probably, he'd have; but this imagery of his best friend not advancing towards her reflected his own past. 
Some people were to undergo the same brutal fate; and who could judge if this would help or destroy their lives?
In the bus, he'd once more talk to the girl he'd met there more frequently some weeks ago; he realized that he could really make friends with everybody, but he felt that this could probably not be a substitute for something \emph{real}. 
Please join me again, when the next text arrives; we should find out together how the story decides to continue. 
I don't want to bother you, but some opinions would come in handy........
To be continued...
This week, by Wilkie Goldentongue...

--------------------------------------------------------------
(If you want to comment on this, please start a new thread; this one is reserved for this story, my life, the story that'll never end, while the end is still near, coming nearer every minute; it's just around the corner...)

People 
around him 
were always coming to him, 
making friends. 
Friends 
around him 
were always giving him tasks, 
help --- and sorrow. 
He 
deep inside him 
was always searching 
for himself. 
\emph{W.G.}

They 
can take away our freedom. 
They 
can take away our speech. 
They 
can take away our hearing. 
They 
can even take away our lives. 
But they 
can't take away our conscience, 
as thought itself is not mortal. 
\emph{W.G.}
