Wilkie's back, once again!

Too much's happening simultaneously, and thus, we're to stick to the most important things. And I'm afraid I can't write more frequently, but that's the problem of writing while attending university: The more things happen, the less time there is to write about them --- and the more important things to write about there are. 
Despite that fact, I'll try to go on --- maybe less frequently, maybe a bit shorter --- but the story is still not finished. 

\textbf{This one drives me crazy:}
\textbf{[u]Do only Fools Rush in?[/u]}
He was wondering whether he was really right, or if everything he'd thought to be so true was completely --- nonsense. 
It was now Thursday evening, and a lot of things had happened --- O. was not to be seen. 
But P. and her friends were. 
And all the rest of the universe, of course, as life would never stop --- at least, not so soon. 
But he was still remembering things, which made all of it even worse. Of course, he'd tried to suppress such thoughts; on the other hand, he could not pretend that he hadn't noticed the name of that female protagonist in that film some days ago, that terrible thing that showed him clearly that everything would be over: The name of that protagonist was O.'s name, and one of her first boyfriends' name was the male version of G.'s. 
Such things didn't happen by chance, not in \emph{one} film; if there had been two of them, and some days between the emissions, he'd have accepted this as a product of chaos. 
Now, he had to believe that there \emph{was} something out there, indeed. 
And it was powerful. 
Before we'll start over with Wednesday, we shall have a look at one of the most special things that happened just today: One of his friends said something to him, somebody he had always noticed but never thought to be \emph{that} important, though he'd always felt that there was something special about him. 
They were just comparing two books at university, and that boy was telling him something he thought to be important. 
It was. 
It was his theory. 
He said, that he somehow made out the way the lives of the protagonists develop, their happy times followed by hardship and vice versa. 
His hand described the form of a wave while he was saying this. 
Though he didn't add that this meant the areas above and below summing up to zero, he was so close to his idea, that he was shocked; 'Cause if two persons are having the same idea, just at a different time, and in a different situation, the probability that it was right was very high. 
He'd once heard that inventions had always been made for at least two times in different parts of the world, without one scientist knowing about the other --- Was this something like?
He felt the importance of this knwoledge, and when the professor advised the boy --- who was only talking to \emph{him}, indeed --- to be quiet, the subject wasn't talked of again. 
That was one of the most important things that had happened in these two days --- however, there are more to come. 
For example, L.-B. was phoning him again, pretending to want to be explained something; he did so, and when she wanted to talk about other things, he tried to ignore her, until his mum called him for supper, relieving his heart which was not born to cheat, though profiling is something quite equal to it. 
P. seemed to be really making friends with him, and he'd also embraced her friend that day \emph{(well, probably the other way round, as she'd taken the initiative)} in quite an equal way P. had embraced her seconds before --- as a sign of friendship, of \emph{missing} one another. But he knew that this embrace was shorter............
He felt it. Was she feeling more, thus being not capable of embracing him just like a friend?
He wouldn't know, and he wasn't sure if he really wished to find out. 
The other friend of P. he'd found on the internet was really nice, and she answered him every day --- it was some kind of ping-pong conversation, but the two of them seemed to care for good answers. 
\textbf{True} answers. 
He was happy with that, and with most other things happening all around, though he felt that something had to be going wrong --- soon. His theory had never been proved wrong, and he was sure that this wouldn't change. 
He'd been sitting next to P.'s friend for more than one hour on Wednesday, and he'd enjoyed it the same way she seemed to have. On Thursday, he was together with Y., but she was really tired and dizzy, not really capable to talk to him, though she laughed here and then; Nevertheless, he noticed that \emph{she} wasn't enjoying his presence as much as P.'s friend did. 
He also had some things in common with both of these girls; on the other hand, they were completely different. 
Wondering whether he was now repeating the same fault of waiting too long which had rendered a relationship to O. impossible (at least for the moment), he realized that the TV-Show he'd seen was right. 
There, several scientists had analyzed the phenomenon of love, discovering that the decision to love somebody was finally a logical one. 
He'd always known that; everybody did, but everybody also pretended not to know, so as not to render love impossible at all. 
Such an explication would make this strong feeling something that would not be special anymore; However, the scientists had also got to know other things. Love was able to block certain areas of the brain; Thus, one would not notice the flaws of the target of his feelings. 
Which would explain his sudden burst of knowledge when it was too late. And, in addition to that, it told him he should try to \emph{stop} being in love, so as to notice whom he was really in love with.........
He'd already begun to do some things: You should remember the time when he was listening to that song for hours; well, he had increased it's speed to exactly 119 %. Firstly, the checksum of that number was O.'s number, the two; Secondly, the voice of the person that sung was then much closer to O.'s voice, and he could imagine her singing it, as she liked singing, even in public. Well, not as much as her best friend did, that is, as she could sing \emph{really well}. 
It helped him to cope. 
But it would still take him some time to lock away the things that had happened; Nevertheless, he would just feel as something like a friend of her, or probably, even less. 
A shadow was passing over his face, as he felt signs closing in, reminding him of the early past, even beyond his memory of O. and G.; A time when he was mocked at school. 
And he noticed, that there were some moments --- even days or weeks --- he couldn't remember, even if he wished to do so. Only the consequences had been saved in his mind: Two or three bad marks, though he could have done much better, and he wouldn't know \emph{why} he hadn't --- Was this some psychologic influence? He couldn't even recall if the other pupils had blackmailed him in some way. 
His brain felt empty when trying to remember such days, though some of the moments could be remembered quite well; vivid and full of detail: Himself riding on a bicycle, talking to one of the teachers, standing somewhere alone, hoping not to be mocked; and when he was writing these lines, these silly letters full of real experience, some things were coming back to his mind, and a shudder ran down his spine. 
It was cold, and grabbing hold of him, as he was gaining deeper insight into the past --- it would take him a long time to find out about everything again, beginning with the mockery in the time before school started; the days he'd stay inside, happy with himself and all those games and devices. 
It had been a happy time, the years before it all started, before the mockery and the bullying; it was gone, and just a shimmer would remain, suppressed by the hurtful and now locked off events. 
He'd have to cope with it, and his head began to ring with fear --- However, he knew that this was the only way to gain individuality that wasn't part of his profiling system. 
Though he knew that hope was useless or even worse, this hope had ruled his life for long, but he had ignored it up to now, thinking these memories to be less important than they were. 
One could only heal himself; even if a psychologist or a psychiatrist could give you introductions --- and he was sure that he didn't need these --- it would be up to you in the end. 
And every genius of these days and the past had something disgusting in his memories he could never forget and probably never know about. 
Once and for all, he wished to find out, for the sake of himself and the world. 
The world --- is it still out there?
Are you still reading?
Or are you all gone?
Tell me, please, as these letters are My Life --- in a Nutshell...........
To be continued...
This week, by Wilkie Goldentongue...

--------------------------------------------------------------
(If you want to comment on this, please start a new thread; this one is reserved for this story, my life, the story that'll never end, while the end is still near, coming nearer every minute; it's just around the corner...)

Sometimes, 
sitting close to somebody in silence 
is a moment of conversation; 
Sometimes, 
sitting close to somebody in steady talk
is a moment of silence. 
\emph{W.G.}

Can one break up 
if there was no relationship at all? 
One can. 
Can one feel so close to someone else 
that one would tell this person everything 
though there is no relationship at all? 
One can. 
Can one feel the claw of death trying to grab hold of his soul 
when breaking up 
with somebody 
one's worshipped without knowing about that? 
I tell you: One can. 
Can one escape this claw, 
this power to exhaust one's candle, 
without exhausting it oneself? 
I hope: One can. 
\emph{W.G.}
