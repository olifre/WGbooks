Hi, this is Wilkie again!

Till now, I just showed you the tiniest patches of what is yet to come; in that text here (the longest up to now), you're invited to follow me to the first time I denied my feelings, not realizing having these at all...........

\textbf{Hoping you're still with me, here we go:}
\textbf{[u]The First Denial[/u]}
Saturday.........
A day he wouldn't see her, and his attempt to contact her was yet to be proven successful. For that reason, his thoughts focused on this happy creature even more, wondering where she was and what she did all the time. 
The time has come to exit his mind and have an objective look at the scenery, something he liked to do all the time, as far as it was possible to escape oneself. 
And he wasn't really capable of doing that; otherwise, he'd have noticed that she didn't have a boyfriend anymore, and that O. was starting to like him, himself just mirroring this love back to her without letting that feeling come close. He missed her, when one day she wasn't talking to him; he was excited, when G. sat next to him, but he was happy, when O. entered the room. This difference is a major thing that separates a deep love and just physically appreciation of beauty; he couldn't have told which one of the two girls was prettier, but his decision --- the selection --- focused thoughts like these on G., leaving O. behind. Subconsciously, he felt what was happening, but these subtle feelings were hidden and not even realized by himself; real love was bound to grow and needed time, and instantaneous preference for somebody he'd "\emph{selected}" --- that was something that could never be the basis of a final unity. The poem he recited last time shows what this means --- thus, he'll post it here again:
[quote]
The mustard seed was planted,
the gift has now been granted. 
Just now it is bound to grow,
what it is does future show.
W. G. 
[/quote]
Real love is a seed that's capable of planting itself, growing all with time, just needing this fourth dimension to reach perfection; but time would be too short, and the explosion the plant would cast off when it was big enough to be noticed by him would come too late. 
We'll go back to the \textbf{"Day of the First Denial"} when there still would have been enough time to change everything and prove that real love was possible even in these days. 
They sat there, facing each other in the bus. She was sitting looking forward, in the direction the bus was heading; he was sitting right in front of her, facing the other direction, thus bound to look into her eyes all the time. His emotions were controlling his eyes, and her emotions were controlling hers; thus, both children or young adults were looking into their eyes, steadily talking, not capable of looking into another direction. Another girl, attending the same place where they studied, was sitting to the left of him, but about one and a half metres away; sometimes, she must have looked at him; he'd find out when he would deny his feelings by looking at her reaction, as she must've sensed the powerful emotions that were escaping the eyes of the two lovers, which would be the foundation of new stories and consciousness. All his senses were focused on her eyes, the mirrors of her soul, and he experienced what is was to be in love. 
And then, he denied it. 
Just after she'd gone, another little boy sat down next to him; both were smiling, and he asked him whether he was in love with her. The other girl that he knew, too, watched him; the bus was nearly empty, and she was giving him a warm, understanding smile, when the question finally was put to him by the small boy; she was together with her boyfriend fo quite a long time, and she still loved him; it was one of the most stable relationships he'd ever seen with people at that age. In a time that wasn't longer than a second, he thought about O. having a boyfriend, about G., about everything he'd experienced and shared with both of them; he knew he was in love, but he would deny when he was asked, as he was denying his feelings to his own consciousness; he couldn't be in love with somebody who had a boyfriend, not at that time. 
Thus, he denied his emotions. 
The other girl in the bus stopped smiling suddenly, and the boy asked him why he'd looked into O.'s eyes all the time and why she'd looked back, if they weren't loving each other. 
He answered that it's normal to look into one's eyes when talking to that person, but he knew it was not that kind of look that connected them. He wanted to add that he couldn't be in love with her, as she'd a boyfriend, and if he had, the other girl would have told him O. was a single again; but he didn't add that, as he thought it would just be a reason for the small boy to ask other questions, and he wanted to be left alone with his pain. 
What followed, was even worse: He didn't stop with denying his feelings to others, he wrapped them up in a package and stored them at some place in his mind where they weren't to be found again till a long time had passed. 
And the pain he'd experience then would be enormously brutal. 
But he didn't think of that. 
He started to see O. even more often, and enjoyed talking to her and listening; finally, we'll come to the moment he heard of her new boyfriend, but before we'll come to that, we'll add something we'd forgotten: What were they talking about?
We'll have a look at a Chinese saying:
Persons who are in love with each other say a thousand words without talking. 
But as the thousands of words were spoken, and as he was a quiet boy, he searched for other things to talk about; he listened closely, when she talked about books she liked or was reading --- and he read them all, at a speed only lovers can obtain. 
Most times, he didn't find an appropriate moment to talk to her about the contents. 
Sometimes, he did --- but then, he didn't tell her he just read the book because she'd done so. 
I think, every human being who is capable of feeling emotions will have realized by now he was in love with her, but he still denied it when he was asking himself about his relationship to her. 
About two weeks before the holidays began --- weeks of time when he wouldn't see her --- he was in the bus again, sitting together with O., G. and another girl. G. was asking O. if she'd kissed \emph{him}, while she didn't explain which \emph{him} was meant; but of course, it wasn't the him we are accompanying right now. It was a friend of O., and she didn't answer, which was the clearest way of telling that she did it. G. --- and he --- were smiling, joining the other girl, while she said that she'd kissed him, just repeating the thing everybody who was present already knew. 
He was smashed, but the smiling saved him, and only seconds were to pass before his head slowly moved to look out of the window. 
As we know, he was good at denying things, and he's denied his love for her; thus, he also didn't make the connection that she had to be single before she found another boyfriend. Everything he saw was saved in a corner of his brain; and about one week later, he saw the two of them kissing each other. Just before that moment, he'd asked her for a photo, and she'd agreed that he could take a photography of her together with G. and some other girls; as he saw their lips touching, he turnt his head as fast a she could, not even keeping an image of HIM in his brain; that was something he'd do later, on another day. 
He exited the room, and just walked away, not thinking of anything anymore for the next hours, just using the profiling system he'd always used with everybody. There was no need to start thinking when talking to somebody; it was superfluous, and most people didn't do it, as most answers and most questions were always the same; he simply answered what he was expected to answer, and that was enough, as nobody noticed anything. 
The other day, she asked him for the pictures he'd taken that day, and he gave her these; she didn't really take an interest in them, though --- and she didn't seem to remember having promised him to stand there for him to take a photo. 
He didn't remind her of it, either. He didn't find a sense in that. 
It was over. 
At least, that's what he thought, but as he'd denied his feelings and still was denying them, there was no obvious chance in his consciousness; but his subconsciousness began struggling. 
Just a day before the last day of the last week before the holidays, it broke through. 
He consciously experienced what had happened, and all the memories he'd locked away came back, bringing the eternal pain with them. He managed not to cry, though he was home alone, lonely. 
He invented the hope of seeing her tomorrow, and the hope that her new relationship wouldn't last long; but as concerning the present, the couple still seems to exist. 
Thinking of all the memories he had, there was something important he noticed: She'd stopped looking into his eyes, and the amount of words that escaped her lips was shrinking after she'd \emph{fallen in love} with \textbf{someone else}. That was something that proved his feeling that she'd loved him --- and it increased his pain to a level at which is was nearly unbearable. 
Caressing his hopes and denying his pain, he started to enter the state he was in before it happened; now just \emph{knowing} what had happened consciously. 
The next day, he was just able to pass her quite closely; no word was exchanged, and he couldn't even wish her a nice holiday, as she was talking to one of the teachers and he'd miss the bus if he'd wait till she finished. 
His holidays consisted of memories boiling in his head; and he even remembered two other scenes that should have proved that their love was to be perfect; he'd ignored them consciously, but they were saved for eternity in the complex memory that makes a human brain. 
We'll have a look at these two important scenes in the next text; It'll be called "Adam, Eve \& The unambiguous Signs of Love".
To be continued...
This week, by Wilkie Goldentongue...

\emph{
BTW: I'm sorry for announcing "The great Betrayal of Belief in Fate"; it's included here, as his subconsciousness believed that fate --- or some almighty power --- would prove a perfect love to be reality. 
BTW: As you're German, you probably won't know what BTW means; it's "by the way". 
}

--------------------------------------------------------------
(If you want to comment on this, please start a new thread; this one is reserved for this story, my life, the story that'll never end, while the end is still near, coming nearer every minute; it's just around the corner...)

Liebe naehrt Gegenliebe und entflamt zum Feuersbrunst, was sonst Aschenfuenkchen bliebe. 
\emph{Gottfried August Buerger}
\emph{(Yes, I was teached German at my school; and we also had a look at German peotry. If there are some errors in it: Sorry, but it's not my mothertongue! You are free to tell me about it (in English) in your "Meinungen"-Thread, if you like, or correct them yourself, if you happen to be an Admin.)}

I'd commit a crime (and pay a dime)
to turn back the wheel of time.
\emph{W.G.}
