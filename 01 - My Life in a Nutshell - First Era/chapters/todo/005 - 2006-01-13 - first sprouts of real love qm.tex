Hi, I'm back again!

WEEKEND! Time to start over and begin with the important parts of my life; the parts that a re influencing my thinking just now. Finally, I'll then write some daily or weekly report, showing what thoughts are swirling in my mind and why they don't return to where they came from. But to understand just the tiniest bit of my personality, you'll need to go on reading:

\textbf{Here we go, another time, another day, another text:}
\textbf{[u]First sprouts of real love?[/u]}
It's Friday, the 13th; a day most people fear. However, this Friday was different: He didn't \emph{know} it was Friday, the 13th, that is, of course he knew it was the 13th day of the month and Friday, but he didn't notice these both conditions were fulfilled at the same time. All the other Fridays like that he could recall always were quite happy days; but then, he'd realized their existence from their beginning. Knowledge can change the world, or at least the way we regard or experience it, and thus, this Friday was normal to him, just up to the moment around noon when somebody said that it was Friday, the 13th indeed. At that time, this day or at least the way he experienced it changed: But finally, you'll notice that thought can change reality or even is its basis. 
That day, he knew he wasn't going to see her and if he would, it would just be for the time of a short blink with his eyes. However, things were different. Not so quite, but at least he was allowed to see her two times, and for a duration of something about two or three seconds, which was much more than the blink of an eye, at least to him who was able to save the experiences of seconds in a detail normal human beings use when they sit together with each other for hours. 
The first occasion was exactly what he had expected: She'd passed him by, walking to his right just past him. While walking, she talked to another person next to her, but even his trained senses were unable to focus on two persons if she was among them. As a consequence, he just renewed his picture of her in his mind, something he did every time he saw her; but what really made him afraid was the undeniable fact that she changed and became less attractive to him. Her colour, the most important part of one's aura, was becoming darker and in danger of burning out and finally disappear completely, and this was the thing all the persons he really liked had in common: That colour that painted his own life that was to be grey from his birth till today. He didn't know if he was ever to become aware of his own colour, though he believed it was some mixture of blue and red; and he didn't stop trying to find himself and thus his colour of soul, while she seemed to loose her orange / red fire. 
As you may have noticed, we are talking about O., as O. means nothing else than orange; you can figure out for yourself what green means. 
He knew that there were people who claimed to be able to see these auras and wrote books about it; and he couldn't help but doubt their abilities, as the colours he saw seemed to stop glaring and became hueless and dull. He was one of the only creatures to notice this loss of souls or at least qualities of soul with his eyes that saw everything and couldn't be betrayed easily; Nevertheless, he was also infected with the virus that caused human fantasy and knowledge to die, and he had just exterminated that sickness of his soul when the next one came. 
But we'll now see when and where he saw her for a second time: She had just exited her car and was walking, while he was watching behind a wall of glass, unable to reach her, and the scene was cut some quarters of a second later. Most interestingly, he'd looked out of the windows some quarter of an hour before this event, thinking that --- once upon a time --- somebody would read his story and try to interpret the fact that the glass he was staring at was quite dirty as a border that separated him from the world outside, sitting alone in a dirty cage of glass, that was a normal bus. Interpretations like that were things he always did; Thus, he knew that nearly every story can be interpreted not just wrongly, but in a way the author himself didn't intend or didn't think of though he'd intended it. As a consequence, luck is the most valuable friend of all important authors. 
One thing was now pretty clear to him: She'd changed, while he'd tried to stay the same; Something must have happened to her, and he wanted to find out, though he couldn't do so easily. However, he used all means he could, and if he was successful, is something he's asking himself at the moment; we'll soon find out, hopefully. 
For the moment, we'll continue by analyzing the events of last year, the year before and the way he stopped to be in love with G. before we'll go on to "The Great Betrayal of Belief in Fate" in one of the next parts. 
We stopped by describing the moment when his hand shook more than ever; We'll go on with the way O. entered his life, preparing the description of the terrible way fate had chosen to cheat him, just explaining some things he only noticed without realizing them consciously. 
One morning, she began talking to him. As she was one of G.'s friends, and as G. was with her, he just answered to her greeting as smiling as he could. This greeting was the first thing that connected the two of them; and he was always happy to hear her voice in the morning. One day, he was so sleepy he couldn't really answer her call clearly; Her reaction was to shout at him that she'd said "Good morning", and he'd shouted back as soothingly as one can shout that he had answered her call, indeed; but as she was just passing by, he wasn't sure if she was still listening and his voice was faded out by himself on purpose. 
One has to know he was a man thinking that the persons he liked or loved remembered at least some details about him; but nearly nobody did, and he was always hurt when he found out about that, but the steady greeting that was never forgotten gave him a warm feeling, and sometimes, he was even looking forward to it, and sometimes, it didn't happen because she wasn't there, especially on the days he hoped for it with all his thoughts. The next step was done about some weeks later, and also this time, she was the person that was acting: She sometimes walked part of his way to school with him, as they'd to go the same way, and talked to him, leaving G. behind. If you've got to know the complexity of his thinking, you'll understand what he thought now, but no "ordinary" human being ever will: As O. was to be ignored by him (she had a boyfriend and that was fixed; his second selection was not to be changed) he thought she was sent out by G. to test him. He tried to be perfect, maintaining his real, up to this time hidden character at the same time. He and O. became friends; finally, really good friends, talking about half an hour per day and sometimes even more about their own lives or their experiences, while she talked more, of course; In fact, he was a quiet boy concerning talking; As a contrast, he read and wrote more than most other children do at that age. Books with over 500 pages were no secret to him, and a whole series of 20 books of that size was paradise to that child. That was the reason his love for language developed; and that was one of his faults, as books and music are arts that are capable of cracking one up. 
We could now go into detail, analyzing this friendship and the way it developed, taking into account her promises and stories, her faults and qualities; but we'll just have a look at some important actions and their outcome before we'll go over to "The Great Betrayal of Belief in Fate"; but that'll be done in other texts that are yet to be written. 
In this text here, I'll add something he'd also experienced this special Friday at noon time: Before he saw her, he entered the bus and had to choose a seat. All his other "friends" had abandoned him, and he was alone to find himself somewhere to sit. There was a place, separated from the others, and quite a similar seat behind it: He \emph{knew} them, he'd always sat on the seat that was behind, while she'd sat in front of him in happier days. Remembering one of these days, when he was killing himself by searching for something he could tell her, his mood changed; He was nearly depressive again, but quickly sat down on "her" seat, as his seat was already used; and as he liked to change things, wishing to feel how the place she'd once chosen was like. The remembrance crashed back into his mind again the moment he'd touched the seat with his back: A terrible pain ran down his spine while he was thinking of the day when he sat behind her and didn't find a thing to tell her. She was wearing sunglasses, perhaps not to protect her eyes from the sun, but from his view --- that was what he hoped for. Her sight was beautiful, breathtaking, but he didn't realize it that way at that time; he just watched her neck, her hair, the way she'd had it cut --- and the wonderful movements of her head when the bus shook while driving on the twisted street. Some part of him had realized he was in love with her, but he didn't understand it; but soon he'd cruelly find out the difference between being crazy about somebody (as with G.) and being in love. 
To be continued...
This week, by Wilkie Goldentongue...

--------------------------------------------------------------
(If you want to comment on this, please start a new thread; this one is reserved for this story, my life, the story that'll never end, while the end is still near, coming nearer every minute; it's just around the corner...)

The mustard seed was planted,
the gift has now been granted. 
Just now it is bound to grow,
what it is does future show.
W. G. 

If the shoe fits, it's probably your size.
\emph{(from a small fortune cookie program)}
