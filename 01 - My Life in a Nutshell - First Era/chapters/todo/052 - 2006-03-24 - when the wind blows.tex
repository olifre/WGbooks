Hello! Here we go again........

Several days have been gone, and another bunch of notes is waiting; however, I'm not feeling as eager to write today as I've felt the last day, but this is probably a result of me having enough time to do so today. However, I'm still planning; currently, I'm discussing with OliFre to rebuild the page so as to make it possible for me to write some other texts, as I feel that my life is more than reality; Yes, there should always be a foundation of fact, but also a fictional story can give you the wisdom to deal with reality, if it's based on something real; and everything man writes is. 
My life seems to be quite stable at the moment, and without any real happy event; thus, I'll try to continue reporting you what is happening, and probably do something more: A more happy, fictional story in another thread which may be more interesting than that thing we call reality, at least, when it's not told in real-time as this story here has to be. 
You can look forward to my adding some other topic, but it will probably take me some weeks, as OliFre has to rebuild the page so as to contain several stories and as I have to free some more time; probably I'll start at some time around Easter. 
For now, we shall have a look at all those things that had impressed me the last days. 

\textbf{Cold, warm, shocking and lovely at the same time:}
\textbf{[u]When the Wind Blows[/u]}
Three days had gone by, and many interesting things had happened; but before we start dealing with these in a chronological order \emph{(as far as that is possible)}, there is something that has to be added. 
He had seen O. in the newspaper some days ago, but just managed to forget about it when he started writing once more. When he'd searched the paper for faces he knew, there was one picture, showing her, him and several other students. The interesting thing about that was that, when the picture was taken, he had thought about exactly this situation and decided not to believe that it would ever come true; it simply was too unrealistic, and for that reason, it \emph{had} come true. The next important thing about him seeing that picture was the fact that he'd seen O. first, and then felt some shock before he realized that it was really O. he was seeing; and it was just some seconds later that he noticed himself standing beside her. 
O.'s best friend and some of his friends were there, too; it took him some time to get back to normal to be able to identify his fellow students. 
But that was something quite normal in comparison to the thing that had happened today; not only had he heard her name on the radio \emph{(a person sharing that name with her was greeting somebody because of a brithday)}, but the special 'coincidence' of this day could \textbf{not} be regarded as \emph{random} anymore. This very day, an orange car stood in front of his house; this was usual, though cars were rarely parked just there. His granny would try to read the text on the door just some seconds after he'd noticed it, and he could hear her voice from below; but before she started, she repeated one word \emph{exactly} three times: ORANGE, ORANGE, ORANGE, accentuating these words in a way so they \emph{should} sound funny. 
For him, they didn't, as he'd already noticed that something was to happen, though he hadn't yet made the connection to O. 
Then, his granny would try to read out the text, but she couldn't see as well as he could anymore; thus, she read it wrongly the first time just to correct it later on. 
The checksum of the most important word printed on a sign on a door of the small truck --- as it was a truck from a firm --- was five, which was equivalent to the checksums of O.'s name and the name of her boyfriend. 
Then, the thing happened which made him feel another shock: His granny asked her partner whether he'd already taken the \emph{sunflowers} out of the car. Sunflowers --- O.'s favourite flowers, it dawned on him. 
This was no coincidence, this was a sign. 
The car of the person that worked just on the other side of the road was parked on this side now, and the plate showed O.'s birthdate, but this was something he'd come to ignore, having realized that it could be the birthdate of that man, too. 
This should have made him sensitive for such signs, but he hadn't realized the meaning up to now. That car of that man was parked exactly at the place where O.'s car had stood once, for a minute or so; he should have figured that out. 
But he hadn't, as his system was working well. 
But what did all this mean? Was O. gone, or would she return? What had happened, what was to come?
These questions were arousing again, and he couldn't help but think about them this time, at least, when it came to the meaning of the signs. His answer was nothing clear, and probably, this was the way it should be; however, it still showed him that there was \emph{something} or \emph{somebody} out there. 
He was surprised when he noticed just some minutes ago why he hadn't been eager to start writing that day: His mind was full of swirling thoughts, and he began to feel turned upside down. The swirling wouldn't stop, but writing down everything would help, and he was just regaining confidence. 
Thus, we may now begin with the chronological order: Wednesday. This was a long day, full of work and without any time left for himself. He'd meet P. and her friend(s) again, and it was probably that day when he noticed something pretty interesting: Sometimes, when P.'s friend stood behind him, looking what he was doing, she'd breathe in such a way that her breath could be sensed by him exactly behind his right ear. He felt a strong friendship, and he wasn't sure about the exact meaning of this; something else P.'s friend had just introduced some days ago was some special way of saying goodbye with the right hand: She used this technique with P. and him, and he liked to do so. However, L.-b. seemed to ignore that, though he would have liked her to notice it so as to see that he was \emph{not} interested in her. Probably, she did, and it was just her jealousy which made her show even more interest. 
But before we lose ourselves again in such discussions, I shall list the recent events, as time's now running out again; In the beginning, I thought I'd be through by now, and there was plenty of time; Now, it's all gone. 
Wednesday, he'd seen some money lying on the floor, but forgot to take it after the lecture was finished; however, he talked with somebody about it \emph{(one of his friends)} which he wouldn't have done a year before. This was something that would point out his change of character: Suddenly, he was open-minded and in the middle of society. 
Something else was quite special: He'd had a reason to phone B., and he'd done so, while she seemed to talk quite fast when answering him; but she was offering him her help, in a way few would have done. He felt that her talking quickly was no sign of nervosity, but of a lack of time; all sensible people lacked time. However, he felt that O. didn't, which made her even more interesting...........
The next day, she seemed not as interested as she should have been in the outcome of that call; he'd had to ask her, but he did so after only seconds of wating for the right moment, and he repeated his question two times, adding her name the second time so she felt that she was meant. 
And this process --- adding the name of the person he addressed --- was something he'd rarely done, though he knew that it would please people. Now, he was even feeling that it was normal to say 'Hello!' and 'Goodbye!' to people he didn't know \emph{that} well. 
These were the most important things that had taken place; everything else seemed normal, as far as that is possible. But there's something we should add: B. seemed a bit rude when he asked her whether she'd received his message, but he had understood, though he was pretty sure he'd have reacted differently; but this was the prerogative of humanity: To be different, to be individual. 
He felt he was doing the right thing, while he understood that the special sign of the orange car of that day was probably just a summary, showing that he'd understood O. wrongly the first time, like his granny had read the words the wrong way, while others could have seen easily. He wasn't sure about all that, and it's meaning was not revealed to him --- not now. Probably, it should not. 
But there's something else we've forgotten: His friends had played something quite silly on him, just assuming he was a bit older and married with P.'s friend, having a child. Once, they'd even done so when P. was there, and he'd told her the story, leaving out that he was meant to be married to her friend, while he didn't know why he hadn't told her. Probably, this was a question of denial or not denial. 
He was left with a swirling mind and felt dizzy; he was tired to the death, and needed some sleep. 
Still, he hopes he'll be able to continue soon and see you again, right here.......
And your opinions, of course..........
To be continued...
This week, by Wilkie Goldentongue...

--------------------------------------------------------------
(If you want to comment on this, please start a new thread; this one is reserved for this story, my life, the story that'll never end, while the end is still near, coming nearer every minute; it's just around the corner...)

The roof 
of the building on the other side of the street 
was under repair; 
the building 
where that person worked 
which was linked to O. 
by that number 
on the plate. 
The roof --- 
protection from nature, 
protection from one's own nature? 
Was O. denying her own character? 
Was that the reason of her changing? 
He wouldn't know; 
he wouldn't like to know 
whether a shell may be transformed 
into something not to protect 
but to kill the things inside, 
slowly, cruelly, finally. 
\emph{W.G.}

Extermination --- 
a word full of cruelty. 
And nevertheless, 
it could mean happiness, 
when the situation was different. 
The problem of man is: 
What is the situation like? 
\emph{W.G.}
