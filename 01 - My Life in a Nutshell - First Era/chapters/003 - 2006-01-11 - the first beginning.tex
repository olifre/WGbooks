\chapter{The first Beginning}
\label{cha:the-first-beginning}
\subsection*{Originally published: \DTMdate{2006-01-11}}
\begin{quote}
Hi, again!

I didn't have a lot of time to write the last day; thus, I'll now try to write a bit more; the story is long enough, but time isn't. 
But one thing, I just want to tell you before I start to go on; in that text, I think, the persons I'm talking about will be able to recognize themselves, and as that is so, I want to tell them one thing if they happen to stumble into this place \emph{(which is quite improbable)}: Please don't talk about this thing, and don't think about it (if possible), until it's finished. This may take some weeks or even months, as I won't be able to write as much as I'm writing now every day; currently, I pretend to be on holiday and there's enough time to try rescuing my soul and my life by writing this. 
\end{quote}

Today was the day he'd looked forward to: Everything seemed to be right, the signs of an upcoming meeting or a happy event together with \textbf{HER} seemed inevitable; but as hopes are never fulfilled, he would just see her twice, being able to smile at her and watch how she smiled back at him. Once, she even embraced a guy he didn't know, but now, it seemed as if he didn't care. He had two reasons for that: First, yelling at the world that he loved her (even if he only did that by writing into a forum none of his friends knew) made things easier. A lot easier. He asked himself if he had stopped being in love, and he couldn't escape quite a negative answer, as she'd ignored him for quite a long time, even if one overinterpreted every single action she took. Secondly, she had at least one homosexual friend and embraced many persons; yes, she'd also embraced him, \emph{ONCE.} \emph{(which didn't mean he was homosexual, as you could figure out on your own while reading that story, but gave him the terrible feeling that she thought so, and he had no power to change that without having to hurt her feelings and destroy their friendship, if there was such a thing existing in the cruelty of love)}. But that was a long time ago; well, not so long, just about a quarter of a year, but things had changed till then. She'd told him once that hope dies last, but who didn't know that saying? He liked it, but he knew that sometimes a hope that died is better than a hope that's alive, because hope and pain are feelings of quite the same kind; at least, finally, when things had sorted out. Well, things won't ever sort out, he knew that; the entropy of the universe was never to be reduced, maybe excluding the moment of the final outburst of energy to destruct everything; that was the way he felt, and he knew that he had to act, otherwise quite a similar outburst of thought would crush his life and render him a dead person or even worse: A person without soul.

But we'll now go back to see how the first blow on his already shaken soul started to develop and finally crippled it; and the inevitable chaos that started to be hidden beneath his forehead. 
It was the day of a kind of feast, and we're not really interested in the reason for that party; we'll just know that he was there; and that \textbf{THEY} were there, too. He'd gotten to know them, several short encounters led him to be able to differentiate their faces in the masses, and he could recognize the voices; he still encountered a special \emph{feeling} that he couldn't localize; not knowing where this powerful emotion came from, he thought he simply loved \textbf{THEIR} \emph{character}, their happy, \enquote{easy goin'} way of living. They cared for everybody, and seemed to be the best friends, though they were completely different. He wanted to know if there was something below that \enquote{shell}, below that profile they presented; he thought of his own profiling system, and decided to try to reduce it to reality while he was with them. Yet, there remained the \enquote{DAY OF SELECTION}; he had to decide into which of them to invest his feelings. And that \textbf{DAY} is the day we're currently looking at, the day that destroyed his life and'll perhaps render it impossible to go on.

But before we go into detail, we've got to add something, some part of him we've really paid not enough attention to until now: After reading his story, one'll notice that he sucked in and saved every detail he encountered, and sometimes even the exact words he was told or told himself; his eyes and his ears, all of his senses, didn't miss a thing concerning the creatures he loved more than himself. But as nobody noticed that special ability of his, he kept it for himself and thought that the fact that he normally couldn't remember a word of a conversation with a \enquote{normal} person --- as long as it didn't come to \textbf{them} --- was the main proof that he was in love, hopeless in love. And he seemed to be right, as all tests he knew were positive: Thinking of the \textbf{person} in question with closed eyes without intending to do that consciously; waking up at night or even being unable to sleep because of steady thinking about \enquote{the one}. But the \enquote{selection} was one of his biggest faults, if not the biggest he'd ever make: Finally, he gained two friends that you could name \enquote{real friends}, taking into account that he was a boy and they were girls; but he lost his chance to love and be loved by the single person on earth that was related to him by soul, a chance a human is rarely offered. The worse was the feeling that overcame him, haunting in his dreams at night, when he'd realized it was too late\ldots\ldots

But we'll come to that later. First, there was the \enquote{selection} itself, and the sense and simplicity of it destroyed every future for emotions, while --- at the same time --- it maybe was the only way to make emotions possible for him \emph{at all}.

As one may imagine, the small boy without any experiences simply had a look at both of \textbf{them} \emph{(You'll notice that I'm accentuating \textbf{them} less as time moves on; that's a result of the presently changing situation)}. But as he wasn't an ordinary boy, he didn't use an ordinary look; he watched for quite a long time and decided who of THEM was Sense and who was Sensibility. He soon noticed that the one --- we'll call her G. for the moment --- seemed to be obviously sensitive and sparkled with emotion, while the other one --- we'll call her O. for the moment --- seemed to be thinking logically and lost in her thoughts. Though he preferred the more sensible, but sensitive type at first, he decided he wouldn't ever receive a chance to be the friend of either of them; for that reason, he decided to select the less obvious one and stop thinking about it for the moment. That decision lasted quite a long time, but it was broken the moment he really saw their faces and drank in every detail he could get.
That was the time when he understood what the following lines meant:
\begin{verse}  
  And when I saw her face,\\
  I was a believer\ldots\ldots\\
  (refrain from a now quite old English song, which is still famous, but nobody I know listens to it, really)
\end{verse}
You must know something special about him: You told him a name, and he could tell you the special colour of that person. That was another one of his special gifts that would only make his experiences worse; he could make out everyone he knew by just looking at something he / she owned, at least concerning the persons he really liked. It was the same for him with smelling their scents or hearing their voices; He wouldn't fail to recognize somebody even if a mass of people was around him, and if he really wanted to find out about one of his fellow creatures, he was even more sensitive. All of us sense a lot of signs and feelings we do not fully understand, but people who listen for these hidden treasures are rarely to be found. 
He was one of them, soon investing all the rest of feelings he had in his dreams of a better future. But his selection wouldn't be worth a penny as he'd find out later, when he received the message that one of the two persons in question (O. the one he'd selected) had just found a new boyfriend, and this relationship seemed to be really fixed. As his decision wasn't really fixed at all, he decided to stow away her record in his brain, encrypted and far away, out of reach in one of the corners of his brain. THAT, then, was his real fault and proved to be his \emph{sentence}.

Now, the investment of feelings began; subconsciously, he knew something had ringed a bell when he saw O. first, but he didn't notice that until it was too late. The signs that begged him to open up the secret feelings he'd packed away were ignored; everything he tried to forget or to ignore was ignored properly, completely and impossible to be brought back into focus again. Up to then, as he finally changed, too, when the secret door was broken by the cruel mass of ignored feelings, just when it was too late\ldots\ldots

As a consequence of his new decision to worship G., he did so. They became friends, though he overinterpreted everything about her. He survived a period of soul-breaking, depressive feelings, so strong he believed --- he knew --- he would die soon if he didn't receive any sign from her. Well, he received signs: Signs of friendship, and signs that she was in love though she didn't have a boyfriend, but she was in love with somebody else, and he'd be the only person to find out about that, as the male person she loved was nearly hated by O., and nobody ever put the pieces of information she gave passively --- and once even actively --- together. He did, and he ignored the results, something he was doing quite perfectly. He received a kind of positive reaction, when \emph{fate} allowed him to talk to her several times, and when she once was completely alone with her; but he wasn't able to say a word, and she didn't, either. The valuable seconds passed, and perhaps it was lucky that it went this way, but he didn't feel like that at this time. The only thing he felt was a pressure he was putting on himself: the pressure to say something, just \textbf{A WORD} to escape this vicious circle, but the more pressure there was, the harder he tried, the less he was able to do anything about it. He managed to end his depressive attitude, and he finally was happy again; but that was a result of a new friendship he started with O., but as he didn't ever think of loving her, he continued to invest all his feelings in G., though he only received weak signals of friendship. 

\begin{verse}
  Love hurts\ldots\ldots\\
  Some fools favour happiness; blissfulness; togetherness;\\
  some fools fool themselves I guess\\
  but they're not fooling me.\\
  \emph{(extract from a famous English song that's now regarded as quite old)}
\end{verse}

\begin{verse}
  Stop searching forever, happiness is just next to you.\\
  \emph{(from a small fortune cookie program)}
\end{verse}

%%% Local Variables:
%%% mode: latex
%%% LaTeX-command: "latex -shell-escape -synctex=1 --file-line-error-style"
%%% ispell-local-dictionary: "british-ize-w_accents"
%%% TeX-master: "../My_Life_in_a_Nutshell_-_First_Era.tex"
%%% End:
