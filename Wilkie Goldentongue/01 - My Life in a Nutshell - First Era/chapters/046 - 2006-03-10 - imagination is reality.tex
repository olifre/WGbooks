\chapter{Imagination is Reality}
\label{cha:imagination-is-reality}
\subsection*{Originally published: \DTMDate{2006-03-10}}
\begin{quote}
A long time has passed\ldots But here we go again!

So many things have happened, and all those people he didn't really know about before grew important now; P. had the greatest problems in her life, O. was still hidden and B. had been offended by him. We shall have a look at all those tiny details\ldots

He'd got to know that: Imagination is Reality.
\end{quote}

As we've already figured out, when time runs short, there are always more things to write about. This theory was proved once more\ldots
We've just stopped Tuesday evening; I can't promise you that everything I'll tell you know has happened in exactly that chronological order, but I'll try to keep close to it. 
Be that as it may; we'll begin with Wednesday evening, as the time at university was the same as every day: Meeting P., her friend(s) and all the others, talking to them\ldots

P.'s friend once wanted to take his seat and pushed him away; the contact they were having that way impressed him, as he was very sensible to any kind of sensation. Even if that was just a game to her, he felt it was important for the development of himself, as he'd never experienced such things.

P. herself was very friendly to him --- again. She would even ask him if he wanted to take a seat, and he felt that she would offer him something she wanted for herself. This was one of the most important proves there are for a good friendship, and a nice basis for the future, which could consist of an everlasting, strong friendship --- or more. The very next day, she would be very troubled; but we want to advance chronologically, as far as that is possible for me.

Wednesday evening was the next important stop. Then, he met P. and some other friends; not at university, but in their spare time. P. wasn't sure whether it would be possible for her to go there, but she succeeded. There was nor real opportunity for the two of them to talk to each other, but they were happy nevertheless. When she decided to leave, she told him they'd see each other tomorrow.

He was a silent person in this group, that evening; he was always quiet. The only people he could talk to easily were his best friends; however, his profiling system was doing pretty well, so that all the people around him didn't take any interest in him when he didn't want them to do so --- on the other hand, when he felt lonely, at least one person would start talking to him out of the blue. And he could cope with nearly every topic there was, as he was interested in all the world out there.

He wondered whether he should ask Y. if she'd take part in his idea to teach her his methods of learning, and the basis of intelligence; it was nothing but mehods, and a different perspective on the world. Everybody could be like him, and he could be like everybody\ldots

He was too idealistic sometimes. He wondered whether Y. would accept, and what the others would think about this idea of his. But he was still waiting for the right moment, which would probably never arrive.
But that was not important now, though life was a dream, finally.

His imagination was stronger than the sensations his senses could give him; thus, his dreams were stronger than his reality. A lack of sleep in the last days had made those tiny details very apparent to him, and the influence he'd have on others became  very strong now. The other students were suddenly tired, and some of them had slept long, while others had slept even shorter than he had; they were all unified in a general tiredness, and when he would wake up and become interested, they would do the same. He wondered whether his dreams were reality and the reality just a dream, as it seemed to be a bit out of focus. Was life just a simulation of his brain? This was --- once more --- one of the questions nobody wanted to answer. One could not prove it wrong, and one would not like to prove it to be false; this was another question you'd better not try to answer.

But we want to continue with all those things that had took hold of him.

Before I go on, I must say \enquote{Sorry!}, as I've just remembered something important that had happened Wednesday morning: He had shown something B. had shown him once to somebody else, and now, she'd blocked that thing, without saying a word about it. He'd tried to contact her digitally, but she hadn't answered. For some time, he'd wished he hadn't done all that, but she had not told him not to show anybody, though he'd somehow \emph{felt} it. And she'd once told him, that she wanted to block it at some time\ldots

Because of him talking about it with her. He thought that she was joking, but now, he wasn't sure. However, if she'd really \emph{seen} him showing that to somebody else --- some boy who seemed like the agressive-excentric-outsider, which means, he was normal, but finally contained some of the black understanding below this crust of shallowness --- she must have watched very carefully.

Sometimes, he thought he was overestimating her; sometimes, he felt the opposite. He hadn't turned around at that time, when he showed that thing to the other boy, as he'd feared her turning her head though he was not talking very loudly. Indeed, he noticed the next day, that she was remembering details, and probably assumed that he had already seen that it was blocked. She was showing the interest of a friend, of a good friend, that is.

On the other hand, she wasn't. Her best friend was more easy to understand; he knew her pretty well, and it was not that complicated to deal with her. She was also very friendly, and sometimes even nervous when asking him something, though this was probably a result of her not wanting to be thought of as somebody stupid.

Well, she wasn't; nobody was, but everybody was different, needing another type of explanation. L.-B., for example, was needing loads of explanations, and they should be formed in a way she could learn by heart systematically. It would not be easy to explain her something in a way that made her understand why things were like they were --- she'd rather learn a system than do so, but nevertheless, she was successful. He hated such a kind of doing without understanding, and couldn't see the point of it. However, he knew it would be very hard to teach her his methods, as she would not accept them. Nevertheless, she was still following him all around, asking him in all detail what he was doing after having asked something she had probably known before. He wondered how her reaction would be if he suddenly told her that sentence with three words; \enquote{I love you.} Well, he didn't love her; it was just that he was offending her so openly that she should notice that. And he was wondering how she would react when her apparently desperate longing would be fulfilled.

He liked to think about people's reactions, but he wasn't able to come close to reality; he was not experienced enough.

That Thursday, he was having closer contact with B., talking to her and starting a kind of friendship. It was complicated to please her, and she was a complex personality; he liked that, but she had some features he would not accept so easily. Things were always difficult, but he couldn't imagine her being his girlfriend, as they had some things in common that normally only one partner is doing. For instance, she liked to take photos; he liked that, too, but his time was too limited to do so. She also liked reading; I guess you should know by now that he did, too. However, she was sometimes saying things without having thought about them, and was thus not always capable of supporting her own opinions.

He was always trying to choose his words carefully, but he did not succeed very often.

The best example has just happened on Thursday afternoon: P. was leaving university, just to come back some hours later; she was always careful not to miss a lecture, if that was possible. He was the same, but I guess you've already figured that out. While she was playing with his arm in a friendly way, jumping and telling him how disgusting she thought it was that she had to go now, he tried to reflect her feelings; but he was quiet, and he couldn't do so perfectly. In addition to that, he rarely touched anybody, even by means of just making someone notice that he had a question. P. was different, apparently, but he liked such contrasts which gave him the possibility to learn; P.'s friend was equally open-minded.

But we're not to stop at the point when P. left. She returned, of course. And then, she was broken. She entered the auditorium when the lecture had just been finished, talking with the professor about what had happened; he felt he couldn't stay, as this would induce suspicious looks from his friends. Thus, he went outside to drink some tea from his flask, but she seemed to stay a little longer, so he went away. He met P.'s friend, and when he was talking to her, he suddenly saw P. standing some metre away, in a way she couldn't see the two of them, and busy with talking to another friend of her's. The two of them went up to her, and listened; he didn't realize the reality of the shock of that message to her, as it would not have been a shock for him. Y. was suddenly passing by, asking him for some help in his spare time; he accepted, of course, and thus, it was impossible for him to concentrate on P.'s problem with all his mind. He saw that she was sad, but this was to change soon. When she had finished talking and all the others had left with the exception of him, the two of them went down together; he asked her for some more information, and then\ldots

She was suddenly not sad anymore. She was so close to tears that one could say she was crying; he was stupefied, and incapable of doing anything. He was just listening\ldots

When others arrived, friends of her's (girls), these asked her what had happened, and she pointed at him, gesticulating that he should explain. He told the few things he had heard in the seconds they had been alone, and he heard how wrong he'd chosen his words, presenting this shock as something too normal. But words can't be called back\ldots

Finally, she was explaining this herself, and one of the girls embraced her, while he could just sit, stare and think. Stupefied. Helpless. Full of emotion, but looking as solid as ice. 
When the next lecture began, he didn't tell her, though he was pretty sure that she'd like to be there in time; on the other hand, he felt that this here was more important, and he didn't like that strong girl walking up to the others with tears on her face.

Need for protection --- one of the key features of female attractiveness.

When the others were finally gone, being late for their lectures, too, she noticed that the lectue had begun, and felt guilty, even though her life had just changed radically; he'd slowly understood how important that was to her. He told her what he had thought, and she was finally gone, and he could not recall having heard a word of goodbye\ldots

The next day, she'd arrive soem time later; she greeted him, and they were talking a bit, though he felt she had changed or was caught in a process of change. When one of his friends walked up to him, he told him better not to talk to her, but he did; she was indeed mastering her emotions, as he did not seem to notice a thing. Then, when that boy was gone, L.-B. arrived, and it was difficult to send her away without inducing hr to stay even longer. She didn't understand any sign\ldots

When he was finally alone with P. again, she told him some more things, and excused herself for having been rude, though he could not recall her having been like that. He told her he understood perfectly, which was true, as he'd imagined an equally awful situation yesterday, as far as that was possible. Finally, she was also talking to that boy whom he had showed that thing from B.; he understood her, and there was suddenly a conversation developing. However, it was different to the discussions he had with her, as these would be full of things she would not tell everybody. 
I think we should stop here and leave the rest of Friday for the next text; and some more details about the other days, of course. You'll probably be astonished when we start dealing with the time when he would be on his way home on Friday\ldots
Please share your opinion with me\ldots

\begin{quote}
Pain \\
is evil. \\
Pain \\
is wonderful. \\
Everthing is ambiguous, \\
and the decision about it's meaning \\
happens by chance; \\
we cannot decide about reality, \\
we \textbf{are} reality. \\
--- W.G.
\end{quote}

\begin{quote}
Friendship \\
stabilizes a person without a relationship; \\
however, it is the basis of all relationship. \\
What makes the difference? \\
Chance. \\
Nothing more, \\
and nothing less. \\
--- W.G.
\end{quote}

%%% Local Variables:
%%% mode: latex
%%% LaTeX-command: "latex -shell-escape -synctex=1 --file-line-error-style"
%%% ispell-local-dictionary: "british-ize-w_accents"
%%% TeX-master: "../My_Life_in_a_Nutshell_-_First_Era.tex"
%%% End:
