\chapter{Flashbacks}
\label{cha:flashbacks}
\subsection*{Originally published: \DTMDate{2006-01-29}}
\begin{quote}
Hello again!

I know that this post is pretty short; but the weekend had been a time of hard work, and I had to do a whole bunch of programming as I'm still attending university. 
You may notice that I contacted the admins of this forum, and that there's now a special place where you can read these posts. In addition to that, they are also available as a newscast. 
Thanks to you, sw137 and OliFre, and don't be offended because I've withdrawn into silence again; I'm still thinking too much, and I want to share my thoughts with you. 
However, there are not many things that have happened this weekend; but there were some flashbacks I had to endure, and I don't want to keep these away from you.

Here we are again: Flashbacks.
\end{quote}

Weekend; time to think. 
A lot of time.

He was happy he had a lot of work to do, so as not to think the complete time, but there were still some memories chasing him. Memories of the past, and memories that had been stored in his brain quite recently.\\
\textbf{FLASHBACK}\\
Thursday; G. was doing some work, and she had a question. Well, he hadn't understood what she'd asked, as he was sitting at some distance; but he was pretty sure he could help her. 
But first, she asked O.

And O. asked her best friend, and in addtion to this, one of his own friends. 
O. must have known that he knew the answer. 

She was to blame for not asking him, as the other persons probably hadn't noticed his presence, but she had. Thus, G. was left without any help. And he didn't want to walk up to her and offer his knowledge. She could have asked, but she didn't; and O. was to blame for that. 

When G. was working alone again, he really planned to go there and help her; but finally, he remembered something else\ldots\\
\textbf{FLASHBACK}\\
About one year ago; G. was working at something she had to do for the same professor, as he'd recognized the book. And she was finished, when she asked another boy for his advise to have a look if everything was right.

Our protagonist had already figured out that there was some special realtionship between G. and that boy; she seemed to be in love with him, though there was no reason for this, as that boy was neither intelligent nor really loyal. He was self-absorbed, a boaster and the ideal example of a windbag, indeed; You rarely find somebody who reflects these ideals in such manner.

But she seemed to love him, or at least was crazy about him, as O. was sitting next to her, probably knowing better about these things than he knew; But she asked him first. And finally, he asked O. 
Our protagonist wasn't asked, though he would probably have known about that stuff better than all of them.

At that time, he thought about walking up to her again, but he decided against that idea; He didn't want to give the impression of a show-off. 
If nobody asked him, they would perfectly be able to succeed on their own.

They knew what they missed. 
And they did it on purpose. 
Again, O. could have called him, but she didn't.\\
\textbf{FLASHBACK}\\
He was sitting there, in a room, working together at some hard work with two friends of O. 
She was sitting some metre away, training for an exam with some of her friends; about one minute before she would ask for his help, he took out his book and searched for the right page, telling the other girls that worked with him, that O. would call him --- soon.

And she did. 
He told her to wait until he'd found the right page, and finally, he explained all of it to her, probably too fast, but he was nervous. 
Very nervous.

He'd also drawn her something, explaining all of it; and then, he was gone again, back to work. 
He was pretty sure that she still was having problems, but she pretended to be off quite perfect. 
And he pretended to believe it.\\
\textbf{FLASHBACK}\\
He was sitting in front of some computer, browsing through the world wide web. 
She was in an exam, that would be quite hard for her. 
Our protagonist had been acquainted with some methods of telepathy and several other esoteric things; Even if he didn't believe in these completely, he could try them without any risk.

He did. 
She wrote the best mark ever, and the boy that G. was in love with --- the same boy O. hated --- wrote one of his worst, as he even forgot to give the exam to the professor. 
He was lucky that she finally accepted it, though he'd kept it with him for a complete weekend. 
But that didn't change his mark for the better.\\
\textbf{FLASHBACK}\\
The same exam had been returned to O., and when he passed by the auditorium she was sitting in (the professor hadn't arrived yet), she stood up and ran to tell him happily about the mark she'd received. 
He returned that happiness.

At that time, she didn't have no boyfriend, but he didn't know about that. 
Now, she wouldn't even turn her head to tell him one of her marks, even if he asked her about it. 
Sometimes, she just told him something like \enquote{bad} or \enquote{good}, nothing more. \\
\textbf{FLASHBACK}\\
Quite recently, she'd told him her computer was not working anymore; She knew he could repair it, but before he could offer his help, she told him some friend would have a look at it. 
He was quiet again.

And that Friday, when he waited for the bus 50 minutes, he thought about the craziness of having joined her on her way home, there repairing her computer\ldots
CUTOFF!

Well, he could have waited there for the bus, too\ldots
CUTOFF!

And it wouldn't have been so cold\ldots
CUTOFF!\\
\textbf{FLASHBACK}\\
She was telling him something about her pet, and finally told him she would probably show it to him once upon a time. 
He noticed she would have to invite him to do so\ldots
CUTOFF!\\
\textbf{FLASHBACK}\\
She had told him once, that she would hang some picture of him that had been in the papers in her room, at home. 
And in his attempt to contact her, he'd asked her if she did. 
But she hadn't answered, and maybe, she hadn't even read it\ldots\\
\textbf{END OF FLASHBACKS}\\
There was still some fire inside him, something that still was burning, that had not been extinguished; the fire of his memory, that there could have been something that would never be. 
But maybe, he would keep it, as our memories are giving us our character --- and our soul. 
Sorrow is always around us, whether we are in love or alone. 
Most times, we are full of sorrow\ldots
Please give me your opinions about these texts; I'm longing for some of your experiences, maybe they could help me to cope with mine\ldots

\begin{verse}
A fire that has once burned, \\
will always be burning, \\
in the memory, \\
of our soul; \\
as the fire of emotion, \\
will never die. \\
--- W.G.
\end{verse}

\begin{verse}
Open your hand \\
and grab the world around you; \\
In everything you've grabbed \\
there will be love, \\
sorrow --- and pain. \\
--- W.G.
\end{verse}

%%% Local Variables:
%%% mode: latex
%%% LaTeX-command: "latex -shell-escape -synctex=1 --file-line-error-style"
%%% ispell-local-dictionary: "british-ize-w_accents"
%%% TeX-master: "../My_Life_in_a_Nutshell_-_First_Era.tex"
%%% End:
