\chapter{Sadness of Hope}
\label{cha:sadness-of-hope}
\subsection*{Originally published: \DTMDate{2006-03-27}}
\begin{quote}
Hello! This is Wilkie Goldentongue speaking, once more!

Special things have happened, and time was limited once more. Now, I'm just going to try to tell you the most important facts about all those events that have just passed by. I hope that you're still with me\ldots

He was realizing that the only possible future was the: Sadness of Hope.
\end{quote}

The weekend had just passed, and only some small special details had caught his attention, as there was a big amount of work to do, and he was not to be through it for long.

For that reason, he couldn't remember all the details there were; however, he could still recall the moment when he saw P.'s name on one page of his calendar, as it was her name day. Of course, as a protestant, this would be of no importance to her; but he was shocked for some seconds, as he didn't make the connection between the name on that piece of paper and it's meaning. This was interesting; in fact, he just had a look at those pages at the \emph{end} of each day, and he was quite astonished about the new reaction he'd feel when reading her name; warm feelings of friendship had developed, and though he knew he'd never show them in the not-so-subtle way everyone used, he knew that they were there --- and that they were important.

The next day, another name appeared; this one was linked to a book they were reading in a lecture he shared with P., and in fact, it was the name of the protagonist. 
Life was always interesting; and he knew that it was pretty astonishing, too, what he'd figured out that day.

P.'s friend had developed some special kind of saying goodbye with her hand, and soon P. wasn't using it anymore as she'd felt that the hand of her friend was quite sweaty; he, however, would continue to do it, though he tol P. that he shared her opinion, but had already got used to it. But something else was quite interesting; something that had just ocured to him; maybe it was just a result of the \enquote{P} being located quite close to the \enquote{O} or just some sort of coincidence; but it had already happened at least two times that he'd typed O. rather than P. But as he typed so fast that he could rarley see what his fingers were doing, he wondered whether this was \emph{really} a coincidence; however, he didn't feel any love concerning P., nor did he with O. anymore, at least not on a conscious basis. Today, he'd sat next to O. again, though just one row behind him, there was a seat next to P., one of the single places that was not taken by anybody; probably, it \emph{was} the last seat which had not been taken by anybody, and he just sat in front of it, wondering whether P. would like him to sit down there.

But O. was changed that day; she seemed overturned completely, and her best friend wouldn't sit next to her, thich led our protagonist to assume that something had happened, something of importance. 
\begin{quote}
Wünsche fliegen, Pläne gehen, Taten hinken.\\
--- A Turkish saying
\end{quote}
You'll understand why this one is here when you've just continued reading. And this one, two: 
\begin{quote}
Hast du Verstand und Herz, zeige nur eines von beiden.\\
--- Friedrich Hölderlin
\end{quote}
I know that these quotations are German, but I happen to like them, and this is the single language I could find them in. So, now there's one for the other Englishmen out there:
\begin{quote}
Appearances are often deceiving.\\
--- from a small fortune cookie program
\end{quote}
So, why are those here? Go on reading to find out\ldots

We already know that he sat next to O. this Monday once more, and that P. was sitting behind her, so that she could see the two of them. And we know that O. had changed\ldots

But in which way? She was silent. The usual happiness and the vivid manner was gone, suddenly. There was nothing of her that would remain, only her beauty; and when he looked at her once more, he felt he shouldn't do so, as this would endanger him once more. Now, he knew what was so wrong with him: Sense and Sensibility should \emph{not} be shown in company, as we've learned above in that German quotation. He had done, and now, both seemed broken and were in need of repair; thus, he'd jump at anything there was out there.

So he did, and he felt that only reason held him back --- and his bashfulness. The warm feeling in his stomach, when he was talking to P., and all the others standing together in other groups he'd never really belonged too; but that was the thing with boys: Certain groups would share their interests, and somehow, they'd stay together; once and then, this would change, but some people were simply too different to take part. If they did, there would be very string friendships, and they'd break down after the time of one or two years, just a normal friendship remaining. This was the very same thing that could happen with love affairs, and he knew that this was the Sadness of Hope. He wondered whether O.'s appearance was really deceiving; then, when some things happened which should have made her laughing, she didn't; she just sat there, quietly, as if she hadn't noticed. The break would reveal some more, as her best friend would then walk up to her; O. would want to tell her something secret, but he seemed to be among the persons that were trusted, while the two of them tried to talk without B.-B. noticing it; and O. was even too sad to go out of the room to tell her best friend about it. Thus, he could catch up some phrases, just making out that she'd have to visit the doctor in the afternoon. However, she wouldn't like to do so; something forced her to. The only thing, she said, that was happy about it, was the fact that she'd miss the second part of one lecture; this was something he'd never have believed her to have said some months ago, but life was apparently changing all the time.

She seemed to be, too. Her mood, which was completely full of sadness, would now change a bit; sometimes, there would be that happy smile he loved so much. Before, he'd promised himself that he'd ask her about it when the break arrived, but when he saw her best friend heading for her, and P. looking at him, he couldn't do so anymore. He thought that this would probably be a sign of her having ended her relationship with her boyfriend, but after he'd heard that thing with the doctor, he'd rather believe that she was pregnant, though this was completely impossible.

She'd never quit university because of a pregnancy, and she wouldn't risk it. But she was unhappy about that visit she'd had to do in the late afternoon, and he couldn't do a thing but trying to cheer her up. He succeeded in doing so, without even noticing he was doing it; the two of them would simply fit together, probably, but he didn't know if that would really be a good foundation for a long relationship. However, it could be a basis\ldots

He'd made her laughing once more, and it was easier in the second part of that lecture, as she seemed to have lost that eternal sadness; at least, some of it.

She'd now be smiling once more, and he was happy to see that. One thought had occured to him: If her boyfriend had left him, and if she was pregnant and willing to bear all these problems; would he still accept her?

Yes; this was the answer. But he felt that this would probably a problem with his parents; and though these were very important to him, he felt that she would be more important, if things were really going this way.
But he was sure they weren't.

In the next two short breaks, he'd do something for one of P.'s friends, and he'd help her gladly; once more, he felt that he was everybody's friend and was still alone. 
\begin{quote}
If music be the food of love, play on.\\
--- William Shakespeare
\end{quote}
Music; he'd once again started to use this impressive storage of feelings. More than a year had gone by since the days when he'd listened to all of the songs he had, and he knew that listening to them once more would probably give him a hard time; in addition to that, it had been one of the \enquote{triggers} --- sort of --- that had been there before there was the \emph{real} friendshp with O., and though he'd buried the hope since long, the sadness of these songs could probably do him some good, as they'd be the recompense for the happiness that would probably arrive.

But he knew that the system of waves he'd found out about couldn't be betrayed, and thus, his time to listen to the music and the ability to interpret it was suddenly limited. 
Nevertheless, he'd try to do so. 
\begin{quote}
Wer den Kopf nicht hebt, kann die Sterne nicht sehen.\\
--- An Armenian saying
\end{quote}
He knew he had to do something to change the world, and he hoped that this music could help him to do so.

But there are still some things we've forgotten about Friday; a girl he'd often talked to in the bus --- he'd often gone on her nerves, in fact, but she seemed to like that childish game --- was suddenly making contact again. This was the girl that would have liked to disguise him some weeks ago; probably, you may remember her. She was another black person, and she was something special, too; the sister of one of her friends he'd also known from the bus --- the one his parents had assumed him to have a relationship with in that silly game they liked to play --- was now often sitting next to him in the morning, while he'd lost contact to the \enquote{Evil one with the Glasses} for the last days. That small girl was funny, but nothing more; and that girl with the colour of black around her was interesting, but she had a boyfriend, and she was not the girl he'd have liked to be together with for eternity, he knew somehow. However, she'd remembered something he'd told her once, and he could hardly remember that; this showed that she \emph{was} in some way interested in him, even if that way was jsut a friendship. But she would not show so publicly, when he was talking to her, she wouldn't stop if she was on her way home, but end the conversation quickly and say \enquote{Goodbye!}.

The other day, the small brother of R. had tried to make out a girl for him, but for our protagonist, this was just some sort of a game he'd play for the sake of that boy. 
And once more, we're nearly finished for this time; the only thing I want to act is the fact that a professor had offered him a ride, which impressed him even more as this was one of the professors that seemed so distanced from the students, though she'd always been different with him.

Now, sleep is the only thing I need. I know that it won't last very long, but it's important. 
I beg you to stay tuned\ldots
And give me your opinions, of course\ldots

\begin{quote}
A lightning \\
was connecting the clouds \\
in the summertime; \\
Rain \\
would connect heaven and earth \\
in spring and autumn. \\
But what was it, \\
that unnamed, \\
powerful thing \\
that could link people? \\
Was it friendship? \\
Or rather love? \\
What was the basis of all this? \\
Can we really answer this question \\
without questioning ourselves? \\
--- W.G.
\end{quote}

\begin{quote}
Sad people\\
were all around. \\
Tired students \\
were running to and fro. \\
Dizzy \\
was he, standing in the middle of all that, \\
knowing that there was a centre of thought \\
in all of them, \\
and happiness \\
lying ahead; \\
and pain, of course, \\
as eternity was awaiting all of them. \\
--- W.G.
\end{quote}

%%% Local Variables:
%%% mode: latex
%%% LaTeX-command: "latex -shell-escape -synctex=1 --file-line-error-style"
%%% ispell-local-dictionary: "british-ize-w_accents"
%%% TeX-master: "../My_Life_in_a_Nutshell_-_First_Era.tex"
%%% End:
