\chapter{Why?}
\label{cha:why}
\subsection*{Originally published: \DTMdate{2006-01-18}}
\begin{quote}
Hello; I'm happy you're back with me again, taking interest in my story!

Wednesday; an interesting, but long day. A time when a lot of sand would pass the wheels of time without moving them quickly; in short: A boring day. Nevertheless, there were some important moments I want to tell you about; and some old sensations that jumped back at me, as memories happen to do all the time.

But before we start, I just want to tell you something of importance: After reading through my texts again, I noticed one may think I'm changing the recipient of my feelings quite fast; I'm gonna explain you why I'm not, before you start thinking about me in a way I wouldn't like. 
There always is a kind of development in human life; some even say 
\begin{quote}
\enquote{[\ldots] Nought may endure but mutability [\ldots]}.\\
--- Mary Shelley, \enquote{Frankenstein}
\end{quote}
But I'm not telling you it's normal to undergo such fast changes; I'm tired to the death, when I remember the times the machines of puberty began to work, when the idea of \emph{"being crazy about somebody"} was normal.

Everything has changed since then. 
Today, I'm not \emph{crazy about somebody like a fool} (that's a pun, though you won't notice it, as you'd have to know about quite an old English song); I'm in love, and that's something comepletely different.

Love is something you won't miss if you didn't experience it; or, to explain it the way Philip Pullman put it (we're currently just finishing that chapter, it's one of the last of the third book): Being in love is something like China; you always knew it was there, but it wasn't of any interest; and suddenly, you realize that you've been there.

Then, you've fallen in love, and you'll always wish to return to that place. 

I guess that's quite an interesting and also easy explanation; However, it may take you some time to understand this metaphor in all its extent. 
Thus, I don't think I ever was in love with several persons or even hopped among them with my emotions; I think, I now fell in love for the first time. 
Something else you should not forget: This all took place in a time of no less than 2 years, and it started even more than 5 years ago. 
But enough for an introduction: We'll have to start now! 
\end{quote}

It's Wednesday, but you already know that; Maybe, just now, you happen to ask yourself why this text is entitled \textbf{\enquote{Why?}}. 
Well, that's a good question.

It's the question that always comes up when you don't understand something; and I don't understand a lot of things, though I seem to be understanding everything; but every human being could be like that. 
So what are these \textbf{\enquote{Why?}}-Questions I found my thoughts trying to answer?
\begin{quote}
  Why did I fall in love?\\
  Why did I realize to late?\\
  Why didn't she tell me?\\
  Why \ldots?
\end{quote}
You see: There are enough questions one may ask oneself. Most of these are not to be answered easily, and a whole bunch of these won't ever be answered.

But questioning and lack of knowledge is the basis of human existence.

Maybe, we're not searching to achieve new sections of knowledge, but struggling not to forget too many of them, as we're always receiving sensations we can't grip; or don't want to realize.

Just remember the moment before the First Denial: I didn't notice the soft light that emanated from her eyes (it reminded him of the sunflower, her favourite plant; but it was glowing in an almighty way, sparkling and sharing her energy with the world around), the power that was crackling in the air, the lightning that everybody felt who was sitting next to me; How could a living creature fail to recognize the strongest emotion of all?

Possibly, he wasn't ready yet; most probably, he was in fear of something he couldn't control, without even realizing what it meant consciously. 
Failures exist to be made, and we exist to learn from the shocking results.

That's our sense of living; that's our mortal creed. 
This day, he didn't really have a chance to see O.; However, he found a picture that was taken from her and published, but her face was hidden beneath her hair, and the resolution was bad. However, when he saw the picture, it took him less than a second to realize she was there, too, among several other people; Nevertheless, he had been faster in the past.

His heart didn't feel the same way: As she passed next to him once, it pounded so loud he was afraid P., who was next to him, would hear it; And the physical effort it took him to rise the flight of stairs before shouldn't have exhausted him in any way comparable to such a feeling.
That moment was today, and there was nothing more to it; but it shows, that he's still in love.

Another occasion to see her that day was rendered impossible by an assembly to plan something for the university; as she passed him and asked the crowd (which consisted of three people, indeed) where it took place, he answered first, smiling at her. However, she looked at the other two girls which were not really good friends of her. Not that she didn't like them, but they were quite new at that university, and she only had a small amount of real friends; apart from her best friend, a girl that might be of importance later on, he'd always thought he had been the other person of most interest to her; now, everything seemed different, as her eyes just flicked towards his for less than the tiniest part of a second, and he could only see that she recognized him; and he felt a terrible cold shiver, because the glow he always saw had faded away, or had been hidden.

He couldn't know what had happened, though he was eager to find out; however, there were three possibilites, which could be proved by different arguments:
\begin{enumerate}[(a)]
\item she had loved him, and right now, she was trying to get him out of her mind;\label{item:why-1}
\item she had never loved him, just starting a friendship which wasn't \emph{really} important to her;\label{item:why-2}
\item she had broken up with her boyfriend and just was in a phase of depression.\label{item:why-3}
\end{enumerate}
Possibility~\ref{item:why-1} could be proved by all the moments I've allowed you to share with me, and by the fact, that she had a new boyfriend and seemed to be happy with him. Thus, the danger to fall in love with \emph{HIM} wasn't worth the fun she'd experience with him in a friendship, as he was quite quiet, thinking all the time, not knowing what to say, while she liked his puns and the statements that passed his lips just at the right time to give her a hearty laugh.

Possibility~\ref{item:why-2} can also be proved when the perspective one takes when looking at his experiences is changed; she was lonely, just searching for somebody who would be a friend. On the other hand, a friendship wouldn't simply cease to be; something had happened; He felt that, but he didn't know how he could find out.

As a conclusion, possibility~\ref{item:why-1} seems to be more probable, while she could also be experiencing pain because she broke up with her boyfriend; but he guessed he would have found out, and he knew that hopes are never to be fulfilled if they are conciously. Thus, he tried to ignore her, too, though he wasn't able to believe in the words he was telling himself, the phrases that commanded him to stop thinking of her. However, she stayed a happy person; but most times, he saw her alone, and she had become more quiet, which may be a result of the fact he didn't really see her for longer moments than some minutes.

And she was never alone, but he didn't know how that would change anything, as she was already in a working relationship. 
That day, he nearly managed to do the trick and ignore her; but nonetheless, he knew that this achievement was temporary. Tomorrow, there would be chances to see her, perhaps even alone.

We'll see what happens; I'll tell you tomorrow, but today, I'm tired and exhausted; denying feelings to oneself to fulfill your dreams is a hard job, and you'll never know if it's worth the effort. 
Now, I'm bound to tell you that I've lied, or at least forgotten something on purpose: There's also a fourth possibility, (d). 
Probably, she thinks I'm gay.

That could be a possible result of the fact I'm spending a lot of my time with girls, and I seem to understand them; that doesn't necessarily mean I'm gay, but some people may think so, as I never had any girlfriends (on the other hand, I never had any boyfriends). The only thing I could do about that without hurting her is telling her to read these texts; but I don't know if this is the right choice, and I'll decide about that at some later time, as even though she doesn't know this pseudonym, she'd find out when she read it. 
We'll see. 
I'm looking forward to meet you again, here. 

\begin{quote}
It was night;\\
all the lights faded away;\\
all the candles burnt out;\\
all the shadows alive. \\
That was the time I started thinking;\\
I thought of my love for you, \\
thus colouring the night;\\
making the shadows, the thoughts and feelings,\\
and finally the world,\\
blue. \\
--- W.G.
\end{quote}

\begin{quote}
No feelings are escaping my enchanted soul;\\
you hold the key to let them out,\\
but you won't ever know\\
unless the mightiest of powers\\
has freed my soul. \\
You hold the key to that power;\\
you happen to hold my life\\
in the soft hands\\
that are touching the palms\\
of someone else. \\
--- W.G.
\end{quote}

%%% Local Variables:
%%% mode: latex
%%% LaTeX-command: "latex -shell-escape -synctex=1 --file-line-error-style"
%%% ispell-local-dictionary: "british-ize-w_accents"
%%% TeX-master: "../My_Life_in_a_Nutshell_-_First_Era.tex"
%%% End:
