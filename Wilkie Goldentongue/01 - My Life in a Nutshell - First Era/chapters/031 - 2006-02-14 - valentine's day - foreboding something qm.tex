\chapter{Valentine's Day --- Foreboding Something?}
\label{cha:valent-day-foreb}
\subsection*{Originally published: \DTMDate{2006-02-14}}
\begin{quote}
Next one --- This time, a bit shorter!

A lot of things have happened, again. 
Nevertheless, that also means that time's running short; and I don't want you to call me a bore. 
Thus, we'll try to concentrate on the most important things, starting with some details I forgot yesterday.

What I'm asking you is: Valentine's Day --- Foreboding Something?
\end{quote}

Tuesday, the 14th of February --- Valentine's day. 
The evening of the day before he'd imagined that he would receive something, a message, a present, or something else. 
Well, he'd receive something else: Two shocks. 
Or, at least, one.

But before we'll go into detail, we shall have a look at something of importance we've forgotten yesterday\ldots
When he was sitting in the bus, on his way home, he passed by O.'s house again. He wasn't so sure if she was already home, but when he saw her car in the front of the house, he was. And then\ldots
He saw that the door to heaven stood wide open. 
And not only that: The door behind it was opened, too. What did that mean?
What was going on?

He wouldn't know, and he's still wondering about it --- whether it means that she's still waiting for him --- sort of --- or whether it was to say that she'd accept everything. 
As it was cold outside --- pretty cold --- and as the door stood open for at least half a minute \emph{(though he was probably just feeling as if it was half a minute, and the time would surely be shorter, in fact)}, this could not have been happening by chance. And no person was to be seen. Ridiculous. It had to be a sign\ldots
But what did it want to tell him?

The moment passed, and today, he would be looking for the door again; However, he had sat down heading to the other side of the street, and though he believed he wouldn't see the door, he \emph{could} make it out, indeed, thus proving his theory once more. But he could just focus on it for some seconds, and before he could see if it was opened \emph{(though he was no pretty sure it had been closed)}, the view was gone, because he'd just instantly hoped to see --- no, \emph{known} he'd see --- if the door was open. Of course, that meant he didn't. 
Uncontrolled thoughts made chance become real; that was one of the most interesting and depressing things to know.

Now, before we'll switch back to this morning, we'll just have a look at something else: When he stood there, waiting with some fellow students for the lecture to begin \emph{(on Monday)}, he saw O. again. She was just some twenty metres away, but he could make her out clearly; she was rushing up some stairs, and probably, she was late, as some other student had been searching for her and her best friend. Now, he was sure that this student was gone; O. \emph{was} too late.

On the other hand, there could be some work she'd have to do, but there was now lecture she'd have to attend right now. Was there? He wasn't sure anymore. 
His centre of life had been destroyed, and he was still searching for a new place; a better place to be\ldots

For now, he'd concentrate on himself, or try to do so, at least. Today, he'd also have contact to P. and her friend, while he'd had exchanged some messages with another friend of P.'s (a girl) via the internet yesterday evening. She seemed friendly, and probably, he'd even seen her several times, but he just knew her name without having any connection to her face. 
But there were more important things happening today: L.-B. walked up to him, while a whole group of students stood around him, all working together, while he would be the centre. He was concentrating, and when L.-B. cried his name, he tried to continue; but he was shocked. It didn't take him by surprise that she'd made up some name for him, but the way she said it --- in public --- was as shameful as if she'd called him darling.

And he'd never be her \enquote{darling}. P. sat next to him, turning her head to face him, presenting an astonished and puzzled look; he tried to return it, though L.-B. was looking at him, too; but he was pretty sure that she wasn't searching for a reaction, as she wouldn't be so focussed on details. And he was really shocked. 
Now, he wasn't just \emph{assuming} that she was crazy about him; he \emph{knew} it.

And P. knew it, too, but she seemed to have realized that he didn't want L.-B. to do what she did. 
The next thing that happened after he'd ignored L.-B. and got loose of her, finally, was P. telling him that she didn't have a boyfriend, but in a way that could also be just normal to tell him, as she was suffering because others thought she had.

However, it was some sort of coincidence; but he was pretty sure that this was of no importance. 
Not really. 
Hopefully?

Well, P.'s friend seemed to be the same way she'd always been, though she appeared to have been waiting for something that didn't happen. Why was he suddenly realizing that all the people around him were making such a fuss --- about him?

Something else: When he stood at the bus stop --- alone, like he did most time --- another friend of P.'s, another girl, would arrive, exactly at the same time his bus was turning round the corner. The only thing he could do was to exchange a smile and say goodbye --- and of course, he wished her a nice time waiting for her bus --- alone. 
Some irony he'd developed, probably some artifact he'd copied from O.

Up to now, nothing more seemed to have happened, though he knew he'd see Y. again, soon --- and, that a lot of lectures would be cancelled tomorrow. 
Finally, he'd see P. and her friends the next evening --- we'll see if anything happens; soon. 
Probably not tomorrow, as time is limited to some number close to zero.
Too close to zero. 
Hope the number of new opinions won't stay so close to zero!
Still with me?

\begin{verse}
The door of heaven stood open, \\
widely; \\
Letting all the people below \\
see the beloved land; \\
would one ever reach it? \\
--- W.G.
\end{verse}

\begin{verse}
One, two, three --- \\
remember me. \\
Four, five, six --- \\
break the sticks. \\
Seven, eight, nine --- \\
you'll be fine. \\
Ten, eleven, twelve --- \\
if you know yourself. \\
--- W.G.
\end{verse}

%%% Local Variables:
%%% mode: latex
%%% LaTeX-command: "latex -shell-escape -synctex=1 --file-line-error-style"
%%% ispell-local-dictionary: "british-ize-w_accents"
%%% TeX-master: "../My_Life_in_a_Nutshell_-_First_Era.tex"
%%% End:
