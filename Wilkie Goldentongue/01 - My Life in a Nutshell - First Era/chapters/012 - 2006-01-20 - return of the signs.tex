\chapter{Return of the Signs}
\label{cha:return-signs}
\subsection*{Originally published: \DTMDate{2006-01-20}}
\begin{quote}
Hello, nice you're here again!

In that text here, we'll just have a look at the tiny details I was capable to notice today, as this Friday seemed to be something special; it wasn't Friday the 13th, that had been last week, and I'd already told you about it; this was a special day for other reasons, as you'll see if you join me again and read the following passages. Another part of my life --- just published publicly. 
\end{quote}

This day was a day of the two; not only th number of partnership, but also her number. That meant, as he knew about it before, that he wouldn't see her today.
But there were to be signs that would remember him of happier times. 
Important signs which meanings he's still figuring out.

It all started at about half past 12; and it took just about an hour. 

This hour was a complete summary of what had happened, and maybe, it would also tell him what would happen; but he didn't know if he really wanted to find out, as this might be the worst thing to do:
One can only hope if he knows something about his future; and if one hopes, this hope will not be fulfilled \emph{(we already figured that out some time ago, didn't we?)}. 
So we'll just have a look at these signs from a completely objective point of view; interpretation is up to you!

It began when he entered the auditorium to hear the next lecture; soon after he'd sat down, some of his fellow students would cry out loudly, because the lecture was cancelled. That meant he'd have to hurry up, as the bus was to come in just about four minutes. So, there were four minutes to go; alone. 
No time to look at the plan if any lectures would be cancelled on Monday; no time to say goodbye to P., who was with him. 
Only time to run. 
He ran.

Or, to be more exactly, he walked fast. 
He reached the bus stop in time, to join one of the girls he knew which wasn't included in one of the groups that inevitably develop in a mass of students; she was standing there alone, and he talked to her for some moments before her bus came and took her away. Another group of students, that had been huddled up with another girl he knew, just raced to their bus who arrived at that moment; He would have to wait for his until P. had arrived at the bus stop, too.

He walked to the other girl standing there alone and talked to her, about the books he liked to read, about the car both of them wanted to own so as not being forced to use the bus and several other things of no real importance to us.

P. arrived and joined them, taking over the active part of the conversation without really changing the topic. 
Finally, the person of our interest told a funny story that had occured to him yesterday in the bus, but it's of no importance to us now. 
Both the girls were looking at him in quite an interested way, something he hadn't experienced since the day O. stopped looking into his eyes. 
For some moments, he was happy again; then, the bus came and took him away the same way the separated girl was taken away before. 
When he entered the bus, he decided to take his seat behind an older lady; Something in his mind seemed to know her face, but on the other hand, she seemed completely new to him. 
Soon, when the bus was nearly filled with people, some children asked whether they could take seat in front of the lady; she granted them this permission friendly. 
The two young girls sat down, and soon after, an even younger boy approached, asking the lady whether he could take seat next to her. 
He was allowed to do so, too.

Only seconds later, the two girls, who were now forced to look into the eyes of that boy, stood up and went to sit on some other place simultaneously. 
The boy didn't follow them.

Some minutes passed, before the lady must've asked the boy where the bus was currently located; the boy must've misunderstood this question and asked the driver whether this line would pass through a certain part of that village.

The confusion of the three persons that were now involved soon cleared, as the lady explained her question again and told the boy where she wanted to go; the village where O. lived. 
He'd once seen her granny, but he wasn't sure if he could recognize her today; he thought the lady might be that person, though. 
However, there were several thing that didn't fit into that idea. 
\begin{enumerate}
\item Why didn't she catch her from the station? She would be there just about an hour later, he knew that. 
\item He was pretty sure he'd memorized her voice correctly, and he could neither recognize that nor any features the two of them should have in common.
\item She seemed older. 
\item O. had visited her granny just two weeks ago, and she was living quite far away; in addition to that, she was not so healthy that she would simply come to visit her surprisingly. 
\end{enumerate}
For that reasons, he decided to sleep a little, just for about five minutes or so, for he was really drowsy, and he liked sleeping in buses or planes; he could sleep there even better than in beds, as the feeling of moving and the regular vibrations of the chassis calmed his racing thoughts down a bit.

Some time later, he found out the boy had shown the lady the button she would have to press for the bus to stop at the next station; After our protagonist had recognized that, he decided to press the button for her.

And he did. 
She didn't notice, and when she turned around to do it, he told her he'd already done so.

She smiled, but he didn't recognize her face nor her smile.

When she stood up after the bus stopped, a coin could be heard falling; It took the person of our interest some time to wake up from the state of oblivion that had caught him for a moment. The lade was gone, and he rised a bit, just high enough to see the coin.

It was of nearly no worth, at least. And as the woman had already exited the bus, he decided not to bother her, as he didn't see or hear other coins. 
But before the coin fell down, there was something that had caused this state of being completely absorbed in contemplation: A garbage truck was passing in front of the bus, just before the last turn they had to make to reach the home of O.

That rang a bell; the end of something; garbage meant the end of something, and a garbage truck would mean carrying the broken (or \enquote{cracked}) pieces of something away to make place for a new beginning. 
What was happening?

Theories were developing in his mind; that was the first time he'd noticed a garbage truck in that village; something was changing, this was a \enquote{sign}. 
But what it meant was something he struggled not to find out, as it could be positive --- or negative. 
He didn't want to know, though he wished he did.

But we stopped at the moment the lady exited the bus; we held the time right there, and now we will try to join in there again and take back our seat in his brain. 
His head turned when the lady went in the direction of O.'s home; but he wasn't able to track her way completely, as the bus started his journey again and gained speed quite fast; faster than normally, it appeared to him. And as the windows weren't really see-through; there were ads fixed on the outside, and little light was passing through, thus separating the passengers from the outside, filtering the light in a way that could be compared to the shadows in a prison, if you were in a desperate mood --- as he was, indeed.

He even imagined having seen the lady turn towards \emph{her} house; but that now seemed just like a dream his mind had shown him. 
The signs were gone again; he would try to ignore them, as he did in the past, when he got to know her, as a repitition might be the only way to lead to another repitition. Maybe, he should try to focus his feelings on somebody else, so he wouldn't think about her, which would be the best way to prepare a new beginning --- with O.

But there was this voice in his head, singing:
\begin{quote}
They passed me by, all of those great romances \\
{[\ldots]}\\
My picture clear, everything seemed so easy \\
And so I dealt you the blow \\
{[\ldots]}\\
Now it's different, I want you to know \\[1\baselineskip]
One of us is crying \\
One of us is lying \\
In h[is] lonely bed \\
Staring at the ceiling \\
Wishing [h]e was somewhere else instead \\
One of us is lonely \\
One of us is only \\
Waiting for a call \\
Sorry for h[im]self, feeling stupid feeling small \\
Wishing [h]e had never left at all \\
--- extract of the lyrics from quite a famous song
\end{quote}
The cruel feelings that always came together with the most beautiful feeling of love wouldn't go away so easily; maybe, a repitition of history would prove to be impossible, as everything had changed --- and would change, forever.

He didn't believe in it, but he hoped for a better future, which could be his greatest error since the time he hadn't realized his and her feelings. 
But he decided to ignore that, too, trusting in fate though it had betrayed him; or he had betrayed fate, that depends on the point of view you're taking. 
It's yours to figure it all out. 

\begin{quote}
  Puberty is the discovery of emotion --- and the complexity of life.\\
  --- W.G.
\end{quote}

\begin{quote}
There was a crack in that shell,\\
and it will soon burst open;\\
maybe, this will be the end --- \\
or a new beginning. \\
The nutshell is going to break,\\
for the sake\\
of emotion. \\
--- W.G.  
\end{quote}

%%% Local Variables:
%%% mode: latex
%%% LaTeX-command: "latex -shell-escape -synctex=1 --file-line-error-style"
%%% ispell-local-dictionary: "british-ize-w_accents"
%%% TeX-master: "../My_Life_in_a_Nutshell_-_First_Era.tex"
%%% End:
