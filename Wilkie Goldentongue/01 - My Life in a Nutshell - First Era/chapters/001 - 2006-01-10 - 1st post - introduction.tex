\mainmatter
\chapter{Introduction}
\label{cha:introduction}
\subsection*{Originally published: \DTMdate{2006-01-10}}
It was a normal day, influenced by the number nine, a strong, impressing one. Thus, he tried to control his thoughts, not to give away something of his real personality, trying to conceal as much as possible to ensure the possibility of his own future. He entered the bus, as he did every morning, and took the same seat he took every day, when he went to school, that is. The next action he had to do was to look around, in the steady hope of finding somebody who would accept his look and answer him, simply by not looking away or feeling that she was watched. His hope wasn't fulfilled, but that was part of the daily ritual. Finally, he turned to look out of the window, seeing the same houses passing by, the same cars and the same numbers on their plates. Nothing changed, though everything was changing continuously. Not a thing was going right, but he did'nt really expect that fact to change. While the bus moved, he watched the people going or running outside, the small cute lights that threw masses of photons on the faces of the children behind the windows of their homes, them eating the breads their mothers had prepared for them. They already looked at the watch, as if in a hurry, calculating the amount of minutes they had to sit before they joined the others outside to reach the bus.

Once, he was one of them: Living as if every day was the same, as if nothing changed or could be changed. Now, he had changed for the worse: He was \emph{thinking}, the biggest crime one can commit. If somebody wishes to understand the world, he first has to understand himself and his fellow men. That was something nobody could manage without strong, uncontrollable feelings of pain, but he was one of the rare people who managed to hide these emotions and smile all the time without ever giving away anything of himself. That was the only thing that had protected him up to today, and all the same, it was the only thing that he felt would kill him one day, if he didn't change. \textbf{Her} house passed; He had a close look at it, for him it was all different: It seemed to be the only coloured thing outside, the only thing having an own personality. He discovered her car in front of the house and wondered if she was still eating or already on her way to it. He would never know, as the moment passed and he wouldn't ever dare to ask her. A short, afraid look in the direction of the building, followed by a small shot of pain right inside his brain and his guts --- that was part of the everyday-ritual --- and he tried to look away, to conceal the short moment when he had been unable to control himself. She was the person that had changed him, and fate would decide if she would ever know. He hoped for it, but didn't believe she'd ever find out about his love for her. She had a boyfriend, he knew him: A serious, but happy boy, he seemed to have a good sense of humour, but he also had a quite determined look: something he himself could have, too; but he knew he wasn't determined at all, pretending everything but being unable to make up his mind to find his own opinion. That was his biggest fault, and it all had begun when he was quite young: He didn't wan't to go outside, not seeing a sense in jumping around and smashing windows with balls. He didn't have real friends, though he tried everything to get some. Finally, he was the friend of his teachers and they were his best friends, too. The only other people that liked him were these who were excluded from the groups of sympathy (you'd rather call them `gangs') that represented the majority. He made presents, he helped every way he could help, but that made it even worse: He didn't happen to know the power unrestrained envy could unfold, if the valves to this all-destructive emotion had been opened once; He was an innocent boy, somebody who really believed in what he said; He believed in God, and this belief was one of the single powers they couldn't take away from him. They couldn't even touch it; others were to come to take that one away, or at least to cast off a fundamental change in his sight of the world. Thus, his self-conciousness was destroyed, when he was still young; he became a quiet boy, in short: A pupil trained to play the role of the adult, just missing the authority of one and thus being left to endure the physical and psychical pain he was thrown into by the people calling themselves his friends. 
That was his youth; he was traumatized before he ever experienced real feelings, and these came to melt the block of ice that had built up around his heart, but destroying the rest of his inner stability, too. Nevertheless, the illusion of a strong person, so strong that no storm would ever be powerful enough to stop him from standing up to it, was perfect; He was different to everybody, and as if he was a computer, he had different profiles that started themselves in regard of his environment. If several of these were activated at the same time, he had a problem; but up to now, he'd always managed that situations, too. He seemed to react exactly the best way one could expect him to act; It had been hard work for him to be able to do that trick, but now he had different opinions when he was in contact with different persons, and all the same, he seemed to be something individual. He wasn't a human being anymore; He'd devoted himself to his environment, he'd given up his life for the sake of the others; that seemed to be his fate, the final fulfilment of his life. 
Up to the day everything changed; the day \textbf{she} walked into his life as a bunch of happiness and innocent, but guilty destruction\ldots

\begin{quote}
Life is full of pain; accept it to be allowed to feel happiness. \\
--- W. G.
\end{quote}

\begin{quote}
Never believe that something will finally sort out right --- it won't. \\
--- W. G.
\end{quote}
