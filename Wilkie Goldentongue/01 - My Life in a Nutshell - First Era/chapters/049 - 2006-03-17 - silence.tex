\chapter{Silence}
\label{cha:silence}
\subsection*{Originally published: \DTMDate{2006-03-17}}
\begin{quote}
Hello, it's me again!

Everything is so time-consuming; though I should have had enough time to write something yesterday, I hadn't. But so many things have happened, and the notes I took to record these events are becoming more and more. 
Thus, we'll have to start right now.

He was listening, and all he could hear was: Silence.
\end{quote}

Before we'll go on having a look at the recent events, we shall go into detail about Monday morning, as I've \enquote{forgot} to tell you some quite important things. Probably, this was just the consequence of trying to forget about them.

Monday morning --- when O. sat down next to him, she complained about his bag, which seemed to be bothering her. He took it away. And she hadn't met his eyes in the whole process\ldots

But this was something that could happen with everybody, something normal; however, something else was yet to happen. When she would be in need of a ruler again (at least, he \emph{thought} she was), he asked her whether she wouldn't like to use his ruler, and she'd proudly point at hers --- which was new --- and say: \enquote{Now, I've got one, too.} Her smile made it impossible for him to become furious; nevertheless, he'd feel hurt. But he didn't show anything, but kept quiet, so as not to disturb the silence of emotion. She seemed not to notice a thing, and the only thing he'd remember when looking at his ruler, which was now to himself, was the colour of her favourite pen which had once been smeared to it.

It was now gone, and what had been left was vanished into thin air. Only the memory, carefully locked away, would remain.

But that was gone, now. Another memory flashed into his mind; something he'd forgotten about Saturday; you've already been told about the sadness of P., which was based on some sort of illness. This illness was painful, and on that very Saturday, our protagonist was stepping into the bedroom while it was pretty dark. The part of his body where P. would feel this pain hit the bed, and he was hurt --- not badly, but strong enough to be forced to remember what P. must feel. That was something he felt to be important, as he'd just been reminded of that when he took a break after having started writing and felt a shot of pain at exactly that part of his body; it was this event which remembered him of the nearly lost note he'd forgotten to tell you about.

That note would also tell him about having seen B.-B. on Friday, and he remembered having seen her on Monday, too; he realized that she'd been nearly next to him, but there was no need to talk to her. 
Another girl caught his attention; she was asking him to help her with some lectures she wasn't that good at, and he agreed, but he'd never accept any payment from anybody. 
So much about that.

So, what had happened on Wednesday? B. had been silent again, and though he'd told her about a new message he'd sent, she seemed not to listen; though she was still laughing about some things he did, she seemed to keep her distance. Yesterday, somebody had assumed that P. and her friend were after him; well, he wasn't so sure they weren't, but he was quite sure that this was part of their character. All those things out there; we don't know anything about them.

Probably, things were better this way. He wondered what progress had given to us humans; What was progress?
Progress was finding out about things; Progress would make our lives longer, and easier. And more cruel\ldots

He realized that progress was something special; it was not really predictable, and nobody knew if it would go on forever. But the question was: Was progress something positive?
It would make us less ignorant. That was the main point of it. And people which had made progress would not understand how they could have lived without it. But the fact that they have done is soon forgotten. And progress renders us incapable of thinking what would have happened without it. Thus, we cannot really judge on it.

Which leaves us with the possibility that ignorance could be something good, because if we don't know that the fire will kill us we won't be afraid. There'd be less fear\ldots
And less happiness? Probably. Everything would be harmonic, again. Yes, this would explain progress as something pretty similar to the rise of the entropy in the universe: The reign of chaos and the reign of logic would both make our lives more extreme.

The only difference was the fact, that chaos was a natural order, and logic was artificial, controlled chaos. 
But we didn't realize that, as we felt that we were not natural, but something more --- \emph{civilized}.

He'd talked to one of his professors for a pretty long time; those people were more like him when it came to some aspects of life. However, they were somehow different; they'd understand him, but --- they'd not be interested in the same things. But his profiling system captivated them for a longer time other students could do, and this was one of his secrets for his success: The feeling of having something in common with everybody.

Probably, I've forgotten many things --- again. But time is limited, and I'm bound to continue working now. 
I hope that more time and events are to arrive; and I hope, too, that you'll accompany me again. 
Opinions?

\begin{quote}
Slow \\
was it's wind. \\
Strong \\
was it's power. \\
Steady \\
was it's reign. \\
And never doubted; \\
as silence \\
was eternal. \\
--- W.G.
\end{quote}

\begin{quote}
Eyes of plastic, \\
ears of paper, \\
mouth of glue, \\
hair of cable, \\
nose of wood --- \\
but heart of gold \\
beneath the shell \\
of rust. \\
--- W.G.
\end{quote}
