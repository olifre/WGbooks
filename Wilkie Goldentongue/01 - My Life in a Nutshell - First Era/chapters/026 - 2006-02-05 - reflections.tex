\chapter{Reflections}
\label{cha:reflections}
\subsection*{Originally published: \DTMDate{2006-02-05}}
\begin{quote}
Hi! Nice you've found that page again!

Saturday, Sunday --- weekend. 
More time to remember things. 
Time to suffer?

Find out by yourself: Reflections.
\end{quote}

The weekend was shorter than ever. 
Now, he didn't \emph{have} to think of the past. 
This time, he could do so if he wanted to. 
And several times, he did.

The signs had come back again, pointing at her and her boyfriend. But this time, he accepted that they could mean that this couple would persist. Together. 
And he wished she was happy. 
Finally, he knew he'd sit next to her again --- the next day. 
But now, he didn't look forward to it with every fiber of his body. 
He knew that his love was locked off, and he just felt a kind of friendship. The pain had gone, and happiness had replaced it; Nevertheless, there was also emptiness among his feelings. 
But he knew that this wasn't to last. 
At least, not forever. 
That was something he knew for sure.

But the memories would also stay with him. 
The glimpse of a second when her hand had touched his shoulder, when he'd presented her somthing; The shudder that accompanied the wisdom that she'd been single. 
But she wasn't now. 
There was even more to that relationship he'd thought her to have; It was not simply something fixed. The relationship she'd now seemed to be something fixed. That first relationship had been more\ldots
She had been \emph{fianced}.

Thus, the feeling that her talking to him was nothing more than a game was based on facts. 
And, in addition to that, she wore a ring. 
Later, he learned that it was nothing more than a piece of jewellery. 
But when he found out about that, it was too late. 
He thought of the many times he'd sat in the bus alone; The minutes when tears were forming, and the filter these emotions were imposing on his eyesight. 
All of it was gone.

He knew that her current boyfriend was about one or two years older than him. 
Maybe, time would change everything, as it always did --- but the possibility that everything would change to his benefit was utopically small. 
A melody got hold of his mind, when he slurped his tea; An old, familiar and happy tune. 
And it told him a better future was to come. 
His emotions were getting hold of him again, just for a second; Then, he felt the power of freedom rushing through his body, slowly killing the sickness that was torturing him. 
There was some sensation in his mind, some sort of signal: It told him to turn on the monitor of his computer, who had turned that visual display of to save energy. 
And, it told him to look at the clock. 
It showed 17:39. 
The checksum was two, her number.

When he finished writing this short text, he looked at the clock again; Now, the checksum was three, the number of her boyfriend, and when he looked at it again, it was four, his number. 
He felt how amazing that was, as he hadn't missed a minute. 
What did it mean? Was there to be some development, from the 2 / 3 --- couple to the 2 / 4 --- couple? Or was it all just an illusion?
Before he could have a look at the lower right corner of his screen, it turned off again. 
It had just been on long enough to show him that numbers, and the most appalling thing was the fact, that he'd \emph{felt} that he had to switch it on. 
Amazing.

But he knew it could just be some sort of trick his subconscious had played on him. 
On the other hand, there was no digital clock next to him; Just am analog clock at the wall. 
And he couldn't recognise the numbers it showed. 
Thus, his subconscious had to be extremely exact. 
He thought that this was quite impossible. 
Nevertheless, it was the only explanation that didn't include some external power; Things like that could happen by chance, of course, but not such a big amount of them. 
The chance would be \SI{10}{\percent}.

On the other hand, when you added several events, it would sink logarhythmically. 
Was that the proof for the existence of good, or some separate entity?
Probably. 
But this was no scientific research at all.

But wasn't everything --- including life itself --- scientific research in the end?
Are we not just searching for wisdom and explanations, religion being the explanation for everything we can't explain?
He was shivering, though it was pretty warm in the room where he typed those texts. 
The agonizing memories were still there, somewhere. 
But he knew that they wouldn't bother him anymore; She was the only person to reactivate them. 
His method had succeeded: He had not forgotten anything, but his love ---  and his pain --- was deactivated, sort of. 
Nevertheless, there was some emptiness inside him. But he felt that this place could be filled again. 
And this time, he would wait until there was something that would stay there. 
Even, if that meant waiting for ages, or for O. to return. 
Time was not important anymore.

If she'd go away, he would hope that she would be successful; And he knew that she'd have better chances at the place where she would probably head to, as he'd been there, once. 
Was that the moment when those stars in the film would say: \enquote{Go and make your fortune, I love you, and I hope you'll always be happy!} ?
He wasn't so sure about that; This was part of a story, but it was quite improbable to happen in reality. 
Nevertheless, he felt like it.

Did that mean he wasn't alone with that feelings?
Of course, the chance that there were other persons who'd endured similar feelings was quite big. 
But in the end, everybody was an individual. 
Suddenly, he realized that everything was linked, that all the consciousness there was could just be part of a gigantic experiment, of an impossible computer. 
Calculating the Great Question of Life, Universe and everything else there is\ldots

He smiled.
Douglas Adams had probably not thought of anybody who'd take that literally. 
Maybe, he had?
Who'd know?

His granny distracted him while he was writing these lines. 
She told him about her sister, who was suffering from cancer. 
Well, in fact, she was suffering from the treatment, as she'd just lost all her hair. 
Did fate show him how happy his life was, finally?
Probably.

But in fact, he \emph{knew} that in the end, everything would just sum up to zero. 
Which meant, that life was predefined, in some way. But we still had the possibility to decide, between a limited amount of options. 
That's what everybody hoped. 
Did that mean that it was wrong?
He didn't dare continue thinking about it. 
He looked outside. It was dark, and the lantern dimly lit the street. The time when the rays of the sun hit that place, and the time when they didn't; They would also finally sum up to zero, if you took the integral of both.

Everything was somehow symmetric. 
But why was the entropy rising then, steadily?
Was there energy leaking out?
Why had this process started?
Was this the final question?

Or was the explanation too plain to see: Was it just the particles of consciousness which were ordered, while the entropy of matter was steadily rising?
But why was he so puzzled, then?
And in the end, this would mean that there was some sort of final aim to it. 
Which brought him back to Douglas Adams again. 
He smiled.

He'd proved something he thought to be completely impossible from two different points of view. 
That's exactly the same thing a mathematician does when he's proving something: Two different starting points finally lead to one statement. 
And everything is connected somehow. 
The number two was the base of it all. 
And it was the base of the number four.

It was fate's turn now, as he didn't know what to do. 
If you're wondering if it will act tomorrow, then you're bound to wait 'till then. 

\begin{quote}
The game \\
was on again; \\
His soul was free \\
once more; \\
Who would be, \\
the one, \\
to knock \\
on that door?\\
--- W.G.
\end{quote}

\begin{quote}
Crack! \\
The tree hit the ground \\
with a giant BANG. \\
The same tree \\
that had supported \\
his life \\
for so long. \\
He'd always known, \\
that it's roots \\
had been given no foundation; \\
and now, it paid the price. \\
But the last apple would be saved; \\
the tree could grow again, \\
if a foundation \\
was ever to return. \\
--- W.G.
\end{quote}
