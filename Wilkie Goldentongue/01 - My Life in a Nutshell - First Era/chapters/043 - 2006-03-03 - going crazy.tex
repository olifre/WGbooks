\chapter{Going Crazy}
\label{cha:going-crazy}
\subsection*{Originally published: \DTMDate{2006-03-03}}
\begin{quote}
Hello!

It's Friday, and just in contrast to the prediction that this day would be accompanied by a whole bunch of time, it wasn't. However, some special things had happened, and his once so stable emotional stability was shaking again; you'll understand when taking into account all those things which had happened, the present situation and the foreshadowed --- or rather unpredictable --- events that were yet to come. 

He was probably\ldots Going Crazy.
\end{quote}

Friday; now, he hadn't seen O. for several days, though they were still attending the very same university. The girl which always sat down next to him was there, again, and Y. would still take another route; thus, he was making friends with that girl, or rather the other way round, as his profiling system gave him the possibility to please the most complicated characters without having to think about all the words he'd have to choose. One may now ask, why this hadn't worked with O.; well, it had, but they were simply becoming very good friends, as he'd never experienced anything more till then.

And he hadn't done so up to today, and probably, it would take him more than a year to do so, while it could also happen the very next week. He wouldn't know, and it was probably better not to try to find out.

That very day, he'd been in contact with all those people he knew and loved, with the only exception of O., G. and O.'s best friend. However, he didn't feel sad. When he was finally alone in the bus again, having talked to one of P.'s friends before, thinking once more that somebody liked him more than a friend would do, he was able to have a look at O.'s house and her car again; having slept some time before in the bus, he'd just come awake some minutes before, without having planned to do so; this proved once more that it was something fate had intended to do with him. But this time, something was different: He'd still look for the car and that building, but now, he wouldn't feel sad about it all anymore; instead, he thought that she was probably together with her boyfriend and enjoying herself, while he was pretty sure she was alone at that moment. The thing that was different was the fact that he seemed to be going through those psychologic cycle of dealing with a broken love: Sadness, hate --- and finally accepting what was happening, while the last step would be --- if it was a healthy and real love --- more than acceptance; it would be the thing one only knew from films, the reality of wishing that beloved person good luck with the partner. Love turning into hatred and then into love again --- there it was again, that structure of a wave. Nevertheless, he'd not believed that theory of psychologists for long, as he couldn't agree with the hatred and the final love, believing that it would be something in between; but the amplitude of that wave could vary, and that was what he'd learned by now.

And something else would result from that story up to now: It meant that love was \emph{dynamic}, as dynamic as logical thinking, while it would be the thing to block the logic itself; that was to say, that one could love nearly everybody, as long as the environment and the events would lead to such a development, while it also assumed that no love could truly last forever, unless the people in question were isolated. As that was unlikely to happen, and as these facts were impossible to accept, but known by everybody somehow, they were one of the concealed secrets of society. The horrors of life that would disillusionize ourselves, as they could steal the foundation of our existence, the only thing we could use as our sense of life; as if there was no \enquote{Mr. or Mrs. Right}, one would not have to start searching; thus, one would have to do --- nothing. And being bound to do nothing is being bound to die. For that reason, we could never accept all the facts there were, and believe in the perfect love, probably thus realizing the illusion of it; and finally, it was probably possible to achieve a perfect illusion, and as thoughts were reality, we could probably realize this illusion. This was the last hope there was for the disillusionized, and nobody would take that away.

There was always the chance of changing the imperfect reality, even if we could only do so for a limited time and in a limited area. 
That was our creed, the belief of the crazy people who wished to understand life, universe and all the rest out there.

He was going crazy, of course, as he felt all the world developing around him, and was trying to be into everything, grabbing such a big amount of information that he was close to being incapable of dealing with it; but it gave him insight, and he wished to learn more, though he knew that this was probably not the best thing to do. However, this was the source of all understanding: The progress, thinking about things in different ways, joining all the things one knew to find out about the foundation of fact.

He felt, that the way he was accepted by so many people was something special; his profiling system was the basis of his studies, and without having developed such a thing subconsciously, he'd never have tried to find out about the things he knew now. Sometimes, he wished he hadn't, but then, he knew that life would have been the same way, then: The only difference being his ignorance, when he wouldn't know about the "wave of zero" or something like. He wondered whether his life would have been happier then, but this was to be impossible, if his theory was true; and he feared it was. His childhood among all those silly people, mocking him, was the foundation of his system, and he thought in terrible wonder, that those \enquote{false friends} had given him more than anybody else could do. He was slowly dealing with the past, and he felt new stability closing in.

He wouldn't know what was to come, and he was incapable of understanding what was really going on, not realizing what P., her friend(s) or O. had in mind. 
A weekend of silence was to come, though he'd probably be able to contact P. and her friend(s) at least several times; but as time's running out again, we'll have to wait until enough events have took place. 
I don't want to bore you with too many details which may seem to be of no interest to anybody. 
But nevertheless, you can give me your opinions\ldots

\begin{quote}
Crazy \\
was the word \\
to describe the foundation \\
of life itself; \\
and crazy was the word \\
to describe the people \\
who wished to \emph{know}. \\
--- W.G.
\end{quote}

\begin{quote}
Knowledge \\
is the foundation of blissfulness, \\
is making us happy, \\
is hard to endure, \\
is the basis of all hardships. \\
Knowledge \\
is the amplitude \\
of the wave of life. \\
Knowledge \\
is everything \\
and nothing \\
at the same time. \\
--- W.G.
\end{quote}
