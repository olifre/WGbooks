\chapter{Strange Observations}
\label{cha:strange-observations}
\subsection*{Originally published: \DTMdate{2006-09-25}}
\begin{quote}
Hello; a steadily changing hello! 

Things had become even more complicated, but this was the way the world was developing itself all the time. If it would not work out in such a strange way, we should find not reason to live on\ldots

Bound to make strange observations.
\end{quote}

Really peculiar things had taken place; He would not be able to make out any reglarity except the single one that nothing could be predicted, really. And the more he tried to find the root of it, the further he was driven away and the more complicated things became. In a short: He was in the very midst of life.

Slowly was a word he'd rarely heard or used in the last days, and he would not use it in the days to come. His best friend had upset him in a way he couldn't tolerate, and he was close to choosing the way of complete seclusion, when his friend finally gave in the same moment he told him nobody else was to be dependent on him. In fact, most people asked him for loads of things, but he was happy to be able to help them; And his best friend had often been the same, though he was now searching for a more important way of controlling things.

And, he was looking for others to share this power.

But he was happy he knew his friend so well; Finally, he could change his mind. The hardships of doing so, however, made him think. The change that had been grabbing O.'s deepest emotions and her soul had also changed the others around him, afflicting his best friend's character, to. No doubt this meant he had to become absolutely independent from everybody in any ways he was still dependent on them.

But he had also realized that some sort of change was going on with himself: Life was simply rushing by, but he felt even more like an observer now, standing there to watch. Often, he noticed more than others did, and maintained control of his mind to a further degree than them. One may wonder whether this is the right way to live, because it also means you're clutching yourself so strongly you may not easily get carried away by destiny and the ideas of others, but up to know, he had been right most times.

He knew this was to change.

The most puzzling observation included O., again; In fact, there were several things he'd now found out about her. Still, he was hoping to realize more of this strange transformation, and she was the subject that he knew pretty well before this wave had grasped the souls of the people around him.

Suddenly, he could understand all those out there that committed so many, cruel crimes: Taking a weapon and running through a school; Raping women; Or simply stealing something from the shop next door. They all had understood part of this process of change, but due to some lack in their character or in the way they were treated by those around them, they couldn't deal with it. Their souls missed \emph{foundation}.

Strangely, with this sentence, such people had become normal to him; They were just part of the mess out there, the extreme counterpart of --- What, really? What kind of perfect happiness was there? There must be something like that, but the media wouldn't tell us. It was just the kind of people that were ignored who gave the foundation for those who'd otherwise make themselves not to be ignored anymore.

Everything seemed so plain now; But O. was still --- different. He had remembered something few people seemed to have thought about, maybe a result of the changing society resulting in individual seclusion: Her birthday.

As a result, she had invited him. He'd never have thought about it, and these few words ("Would you like to come?") hit him out of nothing. Maybe, there was still something orange inside her, but he realized she simply turned around a quarter of the 360 degress that make a circle so she didn't face him anymore. Another girl was talking to her, and he was bound to be silent, for she seemed not to show any interest in any way anymore.

The following hours, he still tucked in some of the information life was giving him: O. seemed sad, not really smiling anymore. G. and some others seemed to share this emotion. And, finally, she didn't notice the details of life anymore --- In fact, the further time was expanding in the (positive?) direction, fewer people were able to see through the mesh that would make us completely ignorant. 
No greeting with a smile; This had only been visible the very moment he had pronounced the words \enquote{Happy Birthday!}. 
No waving at him while passing by; Not even turning her eyes, watching out for \emph{anybody} being next to her.

She had really chosen the way of seclusion; He wondered how that would work out. But as he was invited, he seemed to be among the fifteen or so she still remembered. Maybe even less, he wouldn't know. 
But he was not to be as happy as the last year he had been with her; She was not the O. he had known, anymore. However, he was interested in the way he had changed. 
The way everybody had changed.

If he'd realize the foundation of this past, present and future change, he'd finally understand life and himself. But was he allowed to do so, and did he really want to find out? 
His mind was making up things; The last time, he had written about magic. Magic being still alive. At least, in the minds of some people. Was this change only to be controlled by magic? He was to find out, facing himself with several ancient ideas concerning the way to change the things and happenings around you. Though he didn't really believe it would work out (which was probably the right way to think of things as such), he'd continue trying. All those coincidences must have some reason.

Can you recall a moment when you knew something perfectly well before it happened? Even if it was not possible for this to be a deja-vu, because you already knew it would happen before it really did? 
We all know those moments, and if we don't, we've just missed them. This may be magic.

The second when you say the same thing the person next to you says, as synchronous as the people who have really studied this do when they translate movies; The moment when you know the \emph{exact} words one may pronounce the next second, or probably, the moment when you simply feel that you see things so plainly and forget about this vision that had been so clear not a minute before.

Is this the magic of our lives? How come we have lost control of this apparent connection between us humans and the universe?
How come we don't even know about it, though there are many examples that could make us realize? Don't we want things to happen we cannot explain so easily?

No, we don't want such things to happen. We don't want the time to pass when O. had been so close; We want to go back to the time when her soul was still orange, \emph{I} want to feel her sitting next to me, talking and smiling, our eyes meeting, sinking into our souls. But she seems gone, seems to have passed away, lost her magic in a blink of an eye.

If nothing may be eternal, why does this word exist? Is it just the ideal we long to hold, an image to wake our sorrows? 
How dare we judge the world when we don't even know who we are and which power dominates ourselves?

He began organizing things, sorting some of the mess out, creating a new kind of mess, somehow. Nay, we cannot understand ourselves, because we don't want to lose the impression that we are products of magic, having killed everything that may have been magic in the world. It's like the idea of fairies dying when you do not believe in them; If we don't believe in the magic of our very own souls, we shall die, too. 
And if not so easily, our souls will be the subject to this killing idea.

Being our own death-machinery, we shall try to take a step back and watch, all of us. But we must not forget to look inside ourselves, too. 
I'll continue to watch and search for explanations, even if they seem too fantastic and impossible. Try this for yourself, for I'll once more take a short break now --- probably. But the birthday-party may be the trigger of more events, for there may be O., G., o.'s best friend, and Y. --- probably. Let's lean back and see.

\begin{quote}
A man was standing there. \\
In the rain, wet all through. \\
Watching. \\
Contemplating. \\
Counting the raindrops hitting the road \\
next to him; \\
They knew more about these drops \\
than about their very selves; \\
So he was bound to stay there, \\
destined to get a cold, \\
because he was willing to \emph{know}. \\
But he knew he'd never find out \\
more about himself \\
than about the tiny drops of water \\
hitting the road \\
next to him. \\
--- W.G.
\end{quote}

\begin{quote}
Let life be a bottle of wine: \\
Somewhere inside, there would be \\
the answer; \\
But if you drink the wine, \\
you will be too drunk to understand. \\
But you must do so \\
or continue sharing \\
the common ignorance \\
of man. \\
--- W.G.
\end{quote}

%%% Local Variables:
%%% mode: latex
%%% LaTeX-command: "latex -shell-escape -synctex=1 --file-line-error-style"
%%% ispell-local-dictionary: "british-ize-w_accents"
%%% TeX-master: "../Wilkie_Goldentongue_-_My_Life_in_a_Nutshell_-_Second_Era.tex"
%%% End:
