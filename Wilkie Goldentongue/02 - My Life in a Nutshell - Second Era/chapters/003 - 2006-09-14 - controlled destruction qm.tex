\chapter{Controlled Destruction}
\label{cha:contr-destr}
\subsection*{Originally published: \DTMdate{2006-09-14}}
\begin{quote}
Hello once more! 

Today, some things came for a surprise, at least, some --- probably foolish --- thoughts. You shall be able to find out for yourself when continuing reading\ldots

Was the world suffering from controlled destruction?
\end{quote}

This was another day, just in the middle of life. And the moment he decided to continue writing was caused by O. once more --- probably for one of the last times. 
He had seen her that day, though he hadn't, for it certainly wasn't O. anymore. She had changed completely, as he had to realize right now.

Her colours were gone, and he could not really make out anything special about her anymore, as he could have done so easily some weeks before. In fact, he had already described her loss of expression in the last era, but it had never been as radical as this day\ldots

He didn't recognise her for a few seconds when he saw her. She smiled more rarely, and her clothes --- they were pretty similar to those of L.-B., and all of her natural beauty seemed gone for she had betrayed her own aura. Maybe, this was gone, too\ldots

Right now, she looked like one of these girls that would stay at home for the rest of their lives to care for a husband --- The same thing L.-B. seemed to want to do. Where had O.'s vividness gone? 
This was the moment when he began to think about controlled destruction. Such a sudden change could not simply happen by chance, and even fate could not make people act without any logic. Her nickname was shortened and she had lost individuality, lacking everything that had made her so special once. This seemed not to be so temporary a thing. It really seemed as if his planned forgetting about her had caused something unnatural to happen; Still, he could not allow himself to believe that his thoughts had changed reality, but\ldots

If one really thought about it, having the theory in mind that all our conscience was just a result of the reaction between an unknown sort of energy and our brains, and if these parts of energy were mobile, than it could be explained. O. was still out there, but she had been eradicated from the creature he had once loved; This meant that the change in character would result in a sort of physical change, too, for energy was matter and the other way round. But finally, this would mean that all of us could change the visible reality \textbf{and} all the people around us by controlling this energy. The idea itself seemed logical, but without any rational foundation.

Most things in modern sciences were like that, though. One could be right without being able to prove this to anybody else, and somebody else with a proof could be proved wrong. 
At least, this theory would explain all the things our sciences couldn't: Magic was something mankind had forgotten about. Was it really out there, just under our very noses?

He wondered where O. was gone, but this strange theory gave him hope that she was still somewhere out there. The things he'd copied from her had probably not only been copies, but the real things, then\ldots

Could two become one by just being united in sympathy? The idea seemed wonderful, and it would explain so many things, while it would also give reason to hope, as energy was then conscience and matter, and energy could only be transformed, but never killed or destroyed.

The next step of his thoughts was the very beginning of the complete story: When he had first met O., had he just managed to remember her first words without knowing her because some invisible stream of particles formed by energy were mixing between them, feeling the longing for unity even before they knew each other? Was love then just the expression of this process, and the end of this exchange the end of individuality which must inevitably lead to breaking up? If so, then his idea could explain even more.

The entropy of the universe was rising steadily, and if the universe was a representation of \enquote{his} energy, then the entropy of it must be found rising, too. Then, this meant that not only the complexity was growing, but also the instability. In a sudden burst of energy or in a small wave, the particles could be thrown around, and a character could change within seconds. The parallel changes all around him made him think so, while this theory could also be all rubbish. However, it was the only thing he had for O. was gone, and her presence --- or at least her existence --- had given him the feeling that there was somebody out there who was like him and would accept his strange being. She was the first person he'd really have allowed to open the shell. When she was gone, he had to find a substitute, and if not human, then it must be theoretical.

His vision was growing steadily. His best friend was still changing, and something he proposed him was not accepted; This had happened quite often before, but he had nearly always been able to get to the root of this denial. This time, he could not find it.

If we took the idea that we were defined by others for true, then one would absorb this strange energy of others. Those who were always thinking, trying to sort things out and at the same moment searching for ways to communicate with this strange flow of energy, would then be the people that realized their change and the changes of the others. These would be the ones that would try to control it, playing with something they would not even know.

Was this really true? O. could never have changed so quickly without something \emph{very} special going on, and as far as he knew, there was no such thing. However, he had still been watching for signs, and they had approached him. The plate of the car behind the one he was driving showed a short, encrypted version of O.'s new, shortened nickname and the date of the 16th of September. He wondered what was to happen on this day, at the same time realizing that O.'s birthday was to arrive soon. Even if there was a kind of party, he'd never believe that she'd invite him.

What's for sure is that this month will be very interesting, and that we all shall know more about ourselves, life and everything else by the end of it. While these thoughts were controlling his mind, he felt denied by the world for a knowledge he didn't even know about. With a smile, he recalled the scene in the \enquote{Hitchhiker's Guide to the Galaxy} when you get to know that the earth is a big kind of biologic computer to find the question with the answer "42". The controlled destruction of this planet that had been rendered useless long before was a kind of parallel description of his current ideas. Destiny would be out there, blocking the path to wisdom using many different ways of doing so. If something happened nearly every time, it was regarded a law of physics; But was this just another idea to stop our progress to the wisdom which would destroy ourselves? Was the basis of everything really so stable that we could predict it, and at the same time so instable we could never put the finger on it? Fate and destiny had done a good job confusing us human beings.

This leaves us with the question: Are we really more \emph{advanced} than the people that built \enquote{Stonehenge}? Did they know things we can't understand? 
But I'm getting off track. I was just going to say that most things we regard as normal and secure are certainly not this way. How can we regard ourselves as the masters of the earth when we don't know most of it? And how dare we decide what is to happen with \emph{our} planet and ourselves?

It was just an idea, a seemingly senseless theory --- the way most things began. The theory of strings had been found in an old book of mathematics by chance, and chances are that it's correct --- chances are it isn't.

We should not simply regard the world as logical or based on logic, for we must know better: We can only explain something about ten percent of the universe using our logic. The rest is unknown, and it may be that somewhere out there --- somewhere behind a star or what we take for a star --- God is hiding and watching us. Or a stream of energy that makes our yet inexplicable conscience, or aliens, or another earth, parallel universes, copies of ourselves, people that know how the universe works; or, somewhere out there, there may be the O. he was searching for. The O. he had loved, the creature that had been gone so quickly.
At least, in this universe and this time.

And time is the power that may show if any of these words contain any truth, and if somebody like O. may return in some way. 
For now, we can once more just wait and watch the things going on, discussing their causes and effects without understanding them. 

\begin{verse}
If conscience be a ball of energy \\
and energy be matter \\
can we grasp the key to the world? \\
Would we even try to do it, \\
leaving our current lives behind \\
for something without aim, \\
just knowledge? \\
Would it be in vain? \\
--- W.G.
\end{verse}

\begin{verse}
A star was rising; \\
it's existence, \\
and even so more ours, \\
had been the result \\
of more than a dozen constants; \\
if only one was wrong, \\
all would be gone. \\
Who could predict it? \\
Can we call it \enquote{chance}? \\
And finally: \\
Is it positive? \\
--- W.G.
\end{verse}

%%% Local Variables:
%%% mode: latex
%%% LaTeX-command: "latex -shell-escape -synctex=1 --file-line-error-style"
%%% ispell-local-dictionary: "british-ize-w_accents"
%%% TeX-master: "../Wilkie_Goldentongue_-_My_Life_in_a_Nutshell_-_Second_Era.tex"
%%% End:
